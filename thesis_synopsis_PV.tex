%% ----------------------------------------------------------------
%% Thesis.tex -- MAIN FILE (the one that you compile with LaTeX)
%% ---------------------------------------------------------------- 

% Set up the document
\documentclass[a4paper, 12pt, oneside]{Thesis}  % Use the "Thesis" style, based on the ECS Thesis style by Steve Gunn
\graphicspath{Figures/}  % Location of the graphics files (set up for graphics to be in PDF format)

% Include any extra LaTeX packages required
\usepackage[square, numbers, comma, sort&compress]{natbib}  % Use the "Natbib" style for the references in the Bibliography
\usepackage{verbatim}  % Needed for the "comment" environment to make LaTeX comments
\usepackage{vector}  % Allows "\bvec{}" and "\buvec{}" for "blackboard" style bold vectors in maths
\usepackage[utf8]{inputenc}
\usepackage{aas_macros}
\usepackage{amsfonts}
\usepackage{amsmath}
\usepackage{amssymb}
\usepackage[normalem]{ulem}
\usepackage{graphicx}
\usepackage[dvipsnames]{xcolor}
\usepackage{hyperref}
\usepackage{soul}
\hypersetup{colorlinks=true,allcolors=teal}  % Colours hyperlinks in blue, but this can be distracting if there are many links.

% \hypersetup{colorlinks=true,allcolors=teal}

\newcommand{\doi}[1]{\href{https://doi.org/#1}{doi:#1}}
\newcommand{\Hi}{\textsc{Hi}}
\newcommand{\mHi}{\ensuremath{m_{\Hi}}}

\newcommand{\p}{\ensuremath{\partial}}


\newcommand{\Msun}{\ensuremath{M_{\odot}}}
\newcommand{\Mh}{\ensuremath{h^{-1}M_{\odot}}}
\newcommand{\Mhsq}{\ensuremath{h^{-2}M_{\odot}}}
\newcommand{\Mpch}{\ensuremath{h^{-1}{\rm Mpc}}}
\newcommand{\kpch}{\ensuremath{h^{-1}{\rm kpc}}}
\newcommand{\kms}{\ensuremath{{\rm km\,s}^{-1}}}
\newcommand{\msq}{\ensuremath{{\rm \,m\,s}^{-2}}}

\newcommand{\avg}[1]{\ensuremath{\left\langle \,#1\, \right\rangle}}
\newcommand{\e}[1]{\ensuremath{{\rm e}^{#1}}}

\newcommand{\der}{\ensuremath{{\rm d}}}
\newcommand{\Der}{\ensuremath{{\rm D}}}
\newcommand{\dir}{\ensuremath{\delta_{\rm D}}}

\newcommand{\erfc}[1]{\ensuremath{{\rm erfc}\left(#1\right)}}
\newcommand{\erf}[1]{\ensuremath{{\rm erf}\left(#1\right)}}

\newcommand{\eqn}[1]{equation~\eqref{#1}}
\newcommand{\eqns}[1]{equations~\eqref{#1}}
\newcommand{\eqnref}[1]{equation~\eqref{#1}}
\newcommand{\eqnsref}[1]{equations~\eqref{#1}}
\newcommand{\ph}[1]{\phantom{#1}}
\newcommand{\figref}[1]{Figure~\ref{#1}}
\newcommand{\tabref}[1]{Table~\ref{#1}}
\newcommand{\secref}[1]{Section~\ref{#1}}
\newcommand{\appnref}[1]{Appendix ~\ref{#1}}

\newcommand{\be}{\begin{equation}}
\newcommand{\ee}{\end{equation}}
\newcommand{\Cal}[1]{\ensuremath{\mathcal{#1}}}

\newcommand{\AP}[1]{\emph{\color{blue}[AP: #1]}}
\newcommand{\PV}[1]{\emph{\color{violet}[PV: #1]}}
\newcommand{\red}[1]{\textcolor{red}{#1}}

%% ----------------------------------------------------------------
\begin{document}
\frontmatter      % Begin Roman style (i, ii, iii, iv...) page numbering

% Set up the Title Page
\title  {Interplay of galaxy formation and the evolution of dark matter haloes in the cosmic web}
\authors    {\texorpdfstring{\href{premv@iucaa.in}{Premvijay Velmani}}{Premvijay Velmani}}
\addresses  {\groupname\\\deptname\\\univname}  % Do not change this here, instead these must be set in the "Thesis.cls" file, please look through it instead
\date       {\today}
\subject    {}
\keywords   {}

\maketitle
%% ----------------------------------------------------------------

\setstretch{1.5}
\mainmatter	

\chapter*{Synopsis}
\section*{Background}
Dark matter is inferred exclusively through its gravitational effects to make up over 80\% of all the matter in the Lambda-cold dark matter ($\Lambda$CDM) picture of the Universe. Tiny fluctuations in the initial density field causes them to clump into gravitationally collapsed structures called haloes \citep[][]{1974ApJ...187..425P,2002PhR...372....1C}. They contain crucial information about the cosmology as they form the fundamental building blocks of the large scale structure, and also the particle physics of the dark matter. The dark matter haloes have been extensively studied in gravity-only scenarios both in isolated systems and over cosmological volumes. For example, the haloes identified in cosmological $N$-body simulations on $\Lambda$CDM cosmology, are known to be triaxial \citep[][]{1988ApJ...327..507F}, and their sphericalised mass profiles are found to have a universal form \citep{1996ApJ...462..563N,1997ApJ...490..493N,2010MNRAS.402...21N}.

The dark matter haloes also provide the primary environment for the formation and evolution of the galaxies \citep[e.g.,][]{wr78} and hence a key ingredient in understanding the astrophysics of galaxies. In order to form galaxies, we need gas with their various non-gravitational baryonic interactions that can cool and condense towards the centre \citep{1988MNRAS.234..459S,1998MNRAS.295..319M}, eventually producing stars and other astrophysical bodies. However, the lack of complete knowledge and the complexity of those baryonic processes, along with their large dynamic range make it impossible to study the formation of galaxies directly from first principles using cosmological simulations. 

One approach in studying the galaxy formation is to use semi-analytic techniques to model its formation within the halo, using the properties of the halo known from gravity-only simulations. These semi-analytic models have a variety of free parameters that need to be constrained typically by confronting with observations \citep{2015ARA&A..53...51S}. The formation of galaxies and clusters of galaxies within these haloes driven by the baryonic interactions within the haloes, also affects the spatial distribution of the dark matter through the gravitational coupling. This response of a halo's dark matter content to the baryons it hosts must be accounted for to understand the coupled evolution of haloes and galaxies. 

The state-of-the-art approach to galaxy formation is to do numerical hydrodynamical simulations in cosmological volumes and use subgrid recipes for prescribing the baryonic physics at scales not resolved by the simulation. Even though some of the baryonic processes are included only as effective models with parameters to be constrained by observations, this approach is more realistic than the semi-analytical models in making predictions. This also includes the backreaction of the galaxies on the dark matter halo, but this approach is computationally challenging. Modern cosmological simulations of galaxies such as EAGLE \citep{2015Schaye_EAGLE}, IllustrisTNG \citep{2018MNRAS.480.5113M}, etc. were performed with high performance clusters over a period of several months.

Understanding in detail the mutual effects of the formation of dark matter halo and the formation of galaxy inside the halo on each other is relevant for a variety of problems. For example, the shapes of haloes respond to galaxy formation \citep{2010MNRAS.407..435A,2021MNRAS.501.5679C}, which are consequently important for weak-lensing studies \citep{2021A&A...647A.185G}; similarly the formation of the galaxy changes the dark matter density profile of the halo, which in turn affects the rotation curve of the galaxy. However, previous works have not yet reached a clear consensus on the backreaction of the formation of the galaxy on the properties of the halo in the simulations, and it is still a pressing open problem. 

From a cosmological viewpoint, these `baryonic' effects are often a nuisance in parameter inference; e.g., the effects of feedback due to an active galactic nucleus (AGN) can be degenerate with the effects of a massive neutrino species or a thermally produced `warm' dark matter candidate at Fourier scales $k\sim1\,h\,{\rm Mpc}^{-1}$ (see, e.g., \cite{2019Chisari_etal_Baryfeedback,2020AricoAnguloetal_baryonifi}).
More generally, the response of the dark matter (DM) to the presence of baryons first must be quantified and then distinguished from the observational effects of (non-standard) primordial physics. While the former problem has a long history \cite{1986Blumenthal,2004Gnesin_etal,2005SellwoodMcGaugh,2006Gustafsson_FS,2010Abadi_NFBS,2010DuffySchaye_etal,2010PedrosaTissera_etal,2010TisseraWhite_etal,2019ArtalePedrosa_etal,2022ForouharMoreno_etal,2023Velmani&Paranjape},
the latter has only recently begun to be studied in detail \cite{2011TeyssierMMDM,2015SchneiderTeyssier,2015Mead_PHJH,2020AricoAnguloetal_baryonifi,2021AricoAnguloetal_baryonifi,2023EuclidCastro_etal}.

\section*{Thesis statement}
In this thesis, we study the interplay of the galaxy formation and the dark matter evolution, with a primary focus on the changes in the radial distribution of the dark matter in response to the galaxy formation and evolution. We build a comprehensive understanding of this aspect of the halo response using both hydrodynamical simulations of realistic galaxies in the cosmological volumes and also using more tractable self-similarly evolving systems of individual galaxy with its host halo. This provides a unified picture of the role of various galactic astrophysical processes in mediating the response of dark matter haloes; and this also gives the timescales that are useful in predicting the dynamical changes in the radial mass profiles of the dark matter haloes due to astrophysical processes.


\section*{Overview}
\subsection*{Chapter 1}
Here we introduce the background material and the primary objectives of the thesis along with a review of the existing knowledge and relevance of this work in the broader context of the ongoing research in physics.

\subsection*{Chapter 2}
We describe the simulation data and the techniques that are primarily employed in this thesis. Currently, the most robust technique to understand the consequences of gas assembly and galaxy formation on dark matter structure is the use of high-resolution cosmological hydrodynamical (zoom) simulations, using `sub-grid' recipes for modelling very small-scale astrophysics such as feedback from stellar/supernovae activity or the effects of active galactic nuclei (AGN). In this thesis, we mainly use the publicly available data from such cosmological simulations of galaxies that were performed as part of large collaborative projects namely IllustrisTNG\cite{2019ComAC...6....2N}, EAGLE\cite{2015MNRAS.446..521S} and CAMELS\cite{CAMELS_presentation,CAMELS_DR1}. A brief description of those simulations and the identification of halo objects in their respective cosmological volumes is provided in this chapter, focusing on the details relevant in this thesis.

In the same chapter, we develop a matching algorithm to assign each of the haloes in the full hydrodynamic simulations with a matched `partner' halo in a collisionless, gravity-only simulation performed using the same initial random fluctuations. We also demonstrate its robustness since a comparison of those hydrodynamically simulated haloes against their partner haloes is key to the investigation of the simulated backreaction of the galaxies on the dark matter haloes. 

Early work have modelled the halo response using adiabatic invariants within individual haloes
\citep[][]{osti6457593,1984MNRAS.211..753B,1986ApJ...301...27B,1987ApJ...318...15R}. 
Using simplifying assumptions such as spherical symmetry, no shell crossing, angular momentum conservation with circular orbits for dark matter particles, a simple formula has been derived to quantify the \emph{adiabatic relaxation} of the radial distribution of the dark matter in terms of the radial distribution of the baryons (we discuss this in detail later). This idealized model of the adiabatic relaxation \citep[][]{1986ApJ...301...27B} was found to be an inaccurate description of the response in a variety of simulations \citep[see, e.g.,][]{2004ApJ...616...16G,2006PhRvD..74l3522G,2010MNRAS.402..776P,2010MNRAS.406..922T,2010MNRAS.405.2161D,2010MNRAS.407..435A,2011MNRAS.414..195T,2016MNRAS.461.2658D,2019A&A...622A.197A,2022MNRAS.511.3910F}. This has led to the development of various models that are direct extensions of the idealised model with additional parameters constrained by the simulated response \citep{2010MNRAS.407..435A,2004ApJ...616...16G,2006PhRvD..74l3522G,2010MNRAS.405.2161D}.

We characterize the halo response within this quasi-adaibatic relaxation framework, primarily through the relation between relaxation ratios and enclosed mass ratios, that quantify the changes in sphericalised mass profiles of the dark matter and baryons respectively. We conclude this chapter, with a description and demonstration of our techniques employed in computing these quantities in simulated pairs of haloes. 



\subsection*{Chapter 3}

In this chapter we perform a systematic, statistical study of the dark matter response to galaxies in high-resolution hydrodynamical simulations incorporating realistic feedback and quantify it using simple analytical forms, including the sensitivity of this response to halo-centric distance and halo and galaxy properties. To this end, we use the publicly available suites of simulations from the IllustrisTNG and EAGLE projects. 


Physically, one expects that the overall response of the halo is mediated by a combination of different astrophysical processes that occur in the galaxy. Feedback processes are known to reduce the contraction of the halo significantly; e.g.,
supernova-driven winds can completely transform the inner density profile of the dark matter halo
\citep[][]{1996MNRAS.283L..72N}. This may be the key in reconciling the observation of dark matter cores at the center of various galaxies with the cuspy haloes found in gravity-only $\Lambda$CDM simulations \citep[see][for a review]{2014Natur.506..171P}.
However, such feedback effects do not always produce dark matter cores from cusps, rather, this can depend on the amount of gas ejected, the mass loss time scale and the frequency of starburst events 
(see, e.g., \citealp{2011ApJ...736L...2O,2014ApJ...793...46O,2012MNRAS.421.3464P}, and also \citealp{bfln18}).
In massive haloes hosting galaxy groups or clusters, while the formation of powerful AGN in the central galaxy can strongly suppress star formation,
it can still significantly reduce the adiabatic contraction of the halo \citep[][]{2011MNRAS.414..195T}.
Moreover, the fluctuation in gravitational potential due to such feedback can expel the dark matter from the inner halo producing inner cores \citep[][]{2012MNRAS.422.3081M}.


In our earlier work \cite{2023Velmani&Paranjape} we quantified this response in the IllustrisTNG \cite{2018TNG_Pillepich_etal}
and EAGLE \cite{2015Schaye_EAGLE} %
simulation suites by studying the mass profiles of matched pairs of halos in the hydrodynamical and gravity-only runs of the simulation, and described the response in the language of quasi-adiabatic relaxation \cite{2011TeyssierMMDM}. %
In particular, we showed that the original quasi-adiabatic prescription accurately describes the relaxation of halos in simulations provided we include the halo-centric distance in units of the halo radius as an additional parameter. We further provided convenient fitting forms for this parametrised relaxation at $z=0$ and explored its dependence on various halo and galaxy properties such as halo concentration, stellar mass, star formation rate, etc.



\subsection*{Chapter 4}

\subsection*{Chapter 5}

\subsection*{Chapter 6}

The most realistic description of halo response or `relaxation' in the presence of baryons is provided by hydrodynamical simulations that include calibrated effects of star formation, radiative cooling, feedback due to stellar winds and supernovae as well as the triggering of AGN activity in super-massive black holes and its feedback on star formation \cite{2015Schaye_EAGLE,2018TNG_Pillepich_etal}. 


Physically, however, the mechanisms driving this halo response are not entirely clear. In principle, this question can also be addressed using hydrodynamical simulations by systematically varying the physical prescriptions defining different processes, particularly the observationally uncertain ones associated with stellar and especially AGN feedback. This, however, is a very computationally expensive exercise requiring multiple runs of expensive simulations \cite{2015Schaye_EAGLE,2021camels_presentation}.
We will report the results of such an exercise using the CAMELS simulation suite \cite{2022camels_data_release1} in a separate publication. In the present work, we investigate whether we can gain some physical insights into the problem of halo relaxation due to baryonic backreaction using a simplified, spherically symmetric toy model of galaxy evolution.

Our key assumption will be that the model is `self-similar', so that spatial variations are related to temporal variations through a scale radius that evolves over time. The self-similar assumption offers a powerful tool to simplify the problem and render it more tractable, by converting nonlinear coupled partial differential equations into ordinary differential equations. The self-similar model was pioneered by Fillmore and Goldreich (1984) \cite{1984FillmoreGoldreich} and Bertschinger (1985) \cite{1985Bertschinger}. While Bertschinger devised spherical self-similar solutions for the collapse of dark matter and shocked gas within the Einstein-de Sitter (EdS) universe, under the assumption of a constant accretion rate, Fillmore and Goldreich explored self-similar solutions for the collapse of dark matter from various initial mass distributions, expanding their purview to include not only spherical but also cylindrical and planar self-similar collapses. Bertschinger (1989) \cite{1989Bertschinger} delved into self-similar cooling flows of gas, focusing on the inner regions of galaxies. Owen et al. (1998) \cite{1998OwenWeinberg_etal} derived a set of cooling functions that guarantee self-similar evolution by ensuring that the cooling time-scale of an object with a characteristic clustering mass remains a constant fraction of the Hubble time. Abadi et al. (2000) \cite{2000Abadi_etal_SelfSimCool} studied the self-similar accretion of gas with radiative cooling in the EdS Universe. Shi (2016a) \cite{2016ShiDMLamCDM} generalised self-similar models to Lambda-cold dark matter ($\Lambda$CDM) models, focusing on the outer profile and, notably, the outermost caustic or `splashback radius' of dark matter collapse which has gained recent popularity \cite{2014DiemerKrastov,2014AdhikariDalalChamberlain,2018Changetal_DES_splashback}.
In a complementary study, Shi (2016b) \cite{2016ShiICM} probed the self-similar accretion of shocked gas, dissecting its behavior with respect to accretion rates and revealing correlations between the shock radius of gas and the dark matter's splashback radius.


The present work draws upon these foundations but incorporates two novel additions: (i) a self-similar cooling of the gas and (ii) the formation of a pseudo-disk of gas along with the self-similar collapse of the dark matter halo. Unlike the work of \cite{1989Bertschinger}, where the cooling scale was set by a cooling radius and was primarily applicable to the innermost regions, in our work we set the cooling scale relative to the turn-around radius. This allows us to keep intact the global self-similarity of the model, while allowing for a range of possible amplitudes for the cooling rate. Our inspiration for including the formation of a pseudo-disk in our spherical approach comes from spherical hydrodynamical simulations that have previously explored the influence of cooling on shock-heated gas, tracing its subsequent accretion onto central structures resembling galaxy disks \cite{2006Dekel&Birnboim}. These simulations have also revealed a `cold mode' of gas accretion that occurs in low mass halos, wherein the gas directly accretes onto an inner disk without shocking. While these results are consistent with those of full hydrodynamical simulations \cite{2005Keres_KWD},
the main relevance to us is the fact that the inner disk radius undergoes an approximately self-similar evolution, following the behavior of the turnaround radius, thus rendering it amenable to our self-similar modeling.

\emph{Importantly, we self-consistently solve for the evolution of the gas and the dark matter simultaneously using a novel iterative technique.} This approach allows us to reliably estimate the relaxation of the dark matter due to baryonic backreaction in this self-similar model. The inclusion of cooling and the formation of a pseudo-disk also brings our self-similar model of galaxy formation substantially closer to the more realistic results of hydrodynamical simulations. We leave the incorporation of the formation of stars and a central black hole, and the associated stellar and AGN feedback, to future work.


\subsection*{Chapter 7}
We conclude this thesis with a brief summary of the key findings and their applications. We also discuss the future directions to this work in this chapter.


\section*{Organization of the thesis}

In the first chapter of this thesis, we review the background material and then define the primary objective of this thesis, along with its relevance to the broader context.

In the second chapter, we demonstrate the specific techniques we employed in studying simulated response of the dark matter halo to the galaxy formation and evolution in simulations of cosmological volumes.

In the third chapter, we present our findings from state-of-the-art galaxy formation simulations, IllustrisTNG and EAGLE simulations, characterizing the dark matter response in a wide variety of haloes. We present novel fitting functions that accurately describe this relaxation response, revealing an additional dependence on halo-centric distance and highlighting the significant role of star formation-related feedback processes. 

In the fourth chapter, we explore the role of astrophysical modeling in producing the halo response at different epochs in simulations. We find the gas equation of state to have significant influence in EAGLE haloes, and we also quantify the strong influence of AGN and stellar feedback in CAMELS simulations. Our results also show a more universal relaxation response at an earlier epoch $(z=1)$ compared to $z=0$. These findings are applicable to semi-analytical tools for modeling galactic and large-scale structures. 

In the fifth chapter, we uncover the causal connections between star formation activities, feedback processes, and halo relaxation. Through time-series analyses we provide new insights into the immediate and delayed effects across different halo regions. Our estimates of the time-scales can potentially improve the description of halo profiles in existing baryonification schemes and semi-analytical galaxy formation models. 

In the sixth chapter, we develop a spherical self-similar model for galaxy formation that simultaneously and self-consistently solves for the evolution of gas and dark matter, producing pseudo galaxy disks within the halo. This complementary approach offers a framework to rapidly explore the sensitivity of halo response to various astrophysical processes. We find quasi-adiabatic halo relaxation similar to full non-linear simulations, with similarly strong trends over the gas equation of state.

In the seventh chapter, we summarize the key findings in this thesis and discuss the future outlook.

% \section*{Publications}
\section*{Publications included}

\noindent
1. Velmani P., Paranjape A., 2023, "The quasi-adiabatic relaxation of haloes in the IllustrisTNG and EAGLE cosmological simulations", \textit{Monthly Notices of the Royal Astronomical Society}, 
\textbf{520}(2):2867-2886. 
\doi{10.1093/mnras/stad297}
\href{https://ui.adsabs.harvard.edu/abs/2023MNRAS.520.2867V}{https://ui.adsabs.harvard.edu/abs/2023MNRAS.520.2867V}

\vspace{0.5cm}

\noindent
2. Velmani P., Paranjape A., 2024, "Dynamics of the response of dark matter halo to galaxy evolution in IllustrisTNG", \textit{arXiv e-prints}, 
\href{https://ui.adsabs.harvard.edu/abs/2024arXiv240708030V}{arXiv:2407.08030}. 
\doi{10.48550/arXiv.2407.08030}

\vspace{0.5cm}

\noindent
3. Velmani P., Paranjape A., 2024, "A self-similar model of galaxy formation and dark halo relaxation", \textit{Journal of Cosmology and Astroparticle Physics}, 
\textbf{2024}(5):080. 
\doi{10.1088/1475-7516/2024/05/080}
\href{https://ui.adsabs.harvard.edu/abs/2024JCAP...05..080V}{https://ui.adsabs.harvard.edu/abs/2024JCAP...05..080V}

\vspace{0.5cm}

\noindent
4. Velmani P., Paranjape A., 2024, "Role of astrophysical modeling on dark matter halo relaxation response at redshifts z = 0 and z = 1", \textit{in prep.}


% % Include the bibliography
% \bibliographystyle{plain} % You can choose another style such as plain, abbrv, unsrt, etc.
% \bibliography{publications} % 'publications' is the name of your .bib file without the extension



\bibliographystyle{abntex2-num}
\bibliography{Bibliography,references-AdiabRelxn1,references-SelfSimRelxn}


\end{document}  % The End
%% ----------------------------------------------------------------