\chapter{Conclusion}
\label{chap:conclusion}


% \chapter{Conclusion}

\section{Summary of Key Findings}

This thesis presents a detailed investigation into the dynamical evolution of dark matter haloes in response to galaxy formation and evolution, utilizing both cosmological hydrodynamical simulations and a novel self-similar model. The core findings of this research provide significant insights into the complex interplay between baryonic processes and dark matter dynamics.

\subsection{Dark Matter Relaxation in Simulations}
We have explored in detail the response of the dark matter content of a halo to the galaxy and gas it hosts. Understanding and accurately modelling this response is important for a number of applications including baryonification schemes for small-scale power spectrum emulation, rotation curve modelling, %
constraining the nature of dark matter using inner halo mass profiles, etc. 

Using haloes and galaxies identified in the IllustrisTNG and EAGLE simulations and matched to their gravity-only counterparts, our analysis demonstrates that the simplified analytical schemes used thus far to model the dark matter response \citep[e.g.,][]{1986ApJ...301...27B,2010MNRAS.407..435A,2015JCAP...12..049S} are inadequate in describing its detailed behaviour across a variety of halo and galaxy types. Specifically, we showed that the dark matter response, or relaxation relation (see equation~\ref{eq:qAR}), which connects the relaxation ratio $r_f/r_i$ to the mass ratio $M_i/M_f$ between unrelaxed (gravity-only) and relaxed (hydrodynamical) haloes, explicitly depends on halo-centric distance $r_f$ in the relaxed halo, in addition to being sensitive to a number of halo and galaxy properties including halo mass, halo concentration, stellar and gas mass fraction, and specific star formation rate. These effects, especially the dependence on halo-centric distance, have been typically neglected by existing quasi-adiabatic relaxation models. 

We presented a simple, physically motivated extension (equation~\ref{eq:chi-linear-q0}) of the existing models which accurately captures the dark matter response over 4 orders of magnitude in halo mass ($10^{10}\lesssim M/(\Mh)\lesssim 10^{14}$) and $\sim2$ orders of magnitude in relative halo-centric distance ($0.02\lesssim r_f/R_{\rm vir}\leq1$). Apart from an explicit radial dependence of the relaxation relation (e.g., equation~\ref{eq:q3-model} for low-mass haloes), a second novelty of our model is the inclusion of a parameter $q_0$ which characterises feedback-induced offsets seen in the relaxation relation measured in IllustrisTNG and EAGLE haloes in which, e.g., shells that do not show an overall change in radius ($r_f/r_i\simeq1$) nevertheless have $M_i/M_f>1$ (indicating loss of baryonic material). The existing quasi-adiabatic relaxation models do not allow for the existence of such shells, which are however captured well by our new null-offset parameter $q_0$ (see \secref{subsubsec:sim-relax} for a detailed discussion).
We argued that our results could have a significant impact on the applications listed above.

One of the key parameters, the relaxation offset parameter \( q_0 \), quantifies the excess relaxation of dark matter shells. We found that \( q_0 \) is typically stronger following the peak star formation epoch for a given population of haloes, indicating a strong correlation between star formation activity and halo relaxation.

\subsection{Influence of Astrophysical Processes}
We investigated the influence of astrophysical modeling on the relaxation response of dark matter haloes at different epochs, specifically focusing on \( z=0 \) and \( z=1 \). The analysis is divided into three main parts, each shedding light on the role of various astrophysical processes in shaping the dark matter content of haloes.

\subsubsection*{Early Epoch in IllustrisTNG Simulations}
We began by examining the relaxation response at an earlier redshift (\( z=1 \)) using the IllustrisTNG simulations. Our study reveals that dark matter relaxation tends to be usually stronger at the earlier epoch compared to the present. We assess this using three distinct set of halo samples at \(z=1\), which highlight the variations in relaxation across different halo masses. Notably, we observe that cluster-scale haloes at \( z=1 \) show significant relaxation that becomes a function of the change in enclosed mass, in contrast to similar haloes at the present epoch.

We find that the relaxation relation at this earlier epoch can be described using the same locally linear quasi-adaiabtic model built on the analysis at \(z=0\), demonstrating the robustness of this approach in capturing the dark matter response across redshifts. Moreover, the parameters of the radially dependent relaxation are found to be more universal across a much wider range of masses at \( z=1 \). For example, the progenitors of even the most massive clusters are well characterized by the simple three-parameter model of relaxation that was developed with a focus on galactic-scale haloes at \( z=0 \).

\subsubsection*{Variation in Astrophysical Feedback Using CAMELS Simulations}
Next, we explore variations in astrophysical feedback strengths within the IllustrisTNG model using simulations from the CAMELS project, which varies four different feedback parameters: two for stellar feedback and two for AGN feedback. Our analysis shows that the parameters controlling the energy flux of the feedback have a significant impact on the relaxation of dark matter at different epochs. In contrast, the parameters governing the speed and burstiness of feedback have negligible effects on the halo relaxation response.

We find that variations in stellar feedback strengths have a larger impact among dwarf galaxy-scale haloes, while variations in AGN feedback parameters exert a stronger influence on Milky Way-scale haloes. Notably, the relaxation offset in the outer well-resolved regions is stronger at the present epoch than at \( z=1 \), contrasting with results from the inner regions explored in the IllustrisTNG simulations in the first part of this chapter.

The stronger implementation of AGN feedback tends to result in greater relaxation at both \( z=0 \) and \( z=1 \) in the outer regions of the haloes. However, in the slightly inner regions, stronger AGN feedback implementation leads to a weaker relaxation offset at \( z=0 \) and a stronger offset at \( z=1 \). We interpret this as a consequence of the overall reduction in total feedback at \( z=0 \) due to the suppression of star formation caused by higher AGN feedbacks in the past. These results highlight the significance of feedback mechanisms in building a physical understanding of dark matter halo relaxation.

\subsubsection*{Role of Astrophysical Models in the EAGLE Simulations}
Finally, we assessed the impact of different astrophysical models in the EAGLE simulations. Supernova feedback strengths show a similar trend to that observed in the CAMELS simulations. Additionally, we find that the gas equation of state has the strongest effect on the relaxation response of dark matter, particularly among haloes hosting dwarf galaxies.

Overall, this work underscores the intricate relationship between baryonic processes and dark matter halo relaxation, illustrating the variations that arise due to different astrophysical models and redshifts.

\subsection{Causal Connections and Temporal Dynamics}
This study explored the dynamical evolution of dark matter's response to galaxies within populations of haloes simulated using the IllustrisTNG cosmological volumes. By constructing a detailed population of haloes and tracing their evolutionary tracks, we characterized the relaxation response of these haloes. This was performed by comparing haloes from hydrodynamical simulations, which include subgrid prescriptions for various astrophysical processes, with corresponding haloes from gravity-only simulations. Using a catalogue of evolving matched haloes, we examined the correlation between relaxation quantities and other halo/galaxy properties to elucidate their roles in mediating the relaxation response.

Firstly, we find that the radially-dependent linear relaxation relation model proposed in our previous work is applicable even at earlier redshifts, at least from redshift $z=5$. In this work, we have primarily studied the offset parameter $q_0$ in the relaxation relation that characterizes the amount of relaxation of the dark matter shells with no change in the enclosed mass. In a given population of haloes selected by their final mass, this offset is on average stronger during the peak star formation among those haloes. 
Our findings reveal that star formation activity significantly influences the offset in the halo relaxation response  over the entire evolutionary history of the haloes. While this connection with SFR is immediate on the relaxation in the inner haloes, it is seen 2 to 3 billion years later on average in the outer regions of both Milky Way scale haloes and halo groups. 
We also found that simple tracers of the stellar feedback processes through metal content only show a weaker connection with the relaxation than SFR itself. However, the wind accumulated from various feedback processes did have a stronger connection with the relaxation.

These insights enhance our understanding of the mechanisms driving halo relaxation and contribute to the development of more accurate models of halo profiles in baryonification procedures. 
For example, the knowledge of time lags can, in princple, allow modelling the observed dark matter distribution in large surveys such as Euclid with fewer parameters by exploiting correlations across different redshift bins.
And in semi-analytic galaxy formation models, this %will
may allow simple time-dependent transformation procedures to incorporate the dynamical evolution of the host dark halo with galaxy evolution, which is typically ignored. In the future, the relaxation response of dark matter haloes can also serve as a probe into the evolutionary history of the galaxies they host.

\subsection{Development of a Self-Similar Model}
To complement the insights from hydrodynamical simulations, we developed a spherical self-similar model for galaxy formation. This model simultaneously and self-consistently solves for the evolution of gas and dark matter, producing pseudo galaxy disks within haloes. The iterative method introduced in this model allows for the study of quasi-adiabatic relaxation responses, aligning well with findings from full non-linear simulations.

By systematically varying parameters such as the accretion rate and gas equation of state, we explored their effects on the relaxation response. Our results showed that the accretion rate and the gas equation of state significantly influence the relaxation relation, while other parameters, like the cooling rate, have a minor effect. These findings offer a deeper understanding of the sensitivity of dark matter profiles to various astrophysical processes.

\section{Applications and Broader Relevance}
The findings from this thesis have substantial implications for modeling dark matter halo profiles in astrophysical and cosmological contexts. The relaxation relations and timescales we derived can enhance the accuracy of baryonification schemes and semi-analytical galaxy formation models. For instance, understanding the time lags between star formation and halo relaxation can refine models of observed dark matter distributions in large-scale surveys like Euclid, potentially reducing the number of required parameters.

Moreover, the self-similar model developed here provides a computationally efficient framework for exploring the coupled impacts of multiple astrophysical processes on dark matter profiles. This approach is particularly valuable for studying the effects of baryons on rotation curves and gravitational lensing signals, and for improving cosmological parameter inference through emulators.



\section{Concluding Remarks}
This thesis has elucidated the intricate relationship between baryonic astrophysical processes and dark matter dynamics, offering new insights into the mechanisms driving halo relaxation response. We find that the relaxation response is a dynamical process with a considerable amount of associated timescales. Hence the dark matter at different halo-centric distances take different amount of time to respond to changes in the baryonic distribution caused by the galactic astrophysical processes. A better model 

\section{Future Directions}
While this thesis has made significant strides in understanding dark matter halo relaxation, several avenues for future research remain open. We plan to construct a unified model of relaxation that can predict the relaxation response of the dark matter halo as a function of the time-dependent evolution of the baryonic mass profile. In this regard we would primarily extend our time-correlation analysis to segregate the backreaction of astrophysical processes on the dark matter haloes into two parts. Firstly, the effect of those processes on the change in the baryonic mass profile and the second part is the effect of changing baryonic mass profile on the evolution of dark matter profile.

% but use more direct probes of the baryonic distribution rather than the 

Furthermore, exploring the potential of the self-similar model to yield straightforward analytical relations could benefit the development of quantitative models for relaxation. Incorporating the effects of star formation and central black hole development, along with their associated feedback mechanisms, into the self-similar model is a promising next step. Additionally, refining the model to more accurately replicate NFW density profiles in the virial region and incorporating more realistic cosmological mass accretion histories and major merger events will further enhance its applicability.
