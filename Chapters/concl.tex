\chapter{Conclusion}
\label{chap:conclusion}

\section{Summary of research work}

\section{Applications and relevance}

\section{Future directions}

% \chapter{Conclusion}

\section*{Summary of Key Findings}

This thesis presents a comprehensive study of the dynamical evolution of dark matter haloes in response to galaxy formation and evolution, using a combination of state-of-the-art cosmological hydrodynamical simulations and a novel self-similar model. Our findings significantly advance the understanding of dark matter relaxation processes and their connection to various astrophysical phenomena.

\subsection*{Background and Techniques}
We began by reviewing the foundational concepts and objectives of this research, highlighting the importance of studying dark matter halo responses to galaxy evolution in the broader context of cosmology and galaxy formation theories. We employed advanced simulation techniques to model the response of dark matter haloes, focusing on the IllustrisTNG and EAGLE simulations, as well as leveraging the CAMELS project for detailed feedback analysis.

\subsection*{Key Results from Simulations}
Our analysis of IllustrisTNG and EAGLE simulations revealed a radially-dependent linear relaxation relation model that accurately describes the dark matter relaxation response across a wide range of haloes. This model incorporates a novel fitting function that accounts for variations in halo-centric distance and emphasizes the significant role of star formation-related feedback processes.

Further, we explored the impact of astrophysical modeling on halo relaxation at different epochs. Our findings indicated that the gas equation of state in EAGLE simulations significantly influences halo responses, while AGN and stellar feedback in CAMELS simulations play a crucial role. We also discovered a more universal relaxation response at an earlier epoch ($z=1$) compared to the present epoch ($z=0$).

\subsection*{Causal Connections and Time-Series Analysis}
Through time-series analyses, we uncovered the causal connections between star formation activities, feedback processes, and halo relaxation. Star formation activity was found to have immediate effects on inner halo relaxation and delayed effects (2-3 billion years) on outer halo regions. This nuanced temporal relationship highlights the complex interplay between baryonic processes and dark matter dynamics.

\subsection*{Development of a Self-Similar Model}
To complement the insights gained from hydrodynamical simulations, we developed a spherical self-similar model for galaxy formation within isolated dark matter haloes. This model, which self-consistently solves for the evolution of gas and dark matter, successfully produces pseudo galaxy disks and mimics cold-mode accretion. The iterative method introduced in this model allows for the exploration of quasi-adiabatic relaxation responses, aligning with findings from full non-linear simulations.

\section*{Applications and Future Directions}
The results presented in this thesis have significant implications for the modeling of dark matter halo profiles in various astrophysical and cosmological contexts. The relaxation relations and timescales derived from our studies can enhance the accuracy of baryonification schemes and semi-analytical galaxy formation models. Specifically, understanding the time lags between star formation and halo relaxation can refine the modeling of observed dark matter distributions in large surveys like Euclid, potentially reducing the number of required parameters.

Moreover, the self-similar model developed here offers a computationally efficient framework for exploring the coupled impacts of multiple astrophysical processes on dark matter profiles. This approach has potential applications in studying the effects of baryons on rotation curves and gravitational lensing signals, and in improving cosmological parameter inference through emulators.

\section*{Future Work}
While this thesis has addressed many critical aspects of dark matter halo relaxation, several avenues remain for future research. Incorporating the effects of star formation and central black hole development, along with their associated feedback mechanisms, into the self-similar model is a promising next step. Additionally, refining the model to more accurately replicate NFW density profiles in the virial region and incorporating more realistic cosmological mass accretion histories will further enhance its applicability.

Finally, extending the study of dark matter relaxation responses to include cold accretion and major merger events will provide a more comprehensive understanding of halo dynamics across different cosmic epochs and environments.

\section*{Concluding Remarks}
This thesis underscores the intricate relationship between baryonic processes and dark matter dynamics, offering new insights into the mechanisms driving halo relaxation. The methodologies and findings presented here contribute valuable knowledge to the field of cosmology, paving the way for more precise and accurate models of galaxy formation and evolution. As our understanding of these processes deepens, we move closer to unraveling the complex interplay between dark matter and baryonic matter that shapes the universe.
