\chapter{Introduction}
\label{chap:intro}
\lhead{\emph{Introduction}}

\begin{quote}
Nature may be just a random collection of information becoming more random as we go forward in time, following simple fundamental rules. The real beauty lies in some rarely occurring but self-sustaining patterns that tend to accelerate the overall randomization, producing every interesting thing at different scales, such as galaxies, stars, and biological life on Earth. However, randomness will eventually eliminate every such pattern at some point in time. Yet, the same randomness causes some of these patterns to mutate into different forms that continue to survive, selected by nature. Among these patterns, the intelligent ones, such as humans, go a step further by creating correlations with nature through a process called learning. These correlations allow them to transform themselves, adapting to their environments without relying on random mutations that are only occasionally beneficial. As an organized form of learning, science is a process of systematically creating progressively more correlations with nature. Natural science only considers those patterns in nature not produced exclusively as a result of human intelligence, leaving those to social science, applied science, formal science, etc. More specifically, cosmology involves creating such correlations with patterns at the largest known scales of the Universe. A cosmological simulation is an evolution of these correlated patterns at human scales, leading to the creation of more human-scale patterns with increasingly stronger correlations with the largest scales of the Universe.

    
% While nature may seem like a random collection of information, becoming only more chatic over time as it follows simple, fundamental rules. However, wonderful things emerge at this, However, wonderful things  like galaxies  lies in the rare, self-sustaining patterns that emerge, accelerating the overall randomness and giving rise to wonders like galaxies, stars, and life on Earth. Yet, this same randomness eventually erases these patterns. At the same time, it creates new ones that continue to survive and evolve. Among these patterns, intelligent beings like humans go a step further by learning, a process that creates correlations with the external patterns into their brain patterns. This process allows them to adapt and thrive without relying solely on random, occasionally beneficial mutations. Science, as an organized form of learning, systematically identifies these similarities in nature. Natural science focuses on patterns not produced solely by human intelligence, unlike social, applied, or formal sciences. Specifically, cosmology aims to find the largest known patterns in the Universe. Cosmological simulations bring these grand patterns down to human scales, creating even stronger similarities between us and the cosmos.
    
\end{quote}
    
Dark matter is an hypothetical form of matter that has been exclusivel

% \chapter{Introduction}

\section{Background}

The concept of dark matter, which makes up over 80\% of all matter in the Universe as per the Lambda-cold dark matter ($\Lambda$CDM) model, is fundamental to modern cosmology. For nearly a century, dark matter is inferred exclusively through its gravitational effects and remains invisible to electromagnetic observations. The presence of dark matter was initially hypothesized to account for discrepancies in the rotational speeds of galaxies and the gravitational lensing of light that could not be explained by the visible matter alone.

Dark matter is believed to be crucial in the formation and evolution of the Universe's large-scale structure. Tiny fluctuations in the initial density field, originating from quantum perturbations in the early Universe, grow over time due to gravitational instability. These perturbations eventually collapse to form gravitationally bound structures known as dark matter haloes \citep[][]{1974ApJ...187..425P,2002PhR...372....1C}. These haloes act as the scaffolding upon which galaxies and other cosmic structures are built, making their study essential for understanding both cosmology and the nature of dark matter itself.

Cosmological simulations, particularly those employing the $\Lambda$CDM model, have been instrumental in studying dark matter haloes. These simulations, which often rely on N-body techniques to solve the gravitational interactions of a large number of particles, have revealed several key properties of dark matter haloes. For instance, haloes are generally triaxial in shape \citep[][]{1988ApJ...327..507F} and exhibit a universal density profile described by the Navarro-Frenk-White (NFW) profile \citep{1996ApJ...462..563N,1997ApJ...490..493N,2010MNRAS.402...21N}. These properties have provided significant insights into the distribution and behavior of dark matter on large scales.

\section{The Role of Dark Matter Haloes in Galaxy Formation}

Dark matter haloes are not only fundamental to cosmology but also to the formation and evolution of galaxies. Within these haloes, gas can cool and condense to form stars and other astrophysical objects \citep{1988MNRAS.234..459S,1998MNRAS.295..319M}. This process of baryonic matter interacting with dark matter haloes is complex and involves various non-gravitational processes such as radiative cooling, star formation, and feedback mechanisms from supernovae and active galactic nuclei (AGN).

Due to the complexity and wide dynamic range of these baryonic processes, it is challenging to study galaxy formation directly from first principles. Instead, semi-analytic models are often used, which incorporate simplified recipes for baryonic processes and are calibrated against observations \citep{2015ARA&A..53...51S}. These models allow for the efficient exploration of different galaxy formation scenarios but lack the full predictive power of direct simulations.

\section{Numerical Hydrodynamical Simulations}

The state-of-the-art approach to studying galaxy formation involves numerical hydrodynamical simulations that model both dark matter and baryonic physics within cosmological volumes. These simulations, such as EAGLE \citep{2015Schaye_EAGLE} and IllustrisTNG \citep{2018MNRAS.480.5113M}, use subgrid models to incorporate processes that occur on scales smaller than the resolution of the simulation. These include star formation, stellar feedback, and AGN feedback, which are parameterized based on observational data.

Hydrodynamical simulations provide a more realistic framework for studying the coupled evolution of dark matter and baryons. They account for the backreaction of galaxy formation on dark matter haloes, which can significantly alter the haloes' properties. For instance, the shape of dark matter haloes can be affected by the presence of a central galaxy \citep{2010MNRAS.407..435A,2021MNRAS.501.5679C}, which is important for interpreting weak lensing measurements \citep{2021A&A...647A.185G}. Additionally, the concentration and density profiles of dark matter haloes are modified by baryonic processes, influencing the rotation curves of galaxies.

\section{Open Problems and Research Motivation}

Despite the advances in numerical simulations, there remain significant uncertainties regarding the effects of baryonic physics on dark matter haloes. Different simulations often produce varying results, reflecting the uncertainties in modeling complex baryonic processes. Understanding the response of dark matter haloes to baryonic interactions is crucial for interpreting cosmological observations and for making accurate predictions about the Universe's evolution.

From a cosmological perspective, the effects of baryons are often considered nuisances in the inference of fundamental parameters. For instance, feedback from AGN can mimic the effects of other cosmological parameters, such as the presence of massive neutrinos or warm dark matter candidates, at certain scales \citep{2019Chisari_etal_Baryfeedback,2020AricoAnguloetal_baryonifi}. Distinguishing between these effects requires a detailed understanding of baryonic processes and their impact on dark matter.

This thesis aims to address some of these open problems by investigating the response of dark matter to baryonic physics using state-of-the-art cosmological simulations. Specifically, it will focus on the role of baryonic astrophysical modeling in controlling halo relaxation and the evolution of haloes at early epochs. By leveraging the IllustrisTNG, EAGLE, and CAMELS simulations, this work will explore the coupled evolution of dark matter and baryons and develop a self-similar model for galaxy formation. This research will contribute to a better understanding of the interplay between dark matter and baryonic processes, providing insights that are crucial for both cosmology and astrophysics.



In the standard paradigm of the $\Lambda$CDM cosmology, dark matter haloes are formed from the gravitational collapse around initial overdensities - Galaxies are then formed the baryonic matter within the haloes - dark halo response to the galaxy formation - literature on adiabatic relaxation.

\section{Background and Context:}
A detailed understanding of the formation and evolution of galaxies and their interaction with their environment is a pressing open problem. 
In the Lambda-cold dark matter ($\Lambda$CDM) picture of the Universe, the primary environment of any galaxy is provided by a `halo' of dark matter surrounding it \citep[e.g.,][]{wr78}. 
Therefore, understanding the dynamical behaviour of these dark haloes during the evolution of the galaxies they host is a key ingredient needed for building a complete picture of galaxy evolution.

Dark haloes form through the gravitational collapse of overdensities that developed from tiny fluctuations in the initial distribution of matter (e.g., \citealp[][]{1974ApJ...187..425P}; for a review see \citealp{2002PhR...372....1C}). 
The properties of dark haloes identified in gravity-only cosmological $N$-body simulations have been extensively studied in the literature. For example, while these haloes are known to be triaxial \citep[][]{1988ApJ...327..507F}, their sphericalised mass profiles are found to have a universal form (\citealp{1996ApJ...462..563N,1997ApJ...490..493N}, hereafter, NFW; see also \citealp{2010MNRAS.402...21N}). 
However, as galaxies and clusters of galaxies form within these haloes from various non-gravitational baryonic interactions, their gravitational coupling to the dark matter can affect the spatial distribution and evolution of the latter. Understanding this response of a halo's dark matter content to the baryons it hosts is then critical for understanding the coupled evolution of haloes and galaxies. 

Early work treated this response using adiabatic invariants
\citep[][]{osti6457593,1984MNRAS.211..753B,1986ApJ...301...27B,1987ApJ...318...15R}. 
Using simplifying assumptions such as spherical symmetry, no shell crossing, angular momentum conservation with circular orbits for dark matter particles, \citet[][]{1986ApJ...301...27B} derived a simple formula that quantifies the adiabatic relaxation of the dark matter mass profile in terms of the final baryonic distribution (we discuss this in detail later). Besides the change in their mass profiles, dark haloes can also become more spherical as a result of galaxy formation \citep[][]{1994ApJ...431..617D}.
Currently, the most robust technique to understand the consequences of gas assembly and galaxy formation on dark matter structure is the use of high-resolution cosmological hydrodynamical (zoom) simulations, using `sub-grid' recipes for modelling very small-scale astrophysics such as feedback from stellar/supernovae activity or the effects of active galactic nuclei (AGN) (see, e.g., OWLS, \citealp{2010MNRAS.402.1536S}, Illustris; \citealp{2014MNRAS.445..175G}; FIRE, \citealp{2014MNRAS.445..581H}; EAGLE, \citealp{2015MNRAS.446..521S}; Horizon-AGN,
\citealp[][]{2017MNRAS.467.4739K}; SIMBA,
\citealp[][]{2019MNRAS.486.2827D}; IllustrisTNG, \citealp{2019ComAC...6....2N}). In such simulations, the response of the dark matter to the presence of baryons in a halo can be ascertained by comparing a halo in the full hydrodynamical simulation to a matched `partner' halo in a collisionless, gravity-only simulation performed using the same initial random fluctuations. 
Using this technique, it was found that a simple adiabatic contraction model like that of \citet[][]{1986ApJ...301...27B} is an inaccurate description of the response in a variety of simulations \citep[see, e.g.,][]{2004ApJ...616...16G,2006PhRvD..74l3522G,2010MNRAS.402..776P,2010MNRAS.406..922T,2010MNRAS.405.2161D,2010MNRAS.407..435A,2011MNRAS.414..195T,2016MNRAS.461.2658D,2019A&A...622A.197A,2022MNRAS.511.3910F}. Even in a simpler setting, where star formation and feedback effects are ignored, \citet{2010MNRAS.407..435A} found that the halo responds to the condensation of baryons to the center by becoming more spherical and compact, while the change in its mass profile is found to be significantly less than the prediction of the idealized adiabatic relaxation model.


Several authors have investigated whether discarding some of the assumptions of the idealized model of \citet[][]{1986ApJ...301...27B} can help reconcile with the simulation results. \citet{2004ApJ...616...16G} considered non-spherical orbits for the dark matter particles and suggested a simple modification to the original model. This modified empirical formula shows wide variation in its parameters across haloes from different simulations and at different redshifts \citep[][]{2006PhRvD..74l3522G,2010MNRAS.405.2161D}. On the other hand, \citet[][]{2005ApJ...634...70S} accounted for these random motions within the halo using invariant action integrals, following the method described in \citet{1980ApJ...242.1232Y}. This model gives a reasonably good approximation of the response even in modern hydrodynamical simulations \citep{2020MNRAS.495...12C}; 
however, making predictions using this model requires access to orbital phase space information of the halo, which may not always be feasible.

Physically, one expects that the overall response of the halo is mediated by a combination of different astrophysical processes that occur in the galaxy. Feedback processes are known to reduce the contraction of the halo significantly; e.g.,
supernova-driven winds can completely transform the inner density profile of the dark matter halo
\citep[][]{1996MNRAS.283L..72N}. This may be the key in reconciling the observation of dark matter cores at the center of various galaxies with the cuspy haloes found in gravity-only $\Lambda$CDM simulations \citep[see][for a review]{2014Natur.506..171P}.
However, such feedback effects do not always produce dark matter cores from cusps, rather, this can depend on the amount of gas ejected, the mass loss time scale and the frequency of starburst events 
(see, e.g., \citealp{2011ApJ...736L...2O,2014ApJ...793...46O,2012MNRAS.421.3464P}, and also \citealp{bfln18}).
In massive haloes hosting galaxy groups or clusters, while the formation of powerful AGN in the central galaxy can strongly suppress star formation,
it can still significantly reduce the adiabatic contraction of the halo \citep[][]{2011MNRAS.414..195T}.
Moreover, the fluctuation in gravitational potential due to such feedback can expel the dark matter from the inner halo producing inner cores \citep[][]{2012MNRAS.422.3081M}.

Different aspects of the halo response, such as the change in its mass profile, shape, phase space distribution and substructure population, have been explored to date in a variety of hydrodynamical simulations \citep[see, e.g.,][]{2004ApJ...611L..73K,2008ApJ...681.1076D,2014MNRAS.441.2986D,2015MNRAS.451.1247S,2017MNRAS.466.3876Z,2017MNRAS.472.4343C,2019MNRAS.484..476C,2021arXiv210900012C,2021MNRAS.501.5679C,2020MNRAS.494.4291C,freundlich+20,riggs+22}.
Understanding the nature of this response (or `baryonic backreaction') is critical in building accurate and robust models of halo shapes and sizes, for use in interpreting the results of upcoming large-volume surveys (\citealp{2015JCAP...12..049S,2018MNRAS.480.3962C,2021MNRAS.503.3596A}; see also \citealp{velliscig+14,hwvh15,mead+15}) as well as the detailed prediction of rotation curves and related statistics \citep{2021MNRAS.507..632P,2021arXiv211200026P}.
The goal of this chapter is to perform a systematic, statistical study of this dark matter response in high-resolution hydrodynamical simulations incorporating realistic feedback and quantify it using simple analytical forms, including the sensitivity of this response to halo-centric distance and halo and galaxy properties. To this end, we use the publicly available suites of simulations from the IllustrisTNG and EAGLE projects. 


\subsection{Dark matter haloes}

\subsection{Cosmological simulations}

\subsection{Simulations with galaxy formation}

\subsection{Relaxation response of dark matter}

\section{Objective}

% \section{Quasi-adiabatic relaxation}

\section{Self-similar systems}
\cite{2015LauNagaietal}

\section{Motivation}

\section{Dynamical relaxation}

\section{Outline}
