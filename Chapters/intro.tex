\chapter{Introduction}
\label{chap:intro}
\lhead{\emph{Introduction}}

\quotehere{
    Nature may be just a random collection of information becoming more random as we go forward in time, following simple fundamental rules. The real beauty lies in some rarely occurring but self-sustaining patterns that tend to accelerate the overall randomization, producing every interesting thing at different scales, such as galaxies, stars, and biological life on Earth. However, randomness will eventually eliminate every such pattern at some point in time. Yet, the same randomness causes some of these patterns to mutate into different forms that continue to survive, selected by nature. Among these patterns, the intelligent ones, such as humans, go a step further by creating correlations with nature through a process called learning. These correlations allow them to transform themselves adapting to their environments without relying on random mutations that are only occasionally good. As an organized form of learning science is a process of systematically creating progressively more correlations with nature. Natural science only considers those patterns in nature not produced exclusively as a result of human intelligence; leaving those to social science, applied science, formal science, etc. More specifically, Cosmology involves creating such correlations with patterns at the largest known scales of the Universe. A cosmological simulation is an evolution of these correlated patterns at human scales, leading to the creation of more human-scale patterns with increasingly stronger correlations with the largest scales of the Universe.
}


In the standard paradigm of the $\Lambda$CDM cosmology, dark matter haloes are formed from the gravitational collapse around initial overdensities - Galaxies are then formed the baryonic matter within the haloes - dark halo response to the galaxy formation - literature on adiabatic relaxation.

\section{Background and Context:}
A detailed understanding of the formation and evolution of galaxies and their interaction with their environment is a pressing open problem. 
In the Lambda-cold dark matter ($\Lambda$CDM) picture of the Universe, the primary environment of any galaxy is provided by a `halo' of dark matter surrounding it \citep[e.g.,][]{wr78}. 
Therefore, understanding the dynamical behaviour of these dark haloes during the evolution of the galaxies they host is a key ingredient needed for building a complete picture of galaxy evolution.

Dark haloes form through the gravitational collapse of overdensities that developed from tiny fluctuations in the initial distribution of matter (e.g., \citealp[][]{1974ApJ...187..425P}; for a review see \citealp{2002PhR...372....1C}). 
The properties of dark haloes identified in gravity-only cosmological $N$-body simulations have been extensively studied in the literature. For example, while these haloes are known to be triaxial \citep[][]{1988ApJ...327..507F}, their sphericalised mass profiles are found to have a universal form (\citealp{1996ApJ...462..563N,1997ApJ...490..493N}, hereafter, NFW; see also \citealp{2010MNRAS.402...21N}). 
However, as galaxies and clusters of galaxies form within these haloes from various non-gravitational baryonic interactions, their gravitational coupling to the dark matter can affect the spatial distribution and evolution of the latter. Understanding this response of a halo's dark matter content to the baryons it hosts is then critical for understanding the coupled evolution of haloes and galaxies. 

Early work treated this response using adiabatic invariants
\citep[][]{osti6457593,1984MNRAS.211..753B,1986ApJ...301...27B,1987ApJ...318...15R}. 
Using simplifying assumptions such as spherical symmetry, no shell crossing, angular momentum conservation with circular orbits for dark matter particles, \citet[][]{1986ApJ...301...27B} derived a simple formula that quantifies the adiabatic relaxation of the dark matter mass profile in terms of the final baryonic distribution (we discuss this in detail later). Besides the change in their mass profiles, dark haloes can also become more spherical as a result of galaxy formation \citep[][]{1994ApJ...431..617D}.
Currently, the most robust technique to understand the consequences of gas assembly and galaxy formation on dark matter structure is the use of high-resolution cosmological hydrodynamical (zoom) simulations, using `sub-grid' recipes for modelling very small-scale astrophysics such as feedback from stellar/supernovae activity or the effects of active galactic nuclei (AGN) (see, e.g., OWLS, \citealp{2010MNRAS.402.1536S}, Illustris; \citealp{2014MNRAS.445..175G}; FIRE, \citealp{2014MNRAS.445..581H}; EAGLE, \citealp{2015MNRAS.446..521S}; Horizon-AGN,
\citealp[][]{2017MNRAS.467.4739K}; SIMBA,
\citealp[][]{2019MNRAS.486.2827D}; IllustrisTNG, \citealp{2019ComAC...6....2N}). In such simulations, the response of the dark matter to the presence of baryons in a halo can be ascertained by comparing a halo in the full hydrodynamical simulation to a matched `partner' halo in a collisionless, gravity-only simulation performed using the same initial random fluctuations. 
Using this technique, it was found that a simple adiabatic contraction model like that of \citet[][]{1986ApJ...301...27B} is an inaccurate description of the response in a variety of simulations \citep[see, e.g.,][]{2004ApJ...616...16G,2006PhRvD..74l3522G,2010MNRAS.402..776P,2010MNRAS.406..922T,2010MNRAS.405.2161D,2010MNRAS.407..435A,2011MNRAS.414..195T,2016MNRAS.461.2658D,2019A&A...622A.197A,2022MNRAS.511.3910F}. Even in a simpler setting, where star formation and feedback effects are ignored, \citet{2010MNRAS.407..435A} found that the halo responds to the condensation of baryons to the center by becoming more spherical and compact, while the change in its mass profile is found to be significantly less than the prediction of the idealized adiabatic relaxation model.


Several authors have investigated whether discarding some of the assumptions of the idealized model of \citet[][]{1986ApJ...301...27B} can help reconcile with the simulation results. \citet{2004ApJ...616...16G} considered non-spherical orbits for the dark matter particles and suggested a simple modification to the original model. This modified empirical formula shows wide variation in its parameters across haloes from different simulations and at different redshifts \citep[][]{2006PhRvD..74l3522G,2010MNRAS.405.2161D}. On the other hand, \citet[][]{2005ApJ...634...70S} accounted for these random motions within the halo using invariant action integrals, following the method described in \citet{1980ApJ...242.1232Y}. This model gives a reasonably good approximation of the response even in modern hydrodynamical simulations \citep{2020MNRAS.495...12C}; 
however, making predictions using this model requires access to orbital phase space information of the halo, which may not always be feasible.

Physically, one expects that the overall response of the halo is mediated by a combination of different astrophysical processes that occur in the galaxy. Feedback processes are known to reduce the contraction of the halo significantly; e.g.,
supernova-driven winds can completely transform the inner density profile of the dark matter halo
\citep[][]{1996MNRAS.283L..72N}. This may be the key in reconciling the observation of dark matter cores at the center of various galaxies with the cuspy haloes found in gravity-only $\Lambda$CDM simulations \citep[see][for a review]{2014Natur.506..171P}.
However, such feedback effects do not always produce dark matter cores from cusps, rather, this can depend on the amount of gas ejected, the mass loss time scale and the frequency of starburst events 
(see, e.g., \citealp{2011ApJ...736L...2O,2014ApJ...793...46O,2012MNRAS.421.3464P}, and also \citealp{bfln18}).
In massive haloes hosting galaxy groups or clusters, while the formation of powerful AGN in the central galaxy can strongly suppress star formation,
it can still significantly reduce the adiabatic contraction of the halo \citep[][]{2011MNRAS.414..195T}.
Moreover, the fluctuation in gravitational potential due to such feedback can expel the dark matter from the inner halo producing inner cores \citep[][]{2012MNRAS.422.3081M}.

Different aspects of the halo response, such as the change in its mass profile, shape, phase space distribution and substructure population, have been explored to date in a variety of hydrodynamical simulations \citep[see, e.g.,][]{2004ApJ...611L..73K,2008ApJ...681.1076D,2014MNRAS.441.2986D,2015MNRAS.451.1247S,2017MNRAS.466.3876Z,2017MNRAS.472.4343C,2019MNRAS.484..476C,2021arXiv210900012C,2021MNRAS.501.5679C,2020MNRAS.494.4291C,freundlich+20,riggs+22}.
Understanding the nature of this response (or `baryonic backreaction') is critical in building accurate and robust models of halo shapes and sizes, for use in interpreting the results of upcoming large-volume surveys (\citealp{2015JCAP...12..049S,2018MNRAS.480.3962C,2021MNRAS.503.3596A}; see also \citealp{velliscig+14,hwvh15,mead+15}) as well as the detailed prediction of rotation curves and related statistics \citep{2021MNRAS.507..632P,2021arXiv211200026P}.
The goal of this chapter is to perform a systematic, statistical study of this dark matter response in high-resolution hydrodynamical simulations incorporating realistic feedback and quantify it using simple analytical forms, including the sensitivity of this response to halo-centric distance and halo and galaxy properties. To this end, we use the publicly available suites of simulations from the IllustrisTNG and EAGLE projects. 


\subsection{Dark matter haloes}

\subsection{Cosmological simulations}

\subsection{Simulations with galaxy formation}

\subsection{Relaxation response of dark matter}

\section{Objective}

% \section{Quasi-adiabatic relaxation}

\section{Self-similar systems}
\cite{2015LauNagaietal}

\section{Motivation}

\section{Dynamical relaxation}

\section{Outline}
