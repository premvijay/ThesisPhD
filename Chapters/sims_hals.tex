\chapter{Simulations of halo response}
\label{chap:sims-hals}

In this chapter, we briefly describe the methods employed in studying the halo relaxation through simulations.

\section{Hydrodynamical simulations with galaxies}
\label{sec:sims}
Here we describe the cosmological simulations employed in this work; these are from three different publicly available suites namely IllustrisTNG, EAGLE and CAMELS simulations.
\subsection{IllustrisTNG}
\label{sec:sims-IllTNG}
The IllustrisTNG simulations, conducted by the TNG collaboration, employed the \textsc{arepo} code \citep[][]{2020ApJS..248...32W}, which utilizes a moving mesh approach defined by Voronoi tessellation \citep[][]{2010MNRAS.401..791S}. These simulations incorporate an updated model of galaxy formation that includes cosmic magnetic fields in addition to major baryonic processes such as cooling, star formation, and stellar and AGN feedback \citep[][]{2017MNRAS.465.3291W,2018MNRAS.473.4077P}. The suite comprises three cosmological boxes: TNG50, TNG100, and TNG300, with periodic box sizes of $35 \Mpch$, $75 \Mpch$, and $200 \Mpch$, respectively, consistent with the cosmology from \cite{2016A&A...594A..13P}. Initial conditions were generated at $z=127$ using the Zel'dovich approximation \citep[][]{1970A&A.....5...84Z} with the \textsc{N-GenIC} code \citep[][]{2015ascl.soft02003S}. 

We utilize the highest resolution runs from all three boxes to study a wide range of halo responses to galaxy formation. Specifically, TNG50 offers sufficient resolution for low-mass haloes, while TNG300 provides an adequate number of cluster-scale haloes. Throughout our analysis, we utilize data from redshifts $z=0.01$ and $z=1$ from IllustrisTNG for both hydrodynamical and corresponding gravity-only runs. This allows us to examine the effects of baryonic processes on dark matter halo properties across different scales and epochs.

\subsection{EAGLE}
\label{sec:sims-EAGLE}
The EAGLE (Evolution and Assembly of GaLaxies and their Environments) cosmological simulations were conducted using a modified version of the \textsc{gadget-3} code, which employs smoothed particle hydrodynamics \citep[][]{2005MNRAS.364.1105S}. Initial conditions were generated using the \textsc{ic\_2lpt\_gen} code following \cite{2010MNRAS.403.1859J}. The main large-volume simulation was performed in a cosmological volume of $(100 ~\rm{Mpc})^3$ periodic box with its reference model of galaxy formation, incorporating sub-grid prescriptions for various baryonic processes such as cooling, star formation, and feedback mechanisms \citep[][]{2015MNRAS.446..521S,2015MNRAS.450.1937C}. This reference model of EAGLE simulations has been shown to produce realistic galaxies \citep[][]{2015MNRAS.448.2941S,2015MNRAS.450.4486F,2015MNRAS.452.2879T}. 

In addition, this suite includes multiple small-volume simulations with wide variations in the baryonic subgrid prescriptions for astrophysical processes. These simulations were performed at the same resolution as the large-volume reference simulation but in a $25 ~\rm{Mpc}$ periodic box. The variations include adjustments to the gas equation of state, the threshold for star formation, efficiency of stellar feedback, the viscosity of the black-hole accretion disk and the nature of stochastic heating caused by AGN feedback. In particular we study these following simulations:

\begin{itemize}
    \item \textbf{Ref} (Reference model): This simulation uses the standard EAGLE subgrid physics parameters but in this smaller box.
    \item \textbf{eos53} : This considers a gas equation of state $\gamma = 5/3$.
    \item \textbf{eos1} : This considers an isothermal gas equation of state $\gamma = 1$.
    \item \textbf{FixedSfThresh}: This uses a constant threshold for star formation independent of the metallicity.
    \item \textbf{WeakFB}: This follows a lower efficiency of stellar feedback that is scaled down by $50 \%$ compared to the reference simulation.
    \item \textbf{StrongFB}: This follows a higher efficiency of stellar feedback, twice that of the reference simulation.
    \item \textbf{NoAGN}: AGN feedback is disabled in this simulation to study the effects of stellar feedback alone.
    \item \textbf{AGNdT8}: In this, the temperature of gas raised by the AGN feedback heating is  lower with $\Delta T_{\rm{AGN}} = 10^8$K from the reference value of $\Delta T_{\rm{AGN}} = 10^{8.5}$K.
    \item \textbf{AGNdT9}: In this, the temperature of gas raised by the AGN feedback heating is higher with $\Delta T_{\rm{AGN}} = 10^9$K from the reference value of $\Delta T_{\rm{AGN}} = 10^{8.5}$K.
    \item \textbf{RefHR}: This simulation uses the same EAGLE subgrid prescription as `Ref' simulation but at a $8 times$ higher mass resolution.
    \item \textbf{RecalHR}: This simulation uses the same resolution as `RefHR' but with recalibrated EAGLE subgrid prescription.
\end{itemize}
We use the redshift $z=0$ data from this set of simulations along with their corresponding gravity-only run to study the role of different astrophysical processes on the relaxation response of the halo.



\subsection{CAMELS}
\label{sec:sims-CAMELS}
The Cosmology and Astrophysics with MachinE Learning Simulations (CAMELS) project comprises a comprehensive suite of hydrodynamical simulations designed to explore the interplay between cosmological and astrophysical parameters in shaping the universe's large-scale structure \cite[][]{CAMELS_presentation}. These simulations are performed in a relatively smaller cosmological volume of $(25 \ \mathrm{Mpc}/h)^3$, containing $256^3$ dark matter particles and an equivalent number of baryonic particles \cite{CAMELS_DR1}.

In our study, we specifically utilize the TNG suite's 1P set of simulations, a subset that methodically varies one parameter at a time to isolate the effects of individual parameters in the TNG model. For our analysis, we concentrate on parameters related to supernova (SN) feedback and active galactic nucleus (AGN) feedback, each governed by the following two distinct parameters:
% 
\begin{itemize}
    \item \textbf{Supernova Feedback Parameters:}
    \begin{itemize}
        \item $A_{\mathrm{SN1}}$: Varied between 0.25 to 4, this parameter controls the energy flux of the galactic winds. It is implemented as a prefactor for the overall energy output per unit star formation rate \cite{2018MNRAS.473.4077P,CAMELS_presentation}.
        \item $A_{\mathrm{SN2}}$: Varied between 0.5 to 2, this parameter controls the speed of the galactic winds. For a fixed $A_{\mathrm{SN1}}$, changes in $A_{\mathrm{SN2}}$ affect the galactic wind speed in concert with the mass-loading factor to maintain a fixed energy output.
    \end{itemize}
    \item \textbf{AGN Feedback Parameters:}
    \begin{itemize}
        \item $A_{\mathrm{AGN1}}$: Varied between 0.25 to 4, this parameter controls the overall power injected in the kinetic feedback mode of AGN. It is implemented as a prefactor for the energy per unit black-hole accretion rate \cite{2017MNRAS.465.3291W,CAMELS_presentation}.
        \item $A_{\mathrm{AGN2}}$: Varied between 0.5 to 2, this parameter controls the burstiness and the temperature of the heated gas during AGN feedback "bursts" by changing the wind speed of the AGN feedback.
    \end{itemize}
\end{itemize}
% 
All these parameters have a value of one in the fiducial simulation, and there are five simulations with higher values and another five simulations with lower values for each of these parameters. In total, we utilize these 41 hydrodynamical simulations and compare them against the gravity-only simulation over the same cosmological volume.


% \section{Techniques}
% \label{sec:methods}
\section{Haloes with relaxation}
In the simulations from IllustrisTNG and EAGLE suites, 3D friend-of-friends (FoF) algorithm \citep[see][for details]{2016A&C....15...72M,2019ComAC...6....2N} was used to obtain halo group catalogues. And the \textsc{subfind} code \citep{2001MNRAS.328..726S} was used to identify the subhaloes within these FoF group haloes as gravitationally bound substructures. Within each FoF group halo, the subhalo enclosing the gravitational potential minimum is assigned as its central subhalo. This trough in the gravitational potential is used to define the centre of that halo. Its size is characterized by the `virial' radius $R_{\rm vir}\equiv R_{\rm 200c}$ defined as the radius of the sphere around its centre enclosing a mean matter density that is 200 times the cosmological critical density. Its mass is then defined as the corresponding total mass enclosed $M\equiv M_{200c}$. In the simulations from the CAMELS suite, the haloes were identified in the  phase-space by a 6D FoF algorithm using the \textsc{rockstar} code. We characterize sizes and masses using the same quantities  $R_{\rm 200c}$ and $M_{\rm 200c}$ respectively.
% is defined as the `virial' radius $R_{\rm vir}\equiv R_{\rm 200c}$ of a given FoF group halo; while the total mass enclosed within this radius quantifies the mass of the halo $M\equiv M_{200c}$.

% Following our previous work \citep{2023Velmani&Paranjape}, 
To study the relaxation response of dark matter, we match the haloes from the full hydrodynamic simulations with the haloes in the corresponding gravity-only runs performed over same cosmological volumes. We identify these matched halo pairs based on the amount of overlap in their proto halo patches. In particular, this involves identifying nearby haloes of similar sizes between the hydrodynamical and gravity-only simulations and then match them by simulation particles \cite{2023Velmani&Paranjape}.
%  From this catalogue, we select populations of matched halo pairs by the logarithmic mass of the gravity-only halo $(\log(M/\Mh))$ in bins centred at $11.5, 12, 12.5, 13, 13.5, 14$ with a bin width of 0.3 at redshift $z=0.01$. While the small volume TNG50 offers well-resolved low-mass haloes $10^{11.5} \Mh$ and $10^{12} \Mh$, the TNG100 cosmological box gives $10^{12.5} \Mh$ and $10^{13} \Mh$ haloes and the largest volume TNG300 provides an adequate number of cluster-scale haloes of masses $10^{13.5} \Mh$ and $10^{14} \Mh$. 


\section{Relaxation Response Modeling}
\label{sec:methods-relmodel}
In the catalogues of matched halo pairs, hydrodynamical ones with galaxies are considered relaxed, while the gravity-only counterparts represent unrelaxed dark haloes. The relaxation response of dark matter within a halo, is generally evaluated through variations in their radial mass profiles indicating contraction or expansion due to the galaxy.  These spherically averaged dark matter profiles are computed by the cumulative sum of the mass contributed by all dark matter particles within concentric spherical shells. For the gravity-only halo, the cosmic dark matter fraction of the mass in each particle is considered as contributing to the dark matter. 

We employ the quasi-adiabatic relaxation framework to characterize the relaxation from these profiles.
% \subsection{Quasi-adiabatic relaxation}
% \label{sec:methods-adiab}
The relaxation response in cold dark matter occurs entirely due to gravitational interactions with baryons. This is an aggregate effect of the baryonic mass flow resulting from galactic processes such as inflows and feedback. The quasi-adiabatic relaxation model is a physically motivated framework that relates the change in the spherically averaged dark matter distribution at a given time to the spherically averaged baryonic distribution at the same time. This baryonic profile encompasses all non-dark matter mass, including gas and stars.

Earlier models assumed spherical halo where the dark matter particles maintain their radial ordering while responding adiabatically to baryonic particle flows \citep[][]{1986ApJ...301...27B}. In this scenario, if a dark matter particle initially at radius \( r_i \) in the unrelaxed halo moves to radius \( r_f \) in the relaxed halo, the enclosed dark matter mass within these radii remains equal.
\begin{equation} 
M_f^d(r_f) = M_i^d(r_i)\,.
\label{eq:DMmass}
\end{equation}

However, due to the baryonic mass flow, the total mass enclosed within these spheres is not necessarily equal, \( M_i(r_i) \neq M_f(r_f) \). Further assumption of angular momentum conservation for dark matter particles in circular orbits, implies that the change in total enclosed mass must be consistent with the amount of relaxation \citep[][]{1986ApJ...301...27B}.
\begin{align}
    r_i \,M_i(r_i) = r_f \,M_f(r_f) %
    \implies 
    \frac{r_f}{r_i} = \frac{M_i(r_i)}{M_f(r_f)}\,. 
\label{eq:AR}
\end{align}
% 
The \textbf{Quasi-adiabatic relaxation framework} empirically extends this idealized scenario by considering the relaxation ratio \( r_f/r_i \) as a function of the mass ratio \( M_i/M_f \).
\begin{align}
\frac{r_f}{r_i} &= 1 + \chi \left( \frac{M_i(r_i)}{M_f(r_f)} \right) 
\label{eq:qAR}
\end{align}
In a straightforward extension, the baryonification procedures described in \cite{2015JCAP...12..049S,2021MNRAS.503.4147P} incorporate dark matter response as a quasi-adiabatic relaxation with 
\be
\chi(y) = q\,(y-1)\,.
\label{eq:chi-linear}
\ee
Various quasi-adiabatic models have been proposed, offering different approaches to this framework \citep{2010MNRAS.407..435A,2004ApJ...616...16G,2023Velmani&Paranjape}.


\section{Haloes}

\section{Characterizing the halo response}