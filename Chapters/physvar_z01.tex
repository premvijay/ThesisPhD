\chapter[Role of astrophysical modeling and epoch]{Role of astrophysical modeling on dark matter halo relaxation response at redshifts $z=0$ and $z=1$}
\label{chap:physvar_z01}

Advancements in computational cosmology have produced state-of-the-art hydrodynamical simulations of cosmological volumes with realistic galaxies (see, e.g., OWLS \citep{2010MNRAS.402.1536S}, Illustris \citep{2014MNRAS.445..175G}, FIRE \citep{2014MNRAS.445..581H}, EAGLE \citep{2015MNRAS.446..521S}, Horizon-AGN \citep[][]{2017MNRAS.467.4739K}, SIMBA \citep[][]{2019MNRAS.486.2827D}, IllustrisTNG \citep{2019ComAC...6....2N}). However, many sub-galactic astrophysical processes, such as star formation, are not resolved by these simulations. Instead, they rely on subgrid prescriptions with parameters calibrated to a set of empirical observations. Ambiguities in the modeling and the calibrated quantities have led to inconsistent results. Additionally, these simulations are computationally expensive to perform, which is why dark matter haloes have typically been studied using gravity-only simulations. Nevertheless, the impact of baryonic processes on the dark matter can be significant and must be accounted for before comparing against observations.

Early studies of individual haloes modeled the response of the radial distribution of dark matter to galaxy formation as adiabatic relaxation \citep[][]{osti6457593,1984MNRAS.211..753B,1986ApJ...301...27B,1987ApJ...318...15R}. However, this idealistic model rarely predicts the dark matter distribution within haloes in hydrodynamical cosmological simulations \citep[e.g.,][]{2004ApJ...616...16G,2010MNRAS.407..435A}. Moreover, the halo response was found to vary widely across haloes and in different simulations \citep[][]{2004ApJ...616...16G,2006PhRvD..74l3522G,2010MNRAS.402..776P,2010MNRAS.406..922T,2010MNRAS.405.2161D,2010MNRAS.407..435A,2011MNRAS.414..195T,2016MNRAS.461.2658D,2019A&A...622A.197A,2022MNRAS.511.3910F,2023Velmani&Paranjape}, leading to the development of various models of the response, some of which are direct extensions of adiabatic relaxation \citep[e.g.,][]{2004ApJ...616...16G,2006PhRvD..74l3522G,2010MNRAS.407..435A}. \AP{chapter heading at top left part of the page is showing `Introduction'? this may affect all chapters.}~\PV{Fixed them}

In \chapref{chap:z0_main}, we have demonstrated that introducing an additional dependence on the halo-centric distance makes the relation between $r_f/r_i$ and $M_i/M_f$ linear across a variety of haloes in both IllustrisTNG and EAGLE simulations at the present epoch ($z=0$). Based on those results, we have presented a simple prescription for computing the relaxed dark matter mass profiles. In the first part of this chapter, we investigate if such a relaxation model can also be used at an earlier redshift.

Previous works have shown that feedback from various galactic processes strongly influences halo relaxation \citep{2011MNRAS.414..195T}. These effects primarily affect the offset in the relaxation relation, quantifying the excess relaxation experienced by the dark matter \citep{2023Velmani&Paranjape}. We have argued in \chapref{chap:z0_main}, that this could be the reason for the small deviations in the relaxation offset across haloes with different star formation activities. In this chapter, we systematically the role of various astrophysical processes in the galaxy, such as feedback, in mediating the relaxation response of the halo using simulations with variations in the feedback implementation.

This chapter is organized as follows. In \secref{sec:res-itng-z01}, we explore the relaxation response at an earlier redshift in the IllustrisTNG reference simulation. This allows us to understand how the different galactic processes occurring at earlier times affect dark matter halo relaxation. Then in \secref{sec:res-physvar-CAMELS}, we systematically study the relaxation response as a function of the feedback related parameters in the IllustrisTNG model using the set of simulations from CAMELS as described in \secref{sec:sims-CAMELS}. 
Following this, we investigate the role of a variety of astrophysical processes in the EAGLE physics variation simulations described in \secref{sec:sims-EAGLE} and conclude in \secref{sec:conclusion-ch:physvar}.

\section{Early epoch in IllustrisTNG simulations}
\label{sec:res-itng-z01}

In this section, we investigate the relaxation response in the radial distribution of dark matter in the main simulations of IllustrisTNG simulations at an earlier redshift of $z = 1$ and compare it to the present redshift of $z = 0$. Haloes at both these redshifts were identified and matched between hydrodynamical and the corresponding gravity-only runs as described in \secref{sec:hals}. Additionally, we use the SubLink merger tree catalogues to trace the most massive progenitor haloes at $z=1$ of the haloes considered at $z=0$. While present epoch haloes are sampled only by their masses at the present time, we consider three different methods for sampling the early epoch haloes. This results in four distinct sets of halo samples:

\begin{enumerate}
    \item $z=1$ haloes sampled by their masses at $z=1$.
    \item $z=1$ haloes sampled by the masses of their descendants at $z=0$. Note that not all haloes with a given mass at the present time have valid progenitors with the same mass at $z=1$.
    \item $z=1$ haloes sampled by their masses at $z=1$, but the mass bins are defined by the median masses of the most massive progenitors of the $z=0$ haloes.
    \item $z=0$ haloes sampled by their masses at $z=0$.
\end{enumerate}

In each case, the mass bins are defined as described in \secref{sec:results-mass-ch:z0main} considering the mass of the gravity-only halo. The representative colors to represent this halo mass is shown in \figref{fig:mass_bin_label-z01}. Additionally, this figure indicates the peak heights ($\nu$) corresponding to these halo masses at both $z=0$ and $z=1$. These values correspond to the rarity of haloes with that mass at that redshift with rarer haloes having larger values of $\nu$. This can be used to identify mass of the halo at $z=1$ that will have same rarity as $z=0$ halo of a given mass. 

For each of these four sets of halo samples, the average relaxation relations ($r_f/r_i$ vs $M_i/M_f$) are shown in \figref{fig:fit-view-mass-indep}. The bottom right panel, reproduces the \figref{fig:fit-view-mass-indep-ch:z0main} and shown here for comparison. We find that, for a given halo mass, relaxation is usually stronger at the earlier epoch (top left panel) compared to the present time $z=0$ (bottom right panel). In both cases, the trend in relaxation with halo mass is similar, with the strongest relaxation observed in $10^{12} \Mh$ haloes. Interestingly, cluster-scale haloes with masses of $10^{14} \Mh$ at redshift $z=1$ exhibit significant relaxation, unlike clusters of similar size at the present time.

The progenitors at redshift $z=1$ of the same haloes found at $z=0$ show even stronger  relaxation, especially in Milky Way-scale and larger haloes, as shown in the top right panel of \figref{fig:fit-view-mass-indep}. The relaxation follows the simple quasi-adiabatic model \eqref{eq:chi-linear-ch:sims} with $q=0.33$ among larger cluster-scale ($10^{14} \Mh$) haloes, while group-scale haloes are consistent with the second-order polynomial relation proposed by Abadi et al. (2010) \cite{2010MNRAS.407..435A}. In the bottom left panel, the relaxation relation is shown for all haloes within a narrow mass bin around the median mass of the progenitors of the haloes selected at $z=0$. Notice that the relaxation relation shifts further lower in Milky Way-scale haloes and smaller clusters with the inclusion of those additional haloes.

\begin{figure}[htbp]
\centering
\includegraphics[width=0.49\linewidth]{plots/Mass_bin_labels_z.pdf}
\caption{Representative colors denoting each of the halo mass bins. The numbers in the figure indicate the corresponding values of peak heights $\nu$ at redshifts $z=0$ and $z=1$.}
\label{fig:mass_bin_label-z01}
\end{figure}

\begin{figure*}
\centering
\includegraphics[width=0.48\linewidth]{plots/fit_view_M_T_snap049.pdf}
\includegraphics[width=0.48\linewidth]{plots/fit_view_M_T_snap049_smpl98.pdf}
\includegraphics[width=0.48\linewidth]{plots/fit_view_M_T_snap049_smpl98_allHalsMrange.pdf}
\includegraphics[width=0.48\linewidth]{plots/fit_view_M_T_snap098.pdf}
\caption{The stacked relation between relaxation ratio and mass ratio as a function of halo mass in IllustrisTNG at $z=0$ \emph{(bottom right panel)} and $z=1$ \emph{(other panels)}. In the top right panel, relaxation is shown at $z=1$ for the progenitors of the haloes selected at $z=0$. In the second row, left panel, relaxation is shown at different mass bins at $z=1$, indicated by corresponding mass bins at $z=0$. Points represent stacks over fixed halo-centric distances, and solid lines represent stacks over fixed mass ratios. The color-coding follows Fig.~\ref{fig:mass_bin_label-z01}. The quasi-adiabatic relaxation model \eqref{eq:chi-linear-ch:sims} with $q=0.68$ and $q=0.33$ are shown by the dot-dashed and dashed purple lines, respectively, in each panel.}
\label{fig:fit-view-mass-indep}
\end{figure*}



\begin{figure}[htbp]
\centering
\includegraphics[width=0.48\linewidth,trim={0.5cm 0 0 0},clip]{plots/fit_params_rf_M_T_snap049.pdf}
\includegraphics[width=0.48\linewidth,trim={0.5cm 0 0 0},clip]{plots/fit_params_rf_M_T_snap049_smpl98.pdf}
\includegraphics[width=0.48\linewidth,trim={0.5cm 0 0 0},clip]{plots/fit_params_rf_M_T_snap049_smpl98_allHalsMrange.pdf}
\includegraphics[width=0.48\linewidth,trim={0.5cm 0 0 0},clip]{plots/fit_params_rf_M_T_snap098.pdf}
\caption{Linear quasi-adiabatic relaxation model parameters $q_1$ and $q_0$ as a function of the halo-centric distance at different halo masses in IllustrisTNG at $z=0$ \emph{(bottom right panel)} and $z=1$ \emph{(other panels)}. In the top right panel, relaxation is shown at $z=1$ for the progenitors of the haloes selected at $z=0$. In the second row, left panel, relaxation is shown at different mass bins at $z=1$, indicated by corresponding mass bins at $z=0$. The color-coding follows Fig.~\ref{fig:mass_bin_label-z01}.}
\label{fig:rf-fit-params}
\end{figure}

In \chapref{chap:z0_main}, we proposed a locally linear model of the relaxation relation as follows:
\begin{align}
\label{eq:chi-linear-q0-ch:physvar}
\frac{r_f}{r_i} - 1 &= q_1(r_f) \left[ \frac{M_i(r_i)}{M_f(r_f)} - 1 \right] + q_0(r_f),.
\end{align}
We have tested this relation with our halo samples at $z=1$ and found it to hold reasonably well. \red{For each halo sample, at each halo-centric distance $r_f$, the relationship between mass ratio and relaxation ratio across all haloes is best fitted by a linear curve with the slope and offset parameters namely $q_1(r_f)$ and $q_0(r_f)$. The radial profiles of these two relaxation parameters are shown for the four set of halo populations in \figref{fig:rf-fit-params} with the color-coding given by \figref{fig:mass_bin_label-z01} for the associated halo mass.}

\red{Notice that the universality in these relaxation profiles extends to much larger mass haloes ($\sim 10^{13.5} \Mh$) at $z=1$ (top left panel) compared to $z=0$ (bottom right panel). This is despite the fact that such massive haloes are even rarer at $z=1$ than at $z=0$, as indicated by the value of $\nu$ in \figref{fig:mass_bin_label-z01}. For all the halo samples selected at $z=0$, their traced progenitor populations at $z=1$ show nearly universal relaxation profiles, with the exception of $q_0(r_f)$ in the inner halo (shown in the top right panel). This universality is even more apparent among the halo populations selected in narrow mass bins at $z=1$ based on the median masses of the progenitor populations (see bottom left panel).}

\red{The less universal and relatively noisier profiles in direct progenitors might be due to the inclusion of low mass fast accreting haloes together with high mass slow accreting haloes or simply a systematic issue such as a few misidentified progenitors. This difference between the direct progenitor populations and the haloes selected by the median progenitor masses seems to be more noticeable in the relaxation profiles than in the radially independent relaxation relations. For example, the black curves (corresponding to $10^{14} \Mh$) are noticeably different especially in inner regions between top left and bottom right panels in \figref{fig:rf-fit-params}, however, the overall relaxation relation shown in \figref{fig:fit-view-mass-indep} is well-fitted by the linear $q=0.33$ model in both the panels. While the exact reason is not yet clear, we note that the radially-dependent relaxation might be more sensitive to this difference. \AP{not sure about this... to me the black curves in bottom left and top right look very similar, except in the innermost parts of the halo}~\PV{Edited for clarity} Another interesting thing to note is that $q_1(r_f)$ shows a small non-monotonic behavior in the outer haloes at all masses in $z=1$ (e.g., see lower left panel) that is not seen at $z=0$ (lower right panel). This could indicate some relaxation effects that have already happened by $z=1$, happens later in the outermost regions.}

The offset parameter $q_0$ shown in the lower sub-panels is relatively uniform across the halo especially among the low mass haloes. The \figref{fig:fit-fit-func-q} shows the mean of $q_0$ in all four sets of halo samples. The $q_0$ parameter is usually more negative across all $z=1$ halo populations compared to $z=0$, indicating a stronger relaxation offset. Additionally, the values are more universal with halo mass at $z=1$. \red{Recall that a negative value of $q_0$ is expected to be a result of recent feedback outflows (see \secref{sec:results-rad-dep-qadiab-ch:z0main} for a detailed discussion). These feedback outflows are produced by a combination of AGN and stellar feedbacks. In the high mass haloes, powerful AGN feedbacks at earlier epochs lead to significant suppression in the star formation activity reducing the stellar feedbacks at present epoch. The reduction in the magnitude of $q_0$ from $z=1$ to $z=0$ in the high mass haloes could be simply due to the reduction in overall feedback among those haloes.}
\AP{some physics discussion needed to explain differences seen between different samples}~\PV{Added discussion}



\begin{figure}[htbp]
\centering
\includegraphics[width=0.6\linewidth]{plots/fit_param_q0_M_T_z01.pdf}
\caption{Mean of the radially dependent quasi-adiabatic relaxation offset, $q_{0}$ as a function halo mass in the four sets of halo samples indicated by color.}
\label{fig:fit-fit-func-q}
\end{figure}










\section{Varying feedback strengths using CAMELS simulations}
\label{sec:res-physvar-CAMELS}
In this section, we present the role of various feedback parameters prescribed in the IllustrisTNG simulations using the set of CAMELS simulations performed with the IllustrisTNG model. This includes a set of 41 hydrodynamical simulations, with one replicating the reference TNG model in a smaller cosmological volume, and 10 simulations each by varying 4 different feedback parameters as described in \secref{sec:sims-CAMELS}. To recall, this includes two supernovae feedback parameters ($A_{\mathrm{SN1}}$ and $A_{\mathrm{SN2}}$) and another two AGN feedback parameters ($A_{\mathrm{AGN1}}$ and $A_{\mathrm{AGN2}}$).

Due to the limited resolution and the smaller volumes of the CAMELS simulation, among all the mass bins shown in \figref{fig:mass_bin_label-z01}, we consider only $10^{11} \Mh$, $10^{11.5}$, and $10^{12} \Mh$ at both redshifts $z=0$ and $z=1$. Even in these mass bins, we consider only the outer well-resolved regions of the haloes. Also, since the cosmological volume is smaller than even the smallest TNG50 simulation, we have only a smaller sample of haloes at each of these halo mass bins. We find this sample size insufficient to estimate the radially-dependent relaxation parameters at each of these mass bins. We consider the following two approaches to alleviate this issue.

\subsection*{Intercepts in the Relaxation Relation}

The intercepts of the relaxation relation, given by the relation between $M_i/M_f-1$ and $r_f/r_i-1$, already provide interesting information about the relaxation. For example, the y-intercept denotes the offset in the relaxation ratio $r_f/r_i$ from unity for the shells having a mass ratio of unity $M_i/M_f=1$. Similarly, the x-intercept denotes the offset in the mass ratio $M_i/M_f$ from unity for the shells having a relaxation ratio $r_f/r_i = 1$.

In a given sample of haloes, we denote the average x and y intercepts as $q_x$ and $q_y$ respectively. For example, if we consider the Milky Way scale haloes at redshift $z=0$, the relaxation relation shown by the green curves in the lower right panel of \figref{fig:fit-view-mass-indep} indicates that $q_x$ will be positive and $q_y$ will be negative. In general, we expect that feedback effects would lead to larger $q_x$ and more negative $q_y$.

Due to the limited resolution of the CAMELS simulation, the outer well-resolved regions in most haloes didn't have a mass ratio less than unity. This makes the y-intercept of the relaxation relation available only in a smaller number of haloes. This makes our estimation of $q_y$ very noisy, and hence we only interpret the parameter $q_x$ in these simulations. This parameter $q_x$ is presented in \figref{fig:camels-qx0} as a function of the astrophysical parameters in the CAMELS TNG set of simulations at both redshifts $z=0$ and $z=1$.

\subsection*{Wider mass bin}

We found that the radially-dependent relaxation parameters are usually more uniform across a wider range of halo masses. We leverage this to consider haloes in a wider mass bin from $10^{11} \Mh$ to $10^{12} \Mh$, which gives a sufficient number of haloes to obtain the radially-dependent relaxation model parameters. However, still, the radial range is not sufficient to accurately model the radial dependence of the slope parameter $q_1(r_f)$, so we only investigate the offset parameter $q_0$, defined as the mean of the $q_0(r_f)$. This parameter $q_0$ is usually negative, and it is expected to be more negative when the offset produced by overall feedback effects is stronger. This model-dependent offset parameter is presented in \figref{fig:camels-q0q1} at both redshifts $z=0$ and $z=1$.

\begin{figure}[htbp]
\centering
% \includegraphics[width=0.325\linewidth]{plots/CAMELS_I_qx0_sn18_11.pdf}
% \includegraphics[width=0.325\linewidth]{plots/CAMELS_I_qx0_sn18_11.5.pdf}
% \includegraphics[width=0.325\linewidth]{plots/CAMELS_I_qx0_sn18_12.pdf}
% \includegraphics[width=0.325\linewidth]{plots/CAMELS_I_qx0_sn33_11.pdf}
% \includegraphics[width=0.325\linewidth]{plots/CAMELS_I_qx0_sn33_11.5.pdf}
% \includegraphics[width=0.325\linewidth]{plots/CAMELS_I_qx0_sn33_12.pdf}
\includegraphics[width=\linewidth]{plots/CAMELS_I_qx0.pdf}
\caption{Relaxation offset parameter $q_x$ \AP{axis labels say $q_{0}^x$ while text uses $q_x$}\PV{updated plot} as a function of the baryonic astrophysical feedback parameters in haloes found in CAMELS-TNG at three different halo masses. Top: $z=1$, Bottom: $z=0$}
\label{fig:camels-qx0}
\end{figure}

\begin{figure}[htbp]
\centering
\includegraphics[width=0.49\linewidth]{plots/CAMELS_I_q0_sn18.pdf}
% \includegraphics[width=0.49\linewidth]{plots/CAMELS_I_q1_sn18.pdf}
\includegraphics[width=0.49\linewidth]{plots/CAMELS_I_q0_sn33.pdf}
% \includegraphics[width=0.49\linewidth]{plots/CAMELS_I_q1_sn33.pdf}
\caption{Relaxation offset parameter $q_0$ as a function of the baryonic feedback parameters in CAMELS-TNG. Left: $z=1$, Right: $z=0$.}
\label{fig:camels-q0q1}
\end{figure}

\subsection*{Discussion}

We find that among all haloes investigated, the feedback strength parameters $A_{\mathrm{SN1}}$ and $A_{\mathrm{AGN1}}$ have a strong influence on the relaxation, with $q_x$ typically increasing monotonically when increasing these parameters. On the other hand, 
%whereas 
the wind speed parameters $A_{\mathrm{SN2}}$ and $A_{\mathrm{AGN2}}$ have negligible effect on the relaxation characterized by both $q_x$ and $q_0$.


Recall that when varying only a wind speed parameter, it affects the burstiness of the feedback outflows while keeping the overall flux constant. Suppose the deviations in the relaxation relation from the idealized adiabatic model, quantified by non-zero offset parameters, are caused by the transfer of angular momentum between the dark matter particles and the baryonic particles. In that case, one may expect that the nature of the baryonic feedback quantified by the wind speed parameters will have a significant influence on the value of $q_x$. However, our results suggest otherwise.

% These results suggest that the overall feedback outflows controlled by the 
\red{Recall that in \secref{subsubsec:sim-relax-ch:z0main}, we have argued that the relaxation offset is a reflection of the fact that the dark matter shells have not yet expanded in response to the recent feedback gas outflows. These results suggest that the time taken for the dark matter shells to expand are sufficiently long enough, that only the overall gas outflow is relevant irrespective of the speed and burstiness. We investigate these timescales in \chapref{chap:dynam-relxn}.} 
\AP{explain what this means. why is time delay relevant?} \AP{briefly describe how you infer this consistency}\PV{Edited for clarity}

We find that the AGN feedback strength generally has a stronger influence on the relaxation among the high-mass haloes, whereas the supernova feedback strength has a stronger influence in lower-mass haloes. This is consistent with our expectations that AGN feedbacks dominate in more massive haloes. However, at all masses, AGN feedback starts dominating the value of $q_x$ \red{when their strengths are set to be higher than the reference model of TNG} \AP{on what basis are these unrealistic? do we even care whether they are unrealistic or not? why mention this word?}; this is indicated by the green curves in \figref{fig:camels-qx0}. \red{This suggests that the relative importance of AGN and supernovae feedback on the dark matter relaxation depends strongly on their implementation.}

\AP{this para seems disconnected from previous one.}\PV{Added discussion and edited for clarity}
\AP{which part of which plot shows me the effect on these slightly inner regions, and why?}
\AP{are you talking about $q_0$ or $q_x$?}
\red{While $q_x$ which characterizes the relaxation offset predominantly in the outer halo,} is always larger with the stronger implementation of AGN feedback at both \( z=0 \) and \( z=1 \), \red{the value of $q_0$, which is averaged over both inner and outer halo, shows a different trend.}  Notice, in \figref{fig:camels-q0q1}, that the stronger AGN feedback implementation leads to a weaker relaxation offset at \( z=0 \) and a stronger offset at \( z=1 \).  We interpret this as a consequence of the overall reduction in total feedback at \( z=0 \) due to the suppression of star formation caused by higher AGN feedbacks in the past. This led to reduction in the relaxation offset in the inner haloes, \red{the outer halo has not yet responded by $z=0$}. These results highlight the significance of feedback mechanisms in building a physical understanding of dark matter halo relaxation.

% We find that at $z=1$, the magnitude of $q_0$ is larger with stronger AGN feedback indicated by larger $A_{\rm{AGN1}}$. However, this trend is absent at $z=0$. This could be due to a strong suppression in star formation rate at the present epoch due to the stronger AGN feedback in the past. \AP{did you just forget to delete this para? it seems like a shorter and less informative version of the previous para. what is the point of this para?}









\section{Role of astrophysical modeling in EAGLE simulations}
\label{sec:res-physvar-eagle}

In this section, we present insights from small boxes of EAGLE simulations on the role of different feedback mechanisms on the relaxation response. These physics variation simulations from the EAGLE suite are described in \secref{sec:sims-EAGLE}. Again, due to the significantly smaller box size and resolution, we only consider haloes in three different mass bins centered at $10^{10.5} \Mh$, $10^{11} \Mh$, and $10^{11.5} \Mh$ as discussed in \secref{sec:res-physvar-CAMELS}. In these narrow mass bins, the mean relaxation relation obtained by stacking independent of the halo-centric distance is shown in \figref{fig:EAGLE-rad-indep} for these physics variation simulations. We find that the deviation in the relaxation across different astrophysical modeling is usually much smaller than the differences from one halo mass to the other. However, notice that the gas equation of state has a strong influence on the relaxation relation, especially among low-mass haloes. In particular, stiffer equations of state lead to a very large $r_f/r_i$ indicating a stronger expansion of the dark matter shells in response to galaxy formation.

In \figref{fig:EAGLE-rad-dep}, we present the radially dependent relaxation parameters for a larger sample of haloes in a wider range of halo masses from $10^{10.5} \Mh$ to $10^{11.5} \Mh$. These results also reflect the strong influence of the equation of state of gas on the relaxation response of dark matter haloes. Additionally, these results highlight the effect of supernovae feedback modeling. In particular, the $q_0(r_f)$ is more negative, indicating a stronger offset in the haloes found in the simulation with stronger supernova feedback implementation (brown curve vs. rose curve).    

\begin{figure}[htbp]
\centering
\includegraphics[width=0.32\linewidth]{plots/eagle_physvar_rad_indep_relxn_reln_MiMf_10.5.pdf}
\includegraphics[width=0.32\linewidth]{plots/eagle_physvar_rad_indep_relxn_reln_MiMf_11.pdf}
\includegraphics[width=0.32\linewidth]{plots/eagle_physvar_rad_indep_relxn_reln_MiMf_11.5.pdf}
\caption{Relaxation relation in the physics variation EAGLE simulations for haloes in the mass bins from $10^{10.5} \Mh$ to $10^{11.5} \Mh$. Here colors represent the specific simulation with a variation in the baryonic physics prescription.}
\label{fig:EAGLE-rad-indep}
\end{figure}

\begin{figure}[htbp]
\centering
\includegraphics[width=0.7\linewidth]{plots/fit_params_rf_M_E_physvar_fatmass_uniradb.pdf}
\caption{Radially-dependent relaxation parameters for low-mass haloes from $10^{10.5} \Mh$ to $10^{11.5} \Mh$ as a function of the halo-centric distance in the physics variation EAGLE simulations. Here colors represent the specific simulation with a variation in the baryonic physics prescription.}
\label{fig:EAGLE-rad-dep}
\end{figure}


\section{Conclusion}
\label{sec:conclusion-ch:physvar}

In this chapter, we investigated the influence of astrophysical modeling on the relaxation response of dark matter haloes at different epochs, specifically focusing on \( z=0 \) and \( z=1 \). The analysis is divided into three main parts, each shedding light on the role of various astrophysical processes in shaping the dark matter content of haloes.

% \subsubsection*{1. Early Epoch in IllustrisTNG Simulations}
We began by examining the relaxation response at an earlier redshift (\( z=1 \)) in the IllustrisTNG simulations using three distinct sets of halo samples, which highlight the variations in relaxation across different halo masses. Our study reveals that dark matter relaxation tends to be stronger (smaller \( r_f/r_i \)) at the earlier epoch compared to the present among haloes of the same mass. This is even more prominent among the progenitors of present epoch haloes. Notably, we observe that cluster-scale haloes at \( z=1 \) show significant relaxation (\( r_f/r_i < 1 \)) that is also a function of the change in the enclosed mass (\( M_i/M_f \)). This is in contrast to similar haloes at the present epoch, where \( r_f/r_i \) stayed close to unity on average irrespective of the value of \( M_i/M_f \).

We also find that the locally linear quasi-adiabatic relaxation model is a good description of the relaxation relation at this earlier epoch, demonstrating its robustness in capturing the dark matter response across redshifts. Moreover, the parameters of the radially dependent relaxation are found to be more universal across a much wider range of masses at \( z=1 \). For example, the progenitors of even the most massive clusters are well characterized by the simple three-parameter model of relaxation that was developed with a focus on galactic-scale haloes at \( z=0 \). \red{This suggests that the deviation from the three-parameter model is a result of late-time events such as mergers in the cluster scale haloes.}\AP{speculate on why this is the case.}

% \subsubsection*{2. Variation in Astrophysical Feedback Using CAMELS Simulations}
Next, we explored variations in astrophysical feedback strengths within the IllustrisTNG model using simulations from the CAMELS project, which varies four different feedback parameters: two for stellar feedback and two for AGN feedback. Our analysis shows that the parameters controlling the energy flux of the feedback have a significant impact on the relaxation of dark matter at different epochs. In contrast, the parameters governing the speed and burstiness of feedback have negligible effects on the halo relaxation response. \red{This further strengthens our argument that the relaxation offset is caused by the dark matter shells that have not yet responded to the recent feedback outflows.} \AP{mention this is consistent with discussion of meaning of $q_0$}

We find that variations in stellar feedback strengths have a larger impact among dwarf galaxy-scale haloes, while variations in AGN feedback parameters exert a stronger influence on Milky Way-scale haloes. Notably, the relaxation offset in the outer well-resolved regions is stronger at the present epoch than at \( z=1 \), contrasting with results from the inner regions explored in the IllustrisTNG simulations in the first part of this chapter.

The stronger implementation of AGN feedback tends to result in greater relaxation at both \( z=0 \) and \( z=1 \) in the outer regions of the haloes. However, in the slightly inner regions, stronger AGN feedback implementation leads to a weaker relaxation offset at \( z=0 \) and a stronger offset at \( z=1 \). We interpret this as a consequence of the overall reduction in total feedback at \( z=0 \) due to the suppression of star formation caused by higher AGN feedbacks in the past. These results highlight the significance of feedback mechanisms in building a physical understanding of dark matter halo relaxation.

% \subsubsection*{3. Role of Astrophysical Models in the EAGLE Simulations}
Finally, we assessed the impact of different astrophysical models in the EAGLE simulations. Supernova feedback strengths show a similar trend to that observed in the CAMELS simulations. Additionally, we find that the gas equation of state has the strongest effect on the relaxation response of dark matter, particularly among haloes hosting dwarf galaxies.

Overall, this chapter underscores the intricate relationship between baryonic processes and dark matter halo relaxation, illustrating the variations that arise due to different astrophysical models and redshifts.