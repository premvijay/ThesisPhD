%% ----------------------------------------------------------------
%% Thesis.tex -- MAIN FILE (the one that you compile with LaTeX)
%% ---------------------------------------------------------------- 

% Set up the document
\documentclass[a4paper, 12pt, oneside]{Thesis}  % Use the "Thesis" style, based on the ECS Thesis style by Steve Gunn
\graphicspath{Figures/}  % Location of the graphics files (set up for graphics to be in PDF format)

% Include any extra LaTeX packages required
\usepackage[square, numbers, comma, sort&compress]{natbib}  % Use the "Natbib" style for the references in the Bibliography
\usepackage{verbatim}  % Needed for the "comment" environment to make LaTeX comments
\usepackage{vector}  % Allows "\bvec{}" and "\buvec{}" for "blackboard" style bold vectors in maths
\usepackage[utf8]{inputenc}
\usepackage{aas_macros}
\usepackage{amsfonts}
\usepackage{amsmath}
\usepackage{amssymb}
\usepackage[normalem]{ulem}
\usepackage{graphicx}
\usepackage[dvipsnames]{xcolor}
\usepackage{hyperref}
\usepackage{soul}
\hypersetup{colorlinks=true,allcolors=teal}  % Colours hyperlinks in blue, but this can be distracting if there are many links.

% \hypersetup{colorlinks=true,allcolors=teal}

\newcommand{\Hi}{\textsc{Hi}}
\newcommand{\mHi}{\ensuremath{m_{\Hi}}}

\newcommand{\p}{\ensuremath{\partial}}


\newcommand{\Msun}{\ensuremath{M_{\odot}}}
\newcommand{\Mh}{\ensuremath{h^{-1}M_{\odot}}}
\newcommand{\Mhsq}{\ensuremath{h^{-2}M_{\odot}}}
\newcommand{\Mpch}{\ensuremath{h^{-1}{\rm Mpc}}}
\newcommand{\kpch}{\ensuremath{h^{-1}{\rm kpc}}}
\newcommand{\kms}{\ensuremath{{\rm km\,s}^{-1}}}
\newcommand{\msq}{\ensuremath{{\rm \,m\,s}^{-2}}}

\newcommand{\avg}[1]{\ensuremath{\left\langle \,#1\, \right\rangle}}
\newcommand{\e}[1]{\ensuremath{{\rm e}^{#1}}}

\newcommand{\der}{\ensuremath{{\rm d}}}
\newcommand{\Der}{\ensuremath{{\rm D}}}
\newcommand{\dir}{\ensuremath{\delta_{\rm D}}}

\newcommand{\erfc}[1]{\ensuremath{{\rm erfc}\left(#1\right)}}
\newcommand{\erf}[1]{\ensuremath{{\rm erf}\left(#1\right)}}

\newcommand{\eqn}[1]{equation~\eqref{#1}}
\newcommand{\eqns}[1]{equations~\eqref{#1}}
\newcommand{\eqnref}[1]{equation~\eqref{#1}}
\newcommand{\eqnsref}[1]{equations~\eqref{#1}}
\newcommand{\ph}[1]{\phantom{#1}}
\newcommand{\figref}[1]{Figure~\ref{#1}}
\newcommand{\tabref}[1]{Table~\ref{#1}}
\newcommand{\secref}[1]{Section~\ref{#1}}
\newcommand{\appnref}[1]{Appendix ~\ref{#1}}

\newcommand{\be}{\begin{equation}}
\newcommand{\ee}{\end{equation}}
\newcommand{\Cal}[1]{\ensuremath{\mathcal{#1}}}

\newcommand{\AP}[1]{\emph{\color{blue}[AP: #1]}}
\newcommand{\PV}[1]{\emph{\color{violet}[PV: #1]}}
\newcommand{\red}[1]{\textcolor{red}{#1}}

%% ----------------------------------------------------------------
\begin{document}
\frontmatter      % Begin Roman style (i, ii, iii, iv...) page numbering

% Set up the Title Page
\title  {Interplay of galaxy formation and the evolution of dark matter haloes in the cosmic web}
\authors    {\texorpdfstring{\href{premv@iucaa.in}{Premvijay Velmani}}{Premvijay Velmani}}
\addresses  {\groupname\\\deptname\\\univname}  % Do not change this here, instead these must be set in the "Thesis.cls" file, please look through it instead
\date       {\today}
\subject    {}
\keywords   {}

\maketitle
%% ----------------------------------------------------------------

\setstretch{1.5}  % It is better to have smaller font and larger line spacing than the other way round

% Define the page headers using the FancyHdr package and set up for one-sided printing
\fancyhead{}  % Clears all page headers and footers
\rhead{\thepage}  % Sets the right side header to show the page number
\lhead{}  % Clears the left side page header

\pagestyle{fancy}  % Finally, use the "fancy" page style to implement the FancyHdr headers

%% ----------------------------------------------------------------
% Declaration Page required for the Thesis, your institution may give you a different text to place here
\Declaration{

\addtocontents{toc}{\vspace{1em}}  % Add a gap in the Contents, for aesthetics

I, Premvijay Velmani, declare that this thesis titled, `Interplay of galaxy formation and the evolution of dark matter haloes in the cosmic web' and the work presented in it are my own. I confirm that:

\begin{itemize} 
\item[\tiny{$\blacksquare$}] This work was done wholly or mainly while in candidature for a research degree at this University.
 
\item[\tiny{$\blacksquare$}] Where any part of this thesis has previously been submitted for a degree or any other qualification at this University or any other institution, this has been clearly stated.
 
\item[\tiny{$\blacksquare$}] Where I have consulted the published work of others, this is always clearly attributed.
 
\item[\tiny{$\blacksquare$}] Where I have quoted from the work of others, the source is always given. With the exception of such quotations, this thesis is entirely my own work.
 
\item[\tiny{$\blacksquare$}] I have acknowledged all main sources of help.
 
\item[\tiny{$\blacksquare$}] Where the thesis is based on work done by myself jointly with others, I have made clear exactly what was done by others and what I have contributed myself.
\\
\end{itemize}
 
 
Signed:\\
\rule[1em]{25em}{0.5pt}  % This prints a line for the signature
 
Date:\\
\rule[1em]{25em}{0.5pt}  % This prints a line to write the date
}
\clearpage  % Declaration ended, now start a new page

%% ----------------------------------------------------------------
% The "Funny Quote Page"
\pagestyle{empty}  % No headers or footers for the following pages

\null\vfill
% Now comes the " Quote", written in italics
% \textit{``Write a quote here.''}

\begin{flushright}
If the quote is taken from someone, their name goes here
\end{flushright}

\vfill\vfill\vfill\vfill\vfill\vfill\null
\clearpage  % Funny Quote page ended, start a new page
%% ----------------------------------------------------------------

% The Abstract Page
\addtotoc{Abstract}  % Add the "Abstract" page entry to the Contents
\abstract{
\addtocontents{toc}{\vspace{1em}}  % Add a gap in the Contents, for aesthetics

Dark matter haloes are gravitationally collapsed structures hosting the formation and evolution of galaxies. While being crucial probes in both cosmology and particle physics, they have often been modeled in gravity-only scenarios, neglecting the effects of baryonic astrophysics. In this thesis, we study the interplay between galaxy formation and its host halo evolution, focusing on changes in the radial distribution of dark matter in response to galaxy formation and evolution. We build a comprehensive understanding using both hydrodynamical simulations of realistic galaxies in cosmological volumes and more tractable self-similarly evolving systems of individual galaxies with their host haloes. This unified picture of the role of various galactic astrophysical processes in mediating the response of dark matter haloes also provides useful timescales for predicting dynamical changes in dark matter mass profiles due to those astrophysical processes.

In the first part of the thesis, we characterise this response statistically in a variety of haloes spanning over four orders of magnitude in mass in the cosmological simulations of galaxy formation from the IllustrisTNG and EAGLE suite at present epoch. We present simple fitting functions in the quasi-adiabatic relaxation framework that accurately capture the dark matter response over the full range of halo mass and halo-centric distance we explore. We show that commonly employed schemes, which consider the relative change in radius $r_f/r_i-1$ of a spherical dark matter shell to be a function of only the relative change in its enclosed mass $M_i/M_f-1$, do not accurately describe the measured response of most haloes in IllustrisTNG and EAGLE. Rather, $r_f/r_i$ additionally explicitly depends upon halo-centric distance $r_f/R_{\rm vir}$ for haloes with virial radius $R_{\rm vir}$, being very similar between IllustrisTNG and EAGLE and across halo mass. We also account for a previously unmodelled effect, likely driven by feedback-related outflows, in which shells having $r_f/r_i\simeq1$ (i.e., no relaxation) have $M_i/M_f$ significantly different from unity.  This difference is usually stronger among the less concentrated recently formed haloes than the older haloes. Similarly, this effect is also prominent in haloes with currently strong star formation activity. These results presented in chapter \ref{chap:z0_main} are immediately applicable to a number of semi-analytical tools for modelling galactic and large-scale structure. We conclude this chapter with a discussion on the possible extensions of these results to build a deeper physical understanding of the connection between dark matter and baryons within haloes.

In the second part, we investigate the role of astrophysical modeling in producing the halo response at different epochs in simulations. We find that the quasi-adaiabtic relaxation with additional dependence on the halo-centric distance is a good description also at earlier epoch. Also the parameters in this model are found to be more Universal across much larger variety of haloes at \(z=1\) than the \(z=0\). Through systematic analysis of a large collection of simulations from the CAMELS project, we find that the baryonic prespriptions for both AGN and stellar feedbacks have a strong influence on the relaxation response of the dark matter halo. In particular, only the parameters controlling the overall feedback energy flux had an effect on the relaxation response, while the wind speed and burstiness have negligible effect on the relaxation at fixed amount of energy flux. However, the exact role of these parameters on the relaxation depends on the redshift. Along with these results in \ref{chap:physvar_z01}, we also study the role of a variety of baryonic astrophysical processes through the EAGLE physics variation simulations. While this depict a similar picture regarding the importance of feedback effects, it also reveals that the gas equation of state to have one of the strongest influence on the relaxation response.

We then present the dynamical evolution of the dark matter's relaxation response to galaxies and their connection to the astrophysical properties as simulated in the IllustrisTNG suite of cosmological hydrodynamical simulations in chapter \ref{chap:dynam-relxn}. Our results show that the radially-dependent linear relaxation relation model proposed in chapter \ref{chap:z0_main} is applicable at least from redshift \(z=5\). We focus on the offset parameter \(q_0\), which characterizes the relaxation of dark matter shells without changing the enclosed mass. We perform multiple time-series analyses to determine the possible causal connections between the relaxation mechanism and astrophysical processes such as star formation and associated feedback processes, as well as feedback due to active galactic nuclei. We show that star formation activity significantly influences the halo relaxation response throughout its evolutionary history, with essentially immediate effects in the inner haloes and delayed effects of 2 to 3 Gyr in the outer regions. Metal content shows a weaker connection to relaxation than star formation rates, but the accumulated wind from feedback processes exhibits a stronger correlation. These findings enhance our understanding of halo relaxation mechanisms. Our estimates of the time-scales relevant for dark matter relaxation can potentially improve the description of halo profiles in existing baryonification schemes and semi-analytical galaxy formation models. Our results also show how the relaxation response of dark haloes can probe the evolutionary history of the galaxies they host.

In the last part of this thesis, we develop a spherical self-similar model for the formation of a galaxy through gas collapsing in an isolated self-gravitating dark matter halo. As is well known, the self-similarity assumption makes the problem eminently tractable by reducing it to a system of ordinary differential equations. We improve upon the existing literature on self-similar collapse in two ways. First, we include the effects of radiative cooling and the formation of a pseudo-disk at the center of collapse, in a parametrised manner. More importantly, we solve for the evolution of gas and dark matter \emph{simultaneously and self-consistently} using a novel iterative approach. As a result, our model produces shell trajectories of both gas \emph{and} dark matter that qualitatively agree with the results of full hydrodynamical simulations of self-gravitating systems. We discuss the impact of various ingredients such as the accretion rate, gas equation of state, disk radius and cooling rate amplitude on the evolution of the gas shells. The self-consistent evolution of gas and dark matter allows us to study the relaxation response of the dark matter trajectories to the presence of collapsing gas. This model produces relaxation relations that are in qualitative agreement with more realistic cosmological hydrodynamical simulations, while allowing us to easily study the impact of the model ingredients mentioned above. As an initial application, we vary one ingredient at a time and find that the accretion rate and gas equation of state have the largest impact on the relaxation relation, while the cooling amplitude plays only a minor role. Our model thus provides a convenient framework to rapidly explore the coupled nonlinear impact of multiple astrophysical processes on the mass and velocity profiles of dark matter in galactic halos, and consequently on observables such as rotation curves and gravitational lensing signals.

% In summary, this thesis presents a comprehensive analysis of the dynamical evolution of the dark matter radial distribution in connection to various baryonic astrophysical processes within haloes. We find that this response is quasi-adiabatic

}

\clearpage  % Abstract ended, start a new page
%% ----------------------------------------------------------------

\setstretch{1.5}  % Reset the line-spacing to 1.3 for body text (if it has changed)

% The Acknowledgements page, for thanking everyone
\acknowledgements{
\addtocontents{toc}{\vspace{1em}}  % Add a gap in the Contents, for aesthetics

Firstly, I would like to acknowledge

}
\clearpage  % End of the Acknowledgements
%% ----------------------------------------------------------------

\pagestyle{fancy}  %The page style headers have been "empty" all this time, now use the "fancy" headers as defined before to bring them back


%% ----------------------------------------------------------------
\lhead{\emph{Contents}}  % Set the left side page header to "Contents"
\tableofcontents  % Write out the Table of Contents

%% ----------------------------------------------------------------
\lhead{\emph{List of Figures}}  % Set the left side page header to "List if Figures"
\listoffigures  % Write out the List of Figures

%% ----------------------------------------------------------------
\lhead{\emph{List of Tables}}  % Set the left side page header to "List of Tables"
\listoftables  % Write out the List of Tables

%% ----------------------------------------------------------------
\setstretch{1.5}  % Set the line spacing to 1.5, this makes the following tables easier to read
\clearpage  % Start a new page
\lhead{\emph{Abbreviations}}  % Set the left side page header to "Abbreviations"
\listofsymbols{ll}  % Include a list of Abbreviations (a table of two columns)
{
% \textbf{Acronym} & \textbf{W}hat (it) \textbf{S}tands \textbf{F}or \\
\textbf{LAH} & \textbf{L}ist \textbf{A}bbreviations \textbf{H}ere \\

}

%% ----------------------------------------------------------------
\clearpage  % Start a new page
\lhead{\emph{Physical Constants}}  % Set the left side page header to "Physical Constants"
\listofconstants{lrcl}  % Include a list of Physical Constants (a four column table)
{
% Constant Name & Symbol & = & Constant Value (with units) \\
Speed of Light & $c$ & $=$ & $2.997\ 924\ 58\times10^{8}\ \mbox{ms}^{-\mbox{s}}$ (exact)\\

}

%% ----------------------------------------------------------------
\clearpage  %Start a new page
\lhead{\emph{Symbols}}  % Set the left side page header to "Symbols"
\listofnomenclature{lll}  % Include a list of Symbols (a three column table)
{
% symbol & name & unit \\
$a$ & distance & m \\
$P$ & power & W (Js$^{-1}$) \\
& & \\ % Gap to separate the Roman symbols from the Greek
$\omega$ & angular frequency & rads$^{-1}$ \\
}
%% ----------------------------------------------------------------
% End of the pre-able, contents and lists of things
% Begin the Dedication page

\setstretch{1.5}  % Return the line spacing back to 1.3

\pagestyle{empty}  % Page style needs to be empty for this page
\dedicatory{For/Dedicated to/To my\ldots}

\addtocontents{toc}{\vspace{2em}}  % Add a gap in the Contents, for aesthetics


%% ----------------------------------------------------------------
\mainmatter	  % Begin normal, numeric (1,2,3...) page numbering
\pagestyle{fancy}  % Return the page headers back to the "fancy" style

% Include the chapters of the thesis, as separate files
% Just uncomment the lines as you write the chapters

\chapter{Introduction}
\label{chap:intro}
\lhead{\emph{Introduction}}
In the standard paradigm of the $\Lambda$CDM cosmology, dark matter haloes are formed from the gravitational collapse around initial overdensities - Galaxies are then formed the baryonic matter within the haloes - dark halo response to the galaxy formation - literature on adiabatic relaxation.

\section{Background and Context:}
\subsection{Dark matter haloes}

\subsection{Cosmological simulations}

\subsection{Simulations with galaxy formation}

\subsection{Relaxation response of dark matter}

\section{Objective}

% \section{Quasi-adiabatic relaxation}

\section{Self-similar systems}
\cite{2015LauNagaietal}

\section{Motivation}

\section{Dynamical relaxation}

\section{Outline}
 % Introduction

\chapter{Simulations of halo response}
\label{chap:sims-hals}

This chapter briefly describes the methods employed in studying halo relaxation through simulations. Currently, the most accurate description of the relaxation response of a halo to galaxy formation is provided by the hydrodynamical simulations of cosmological volumes that include comprehensive baryonic prescription for a variety of unresolved astrophysical processes.

\section{Hydrodynamical simulations with galaxies}
\label{sec:sims}
Here we describe the cosmological simulations employed in this work; these are from three different publicly available suites namely IllustrisTNG, EAGLE and CAMELS simulations.
\subsection{IllustrisTNG}
\label{sec:sims-IllTNG}
The IllustrisTNG simulations, conducted by the TNG collaboration, employed the \textsc{arepo} code \citep[][]{2020ApJS..248...32W}, which utilizes a moving mesh approach defined by Voronoi tessellation \citep[][]{2010MNRAS.401..791S}. These simulations incorporate an updated model of galaxy formation that includes cosmic magnetic fields in addition to major baryonic processes such as cooling, star formation, and stellar and AGN feedback \citep[][]{2017MNRAS.465.3291W,2018MNRAS.473.4077P}. The suite comprises three cosmological boxes: TNG50, TNG100, and TNG300, with periodic box sizes of $35 \Mpch$, $75 \Mpch$, and $200 \Mpch$, respectively, consistent with the cosmology from \cite{2016A&A...594A..13P}. Initial conditions were generated at $z=127$ using the Zel'dovich approximation \citep[][]{1970A&A.....5...84Z} with the \textsc{N-GenIC} code \citep[][]{2015ascl.soft02003S}. 

We utilize the highest resolution runs from all three boxes to study a wide range of halo responses to galaxy formation. Specifically, TNG50 offers sufficient resolution for low-mass haloes, while TNG300 provides an adequate number of cluster-scale haloes. Throughout our analysis, we utilize data from redshifts $z=0$ to $z=5$ from IllustrisTNG for both hydrodynamical and corresponding gravity-only runs. This allows us to examine the effects of baryonic processes on dark matter halo properties across different scales and epochs.

\subsection{EAGLE}
\label{sec:sims-EAGLE}
The EAGLE (Evolution and Assembly of GaLaxies and their Environments) cosmological simulations were conducted using a modified version of the \textsc{gadget-3} code, which employs smoothed particle hydrodynamics \citep[][]{2005MNRAS.364.1105S}. Initial conditions were generated using the \textsc{ic\_2lpt\_gen} code following \cite{2010MNRAS.403.1859J}. The main large-volume simulation was performed in a cosmological volume of $(100 ~\rm{Mpc})^3$ periodic box with its reference model of galaxy formation, incorporating sub-grid prescriptions for various baryonic processes such as cooling, star formation, and feedback mechanisms \citep[][]{2015MNRAS.446..521S,2015MNRAS.450.1937C}. This reference model of EAGLE simulations has been shown to produce realistic galaxies \citep[][]{2015MNRAS.448.2941S,2015MNRAS.450.4486F,2015MNRAS.452.2879T}. 

In addition, this suite includes multiple small-volume simulations with wide variations in the baryonic subgrid prescriptions for astrophysical processes. These simulations were performed at the same resolution as the large-volume reference simulation but in a $25 ~\rm{Mpc}$ periodic box. The variations include adjustments to the gas equation of state, the threshold for star formation, efficiency of stellar feedback, the viscosity of the black-hole accretion disk and the nature of stochastic heating caused by AGN feedback. In particular we study these following simulations:

\begin{itemize}
    \item \textbf{Ref} (Reference model): This simulation uses the standard EAGLE subgrid physics parameters but in this smaller box.
    \item \textbf{eos53} : This considers a gas equation of state $\gamma = 5/3$.
    \item \textbf{eos1} : This considers an isothermal gas equation of state $\gamma = 1$.
    \item \textbf{FixedSfThresh}: This uses a constant threshold for star formation independent of the metallicity.
    \item \textbf{WeakFB}: This follows a lower efficiency of stellar feedback that is scaled down by $50 \%$ compared to the reference simulation.
    \item \textbf{StrongFB}: This follows a higher efficiency of stellar feedback, twice that of the reference simulation.
    \item \textbf{NoAGN}: AGN feedback is disabled in this simulation to study the effects of stellar feedback alone.
    \item \textbf{AGNdT8}: In this, the temperature of gas raised by the AGN feedback heating is  lower with $\Delta T_{\rm{AGN}} = 10^8$K from the reference value of $\Delta T_{\rm{AGN}} = 10^{8.5}$K.
    \item \textbf{AGNdT9}: In this, the temperature of gas raised by the AGN feedback heating is higher with $\Delta T_{\rm{AGN}} = 10^9$K from the reference value of $\Delta T_{\rm{AGN}} = 10^{8.5}$K.
    \item \textbf{RefHR}: This simulation uses the same EAGLE subgrid prescription as `Ref' simulation but at a $8 times$ higher mass resolution.
    \item \textbf{RecalHR}: This simulation uses the same resolution as `RefHR' but with recalibrated EAGLE subgrid prescription.
\end{itemize}
We use the redshift $z=0$ data from this set of simulations along with their corresponding gravity-only run to study the role of different astrophysical processes on the relaxation response of the halo.



\subsection{CAMELS}
\label{sec:sims-CAMELS}
The Cosmology and Astrophysics with MachinE Learning Simulations (CAMELS) project comprises a comprehensive suite of hydrodynamical simulations designed to explore the interplay between cosmological and astrophysical parameters in shaping the universe's large-scale structure \cite[][]{CAMELS_presentation}. These simulations are performed in a relatively smaller cosmological volume of $(25 \ \mathrm{Mpc}/h)^3$, containing $256^3$ dark matter particles and an equivalent number of baryonic particles \cite{CAMELS_DR1}.

In our study, we specifically utilize the TNG suite's 1P set of simulations, a subset that methodically varies one parameter at a time to isolate the effects of individual parameters in the TNG model. For our analysis, we concentrate on parameters related to supernova (SN) feedback and active galactic nucleus (AGN) feedback, each governed by the following two distinct parameters:
% 
\begin{itemize}
    \item \textbf{Supernova Feedback Parameters:}
    \begin{itemize}
        \item $A_{\mathrm{SN1}}$: Varied between 0.25 to 4, this parameter controls the energy flux of the galactic winds. It is implemented as a prefactor for the overall energy output per unit star formation rate \cite{2018MNRAS.473.4077P,CAMELS_presentation}.
        \item $A_{\mathrm{SN2}}$: Varied between 0.5 to 2, this parameter controls the speed of the galactic winds. For a fixed $A_{\mathrm{SN1}}$, changes in $A_{\mathrm{SN2}}$ affect the galactic wind speed in concert with the mass-loading factor to maintain a fixed energy output.
    \end{itemize}
    \item \textbf{AGN Feedback Parameters:}
    \begin{itemize}
        \item $A_{\mathrm{AGN1}}$: Varied between 0.25 to 4, this parameter controls the overall power injected in the kinetic feedback mode of AGN. It is implemented as a prefactor for the energy per unit black-hole accretion rate \cite{2017MNRAS.465.3291W,CAMELS_presentation}.
        \item $A_{\mathrm{AGN2}}$: Varied between 0.5 to 2, this parameter controls the burstiness and the temperature of the heated gas during AGN feedback "bursts" by changing the wind speed of the AGN feedback.
    \end{itemize}
\end{itemize}
% 
All these parameters have a value of one in the fiducial simulation, and there are five simulations with higher values and another five simulations with lower values for each of these parameters. In total, we utilize these 41 hydrodynamical simulations and compare them against the gravity-only simulation over the same cosmological volume.



\subsection{Halo matching}
\label{sec:methods-match-ch:z0main}
To study how a dark halo responded to the galaxy forming in it, we need to first reliably match the catalogue of haloes found in the hydrodynamic simulation (which includes galaxy formation physics) to those found in its gravity-only run. For various numerical reasons a given halo in the hydro run may not have a true match in the halo catalogue of gravity-only run. So we first try to obtain an exhaustive catalogue of matched haloes that will be used to build a statistical description of this halo response.


\subsubsection{Matching procedure}
We match the haloes using the particle data associated with the haloes;
while the mass of each dark matter particle differs between the hydrodynamical and gravity-only runs, the \emph{number} of dark matter particles within the same initial periodic box is the same for each of the five pairs of simulations that we consider in IllustrisTNG and EAGLE. So a given particle in a hydro simulation has originated from the same region as the particle of the same ID in its corresponding gravity-only run.
For any given pair of haloes, with one in the hydrodynamical simulation and the other in the corresponding gravity-only run, we define the matching fraction of each of those two haloes (with respect to the other) 
as the fraction of its dark matter particles that are also present in the other halo. Below, we describe how we use 
these matching fractions to decide if the given pair can be considered a valid matched pair. 


\begin{figure}
    \centering
    \includegraphics[width=\linewidth]{plots/hal_match_efficiency_mass_all.pdf}
    \caption{Fraction of haloes in the TNG hydrodynamical simulations that have not found a match is shown as a function of mass $M_{200c}$. coloured vertical bands represent the mass range relevant for this work in each of the three TNG simulation boxes.}
    \label{fig:matching-loss-all-ch:z0main}
\end{figure}

\begin{figure}
\centering
\includegraphics[clip,trim={0.5cm 0cm 2cm 0.5cm}, width=\linewidth]{plots/visual_single_halo.pdf}
\includegraphics[clip,trim={0.5cm 0cm 2cm 0.5cm}, width=\linewidth]{plots/visual_single_halo_E.pdf}
\caption{Visually inspecting two matched FOF halo pairs, one each from IllustrisTNG \emph{(top row)} and EAGLE \emph{(bottom row)} using 2D-projected dark matter density field around the center of the hydrodynamical halo in a thick slice. The \emph{left panel} shows the halo in the hydrodynamical simulation and the \emph{right panel} shows the corresponding matched pair in the gravity-only simulation. The black circle shows the virial boundary of the gravity-only halo and blue circle shows that of the hydrodynamical halo. In both cases, the hydrodynamical halo is noticeably more spherical and compact than its gravity-only counterpart, with a spatially offset center. See text for a discussion.} 
\label{fig:single-halo-pair-ch:z0main}
\end{figure}

Computing the matching fractions 
for every pair of haloes is computationally expensive with $O(n^2)$ for millions of haloes in the catalogue. We decrease the complexity to $O(n)$ by supplying, for each halo in the hydro-simulation, an ordered list of most probable match candidates in the gravity-only run. These candidate lists of gravity-only haloes are generated and ranked based on the spatial positions of the haloes and their masses using a KD-tree based neighbour finding algorithm, implemented using \texttt{scipy.spatial.KDTree}.
For each halo in the hydro run, we test if the matching fraction of this halo with respect to any of the haloes in its match candidate list exceeds the value of 0.5. This ensures that at most one halo is selected as a match for each of the hydrodynamical haloes, so that our matched catalogue of halo pairs will be a subset of the source catalogue without repetitions.
While we match as many haloes as possible, it is also important to ensure that false matches don't plague our study. Based on the results in Appendix \ref{sec:apndx-matching-ch:z0main}, we therefore additionally require that in a valid matched pair, the gravity-only FOF halo must also have a matching fraction of more than 0.5 with respect to the hydrodynamical halo.
Our FOF based matching can be compared with
\citet[][]{2018MNRAS.481.1950L} where a similar matching algorithm has been followed for central subhaloes.

\subsubsection{Halo pair catalogue}
We generate a matched catalogue of haloes for each of the five simulations studied in this work, including FOF haloes resolved with more than 1000 particles\footnote{Mass resolution in the gravity-only runs are $7 \times 10^7 \Mh$, $8.9 \times 10^6 \Mh$ and $5.4 \times 10^5$ for TNG300, TNG100 and TNG50 respectively; whereas $9.7 \times 10^6 \Mh$ for the L100 and $1.21 \times 10^6 \Mh$ for the L25 simulation of EAGLE.}. The fraction of hydrodynamical haloes that fails to be part of the matched catalogue is shown in \figref{fig:matching-loss-all-ch:z0main} in bins of halo mass for the IllustrisTNG simulations. %
In the mass range in which the halo samples are selected for this work (see \secref{sec:results-mass-ch:z0main}), more than 96\% of the haloes in hydrodynamical simulation have been assigned a match in the gravity-only run. The small fraction of unmatched haloes primarily reside in dense environments 
where our algorithm presumably fails due to
the inherent issues with 3D FOF algorithm in dealing with mergers. Similar results hold for the EAGLE simulations as well. 
For illustrative purpose, a visual representation of two randomly chosen halo pairs, one each from IllustrisTNG and EAGLE, is shown in \figref{fig:single-halo-pair-ch:z0main}. 










\subsection{Methods to study the halo response}
\label{sec:method-ch:z0main}
The response of dark matter halo to galaxy formation has primarily two aspects, contraction or expansion of the halo towards the centre and a change in its triaxial shape. In this work, we study the former aspect of the halo response, by focusing on spherically averaged mass profiles. 
The illustrative haloes shown in \figref{fig:single-halo-pair-ch:z0main} become more compact and spherical in the hydrodynamical simulation that includes galaxy formation.
Also notice that there is an offset between the center-of-potential locations of matched pairs of haloes. These offsets are likely correlated with the halo tidal environment and will be interesting to follow-up in future work.

\subsubsection{Mass profiles}
\label{subsec:massprofiles-ch:z0main}

The overall expansion and/or contraction of dark matter in response to galaxy formation can be studied through the differences in spherically averaged mass profiles between matched haloes. For the dark matter, these radial profiles are obtained by adding up the mass of all dark matter particles contained within concentric spherical shells. In addition to these, we also need baryon mass profiles in modelling the dark matter response. While stellar mass profiles are computed in a similar fashion as dark matter,
for the gas mass profiles we use a Gaussian kernel to assign mass enclosed to each of the spherical shells\footnote{Throughout this work, we consider concentric shells defined by their radii and the mass enclosed by such a shell is the mass in the sphere bounded by that shell.
}. 
The width of this Gaussian kernel was taken to match the SPH smoothing length for the EAGLE simulation, whereas for IllustrisTNG we use the cube root of the Voronoi cell volume to define the kernel smoothing scale. We have tested that our results are robust to differences in the choice of this kernel.


\subsubsection{Quasi-adiabatic relaxation model}
\label{sec:methods-adiab-ch:z0main}
The impact of galaxy formation on the dark halo is expected to be primarily an adiabatic relaxation of dark matter particle orbits in response to baryon condensation \citep[][]{1986ApJ...301...27B}. We start by discussing this simplified model and study more complex effects such as the impact of baryonic feedback processes below.
Assuming that the dark matter halo is spherical and doesn't undergo shell crossing while baryons condense towards the centre, the adiabatic relaxation of any given dark matter shell is determined by the change in baryonic mass within that shell. 
Consider a shell enclosing a \emph{dark matter} mass $M_i^d(r_i)$ in radius $r_i$ in the unrelaxed halo. After relaxation, the radius of the shell changes to $r_f$. By definition, the dark matter mass $M_f^d(r_f)$ enclosed in $r_f$ in the relaxed halo is simply
\be 
M_f^d(r_f) = M_i^d(r_i)\,.
\label{eq:DMmass1-ch:z0main}
\ee
The \emph{total} mass $M_i(r_i)$ enclosed in $r_i$ in the unrelaxed halo, on the other hand, does not necessarily equal the total mass $M_f(r_f)$ enclosed in $r_f$ in the relaxed halo.
If angular momentum were to be conserved and the dark matter particle orbits stay circular, then the amount of relaxation of the shell is completely determined by the change in this total mass within the shell
\citep[][]{1986ApJ...301...27B},
\begin{align}
    r_i \,M_i(r_i) = r_f \,M_f(r_f) %
    \implies 
\frac{r_f}{r_i} = \frac{M_i(r_i)}{M_f(r_f)}\,. 
\label{eq:AR1-ch:z0main}
\end{align}
Extending this idealised scenario, quasi-adiabatic relaxation models consider the relaxation ratio $r_f/r_i$ as a function of the mass ratio $M_i/M_f$.
\begin{align}
\frac{r_f}{r_i} &= 1 + \chi \left( \frac{M_i(r_i)}{M_f(r_f)} \right) 
\label{eq:qAR1-ch:z0main}
\end{align}
For example, the baryonification procedures in \cite{2015JCAP...12..049S,2021MNRAS.503.4147P} include dark matter response as a quasi-adiabatic relaxation with
\be
\chi(y) = q\,(y-1)\,.
\label{eq:chi-linear-ch:z0main}
\ee

\subsubsection{The relaxation relation} %
\label{sec:methods-relx-reln-ch:z0main}
Our focus in this work is to characterise the relaxation relation \eqn{eq:qAR1-ch:z0main} as a function of halo and galaxy properties over a wide dynamic range; e.g, we would like to ask whether \eqn{eq:chi-linear-ch:z0main} is a good description of this relation.
To study this, we must extract this relation for individual haloes in hydrodynamical simulations.
For a given hydrodynamical halo in the matched catalog, we can obtain this relaxation relation by considering its matched halo in the gravity-only run to represent its unrelaxed state. 
We find it convenient to work with $r_f$ as a control variable. In this case, the values of $r_i$, $M_i(r_i)$ and $M_f(r_f)$ must be obtained from the matched halo pair, which can be done as follows.

For a dark matter shell at radius $r_f$ in the relaxed halo enclosing a dark matter mass of $M_f^d(r_f)$, its unrelaxed radius $r_i$ can be obtained by applying \eqn{eq:DMmass1-ch:z0main} and inverting the mass profile $M_i^d(r)=(1-f_b) M_i(r)$ of the gravity-only halo, where $f_b$ is the cosmic baryon fraction, to obtain 
\begin{align}
\label{eq:inv-mass-ch:z0main}
r_i = {M_i}^{-1} \left( \frac{M_f^d(r_f)}{(1-f_b)} \right)\,.
\end{align}
This is because each `particle' in the gravity-only halo consists of collisionless baryons and dark matter in precisely the proportion $f_{b}$. 
The value of $M_i(r_i)$ then follows from direct mass counting in the unrelaxed (i.e., gravity-only) halo in radius $r_i$, and the value of $M_f(r_f)$ follows from direct mass counting in the relaxed (i.e. hydrodynamical) halo in radius $r_f$, as described in \secref{subsec:massprofiles-ch:z0main}. In practice, we first obtain the unrelaxed mass profile $M_i(r_i)$ for a wide range of radii in finely spaced bins, in order to then compute the inverse in \eqn{eq:inv-mass-ch:z0main} by interpolation.

Thus, for any shell defined by its relaxed radius $r_f$, we can obtain both the relaxation ratio $r_f/r_i$ and the mass ratio $M_i/M_f$ from its unrelaxed radius computed from mass profiles. Hence we can obtain the relaxation relation by placing multiple concentric shells around the halo all the way to its virial radius $R_{\rm{vir}}$.
In Appendix \ref{appen:Mock-ch:z0main}, we have tested this algorithm on mock halo + galaxy systems generated with fixed known relaxation relations.


% \section{Techniques}
% \label{sec:methods}
\section{Haloes with relaxation}
In the simulations from IllustrisTNG and EAGLE suites, 3D friend-of-friends (FoF) algorithm \citep[see][for details]{2016A&C....15...72M,2019ComAC...6....2N} was used to obtain halo group catalogues. And the \textsc{subfind} code \citep{2001MNRAS.328..726S} was used to identify the subhaloes within these FoF group haloes as gravitationally bound substructures. Within each FoF group halo, the subhalo enclosing the gravitational potential minimum is assigned as its central subhalo. This trough in the gravitational potential is used to define the centre of that halo. Its size is characterized by the `virial' radius $R_{\rm vir}\equiv R_{\rm 200c}$ defined as the radius of the sphere around its centre enclosing a mean matter density that is 200 times the cosmological critical density. Its mass is then defined as the corresponding total mass enclosed $M\equiv M_{200c}$. In the simulations from the CAMELS suite, the haloes were identified in the  phase-space by a 6D FoF algorithm using the \textsc{rockstar} code. We characterize sizes and masses using the same quantities  $R_{\rm 200c}$ and $M_{\rm 200c}$ respectively.
% is defined as the `virial' radius $R_{\rm vir}\equiv R_{\rm 200c}$ of a given FoF group halo; while the total mass enclosed within this radius quantifies the mass of the halo $M\equiv M_{200c}$.

% Following our previous work \citep{2023Velmani&Paranjape}, 
To study the relaxation response of dark matter, we match the haloes from the full hydrodynamic simulations with the haloes in the corresponding gravity-only runs performed over same cosmological volumes. We identify these matched halo pairs based on the amount of overlap in their proto halo patches. In particular, this involves identifying nearby haloes of similar sizes between the hydrodynamical and gravity-only simulations and then match them by simulation particles \cite{2023Velmani&Paranjape}.
%  From this catalogue, we select populations of matched halo pairs by the logarithmic mass of the gravity-only halo $(\log(M/\Mh))$ in bins centred at $11.5, 12, 12.5, 13, 13.5, 14$ with a bin width of 0.3 at redshift $z=0.01$. While the small volume TNG50 offers well-resolved low-mass haloes $10^{11.5} \Mh$ and $10^{12} \Mh$, the TNG100 cosmological box gives $10^{12.5} \Mh$ and $10^{13} \Mh$ haloes and the largest volume TNG300 provides an adequate number of cluster-scale haloes of masses $10^{13.5} \Mh$ and $10^{14} \Mh$. 


\section{Relaxation Response Modeling}
\label{sec:methods-relmodel}
In the catalogues of matched halo pairs, hydrodynamical ones with galaxies are considered relaxed, while the gravity-only counterparts represent unrelaxed dark haloes. The relaxation response of dark matter within a halo, is generally evaluated through variations in their radial mass profiles indicating contraction or expansion due to the galaxy.  These spherically averaged dark matter profiles are computed by the cumulative sum of the mass contributed by all dark matter particles within concentric spherical shells. For the gravity-only halo, the cosmic dark matter fraction of the mass in each particle is considered as contributing to the dark matter. 

We employ the quasi-adiabatic relaxation framework to characterize the relaxation from these profiles.
% \subsection{Quasi-adiabatic relaxation}
% \label{sec:methods-adiab}
The relaxation response in cold dark matter occurs entirely due to gravitational interactions with baryons. This is an aggregate effect of the baryonic mass flow resulting from galactic processes such as inflows and feedback. The quasi-adiabatic relaxation model is a physically motivated framework that relates the change in the spherically averaged dark matter distribution at a given time to the spherically averaged baryonic distribution at the same time. This baryonic profile encompasses all non-dark matter mass, including gas and stars.

Earlier models assumed spherical halo where the dark matter particles maintain their radial ordering while responding adiabatically to baryonic particle flows \citep[][]{1986ApJ...301...27B}. In this scenario, if a dark matter particle initially at radius \( r_i \) in the unrelaxed halo moves to radius \( r_f \) in the relaxed halo, the enclosed dark matter mass within these radii remains equal.
\begin{equation} 
M_f^d(r_f) = M_i^d(r_i)\,.
\label{eq:DMmass}
\end{equation}

However, due to the baryonic mass flow, the total mass enclosed within these spheres is not necessarily equal, \( M_i(r_i) \neq M_f(r_f) \). Further assumption of angular momentum conservation for dark matter particles in circular orbits, implies that the change in total enclosed mass must be consistent with the amount of relaxation \citep[][]{1986ApJ...301...27B}.
\begin{align}
    r_i \,M_i(r_i) = r_f \,M_f(r_f) %
    \implies 
    \frac{r_f}{r_i} = \frac{M_i(r_i)}{M_f(r_f)}\,. 
\label{eq:AR}
\end{align}
% 
The \textbf{Quasi-adiabatic relaxation framework} empirically extends this idealized scenario by considering the relaxation ratio \( r_f/r_i \) as a function of the mass ratio \( M_i/M_f \).
\begin{align}
\frac{r_f}{r_i} &= 1 + \chi \left( \frac{M_i(r_i)}{M_f(r_f)} \right) 
\label{eq:qAR}
\end{align}
In a straightforward extension, the baryonification procedures described in \cite{2015JCAP...12..049S,2021MNRAS.503.4147P} incorporate dark matter response as a quasi-adiabatic relaxation with 
\be
\chi(y) = q\,(y-1)\,.
\label{eq:chi-linear}
\ee
Various quasi-adiabatic models have been proposed, offering different approaches to this framework \citep{2010MNRAS.407..435A,2004ApJ...616...16G,2023Velmani&Paranjape}.





\section{Choice of matching algorithm}
\label{sec:apndx-matching-ch:z0main}
In this work, we studied the response of dark matter halo to galaxy by comparing the haloes in hydrodynamical simulations to their counterparts in gravity-only runs. We described the matching procedure in \secref{sec:methods-match-ch:z0main}, here we discuss some of the specific choices in that procedure.
For this purpose, let us consider the FOF group haloes in TNG300 simulation with $\log M(\Mh)>10.5$.  There are 543588 FOF groups satisfying this criterion in the hydrodynamical run of TNG300, with each of them having more than 500 particles within their $R_{\rm vir}$ in the highest resolution run.
Following the matching procedure described in \secref{sec:method-ch:z0main} (requiring only that the matching fraction of the hydrodynamical halo with respect to the gravity-only halo is greater than 0.5), we get a matching halo in the gravity-only run for 541594 of them, leaving out only 1994 haloes unmatched, that is a negligible 0.4\% spread across the mass range (\figref{fig:efficiency-mass-ch:z0main}).
However, if we follow the same procedure using matching fraction between the central subhaloes instead of FOF group themselves, then we get a order of magnitude more unmatched haloes as can be seen in the \figref{fig:efficiency-mass-ch:z0main}.  

\begin{figure}
    \includegraphics[width=\linewidth]{plots/hal_match_efficiency_mass.pdf}
    \caption{Fraction of haloes in the TNG300-1 hydrodynamical simulation that have not found a match is shown as a function of mass $M_{200c}$.}
    \label{fig:efficiency-mass-ch:z0main}
\end{figure}

\begin{figure}
\centering
\includegraphics[width=\linewidth]{plots/hal_match_accuracy_hist2d.pdf}
\caption{Histogram of matching accuracy of haloes in the FOF group matched catalogue (left) and central subhalo matched catalogue (right). The x-axis is the fraction of dark matter particles in the hydrodynamical halo that is also in the gravity-only halo.}
\label{fig:accuracy-hist2d-ch:z0main}
\end{figure}

\begin{figure}
    \includegraphics[width=\linewidth]{plots/hal_match_efficiency_mass_rev.pdf}
    \caption{Fraction of haloes in the TNG300-1 hydrodynamical simulation that have not found a match is shown in left panel as a function of mass $M_{200c}$ after removing asymmetrical matches (see text in appendix \ref{sec:apndx-matching-ch:z0main}.}
    \label{fig:efficiency-mass-rev-ch:z0main}
\end{figure}

While obtaining matches for as many haloes as possible, we also have to ensure the quality of match. To check how well the haloes in each of the pairs in our catalogue are matching we look at the particle matching fraction of each halo in the pair with respect to the other as shown in \figref{fig:accuracy-hist2d-ch:z0main}. By definition the gravity-only halo in every pair has atleast 50\% of the dark matter particles of the hydrodynamical halo in that pair. But note that in FOF group matching based catalogue, for a significant number of pairs the hydrodynamical halo in the pair has less than 10\% of the gravity-only halo's particles. By visually inspecting some of those halo pairs we find that, they represent haloes with ongoing merger events. This explains the significant loss in matching efficiency when we used central subhalo matching. 

In the \figref{fig:efficiency-mass-rev-ch:z0main}, matching loss as a function of halo mass is shown after removing all those pairs in which the particle matching fraction of the gravity-only halo with respect to hydrodynamical halo is less than 50\%. Since this symmetrical matching condition produces consistent matched catalogue of haloes, we apply this additional matching condition  but stick to matching whole FOF groups. With this procedure, our final matched catalogue for TNG300 contains 524841 pairs, leaving 18747 out of 543588 hydrodynamical haloes unmatched. The additional unmatched haloes primarily reside in dense environments, and we can't find match for those haloes because of the inherent issues with 3d FOF algorithm in dealing with mergers. 


\section{Mock particles in a Galaxy-Halo system}
\label{appen:Mock-ch:z0main}
Here we validate our method used in extracting the relaxation relation (see \secref{sec:methods-relx-reln-ch:z0main}) using mock galaxy-halo systems.
In this, we use Hernquist profile for the initial unrelaxed dark matter halo and similar analytical mass profiles for the baryon components \citep[see appendix of][]{2021MNRAS.507..632P}. We then get the relaxed dark matter profile using quasi-adiabatic relaxation model with different values of $q$. Mock particle data is then generated with these mass profiles for different components of hydrodynamical halo and the corresponding gravity-only halo.

\begin{figure*}
    \centering
    (a) Mass profiles\\
    \includegraphics[clip,trim={0.2cm 0.5cm 0.4cm 0cm}, width=0.49\linewidth]{plots/mock_stargas_N=10000-Mr}
    \includegraphics[clip,trim={0.2cm 0.5cm 0.4cm 0cm}, width=0.49\linewidth]{plots/mock_stargas_N=100000-Mr}
    (b) Relaxation ratio\\
    \includegraphics[clip,trim={0.2cm 0.5cm 0.4cm 0cm}, width=0.49\linewidth]{plots/mock_stargas_N=10000-ratio}
    \includegraphics[clip,trim={0.2cm 0.5cm 0.4cm 0cm}, width=0.49\linewidth]{plots/mock_stargas_N=100000-ratio}
    \caption{Mock halo with Hernquist profile for unrelaxed dark matter and with the analytical baryon mass profile from \citet{2021MNRAS.507..632P} for gas and star components. The relaxed dark matter profile is modelled by quasi-adiabatic relaxation with different choices of q parameter. Haloes are sampled by \textbf{$10^4$} particles in left panel and \textbf{$10^5$} particles in the right panel.
        (a) Mass enclosed within spherical shell as a function of the shell radius for different components. 
        (b) Relaxation ratio as a function of the ratio of total mass enclosed. The linestyle indicate the different values for the model parameter $q$ used in generating the mock data, while the colourbar shows the final radius. While the solid lines represent the expected relaxation relation for each case, the colored markers show the computed relaxation relation from the mock particle data parametrized by the relaxed radius in units of the virial radius.}
    \label{fig:mock-ch:z0main}
\end{figure*}

For those mock halo pairs, we compute the mass profiles and obtain the relaxation ratio as a function of the dark matter shell radius as discussed in \secref{sec:method-ch:z0main}. We then repeat this for mock halo pairs generated with different particle resolutions. We find that the computed relaxation relation shown by colored markers matches with expected relaxation relation shown in solid lines; however, the choice of radial bins is limited by the particle resolution (see \figref{fig:mock-ch:z0main}).
% \section{Haloes}

% \section{Characterizing the halo response}

\chapter[Modern realistic simulations - across a variety of haloes at present epoch]{The quasi-adiabatic relaxation of haloes in the IllustrisTNG and EAGLE cosmological simulations at $z=0$}
\label{chap:z0_main}
% \lhead{\emph{Simulations of Halo Response}}
% \section{Introduction}
% \label{sec:intro-ch:z0main}
% \noindent

In this chapter, we perform a systematic, statistical study of the response in the radial distribution of dark matter due to baryonic processes with haloes identified in high-resolution cosmological simulations of galaxies. We perform this study at the present epoch (\(z=0\)) in the main simulations from the IllustrisTNG \citep[][]{2019ComAC...6....2N} and EAGLE \citep[][]{2017arXiv170609899T} suites. This includes TNG300, TNG100, and TNG50 simulated with the reference TNG model, and L0100N1504 and L025N0752 (denoted L100 and L25, respectively) simulated with the reference EAGLE model. These two different models of baryonic evolution include different subgrid prescriptions for various astrophysical processes. However, they both have produced state-of-the-art galaxies in the cosmological setting. Hence, all these simulations are expected to capture a reasonably accurate description of the backreaction on the dark matter haloes; see \secref{sec:sims} for details.

We isolate the effects of the galaxy formation process on dark matter haloes identified in these hydrodynamical simulations by comparing them against their matched partner haloes from their corresponding gravity-only runs that evolve the same initial cosmological volumes (see \secref{sec:hals} for details). In particular, we focus on the differences in the radial distribution for each of these halo pairs, characterized by the relaxation relations described in \secref{sec:char_relxn_reln-ch:sims}.

We begin by exploring the relaxation relations in a variety of individual halo pairs in \secref{sec:results-1-ch:z0main}. By stacking these relations in a population of haloes, we could capture the average relaxation behavior in the population. This allows us to study the dependence of relaxation on specific physical quantities of interest by excluding the statistical variation between individual haloes. In \secref{sec:results-mass-ch:z0main}, we study these stacked relaxation relations in populations of haloes selected within narrow bins of halo masses from $10^{10} \Mh$ to $10^{14} \Mh$.

While these stacked relations could be directly compared against existing quasi-adiabatic relaxation models such as Abadi et al. (2010) \citep{2010MNRAS.407..435A}, they are harder to interpret quantitatively. Instead, we demonstrate in \secref{sec:results-rad-dep-qadiab-ch:z0main} that considering an explicit dependence on the halo-centric distance allows a simple and accurate description of the relaxation behavior in halo populations. Using this method, we characterize the dependence of the dark matter relaxation response on a variety of relevant halo and galaxy properties in \secref{sec:dep-on-hal-gal-props-ch:z0main}.

These results are expected to be relevant for a variety of problems; for example, the change in the dark matter density profile of the halo caused by the galaxy formation affects the rotation curve of the galaxy. We discuss such applications in \secref{sec:applic-ch:z0main} and conclude in \secref{sec:conclusion-ch:z0main}. 





% \section{Results}
% \label{sec:results-ch:z0main}


\section{Relaxation of Haloes in the IllustrisTNG}
\label{sec:results-1-ch:z0main}

We find that the relaxation relation $r_f/r_i$ vs. $M_i/M_f$, estimated as described in \secref{sec:char_relxn_reln-ch:sims}, varies widely across haloes in the matched catalogue. In \figref{fig:relx-results-simple-ch:z0main}, we show the relaxation relation for four different samples of haloes selected by their unrelaxed mass from the IllustrisTNG simulations. 

The first two samples are from TNG50, with masses $M \sim 10^{11.5} \Mh$ and $10^{12} \Mh$, respectively. Similarly, the other two samples are from TNG100 and TNG300, with masses $M \sim 10^{12.5} \Mh$ and $10^{13.5} \Mh$, respectively. The relaxation relations of a few individual randomly chosen haloes from each sample are shown by grey lines; we also show stacked relaxation relations for each sample (see below for measurement details). The quasi-adiabatic relaxation model \eqn{eq:chi-linear-ch:sims} with $q=0.68$ and $q=0.33$ is shown by the dot-dashed and dashed purple lines, respectively, in each panel. The value $q=0.68$ was proposed by Schneider $\&$ Teyssier (2015) \citep{2015JCAP...12..049S} as being a reasonable description of cluster-sized haloes, while Paranjape $\&$ Sheth (2021) \citep{2021MNRAS.507..632P} argued that $q=0.33$ leads to a good description of the radial acceleration relation of Milky Way-sized spiral galaxies (see their Appendix A1). We will use these two models as reference points in the comparisons below. Since the samples shown are representative of the haloes in IllustrisTNG over a large mass range, it is clear that \emph{\eqn{eq:chi-linear-ch:sims} with a constant $q$ does not work for the majority of haloes in IllustrisTNG.} Similar results hold for EAGLE haloes as well. This motivates a systematic study of the relaxation relation as a function of halo mass and other properties.

For each of the four samples selected by halo mass, we compute the relaxation ratio $r_f/r_i$ and the enclosed total mass ratio $M_i/M_f$ at 20 concentric spherical shells for all haloes in the sample. We take the largest shell at the relaxed virial radius $r_f=R_{\rm{vir}}$, while the remaining 19 shells are taken at fixed values of $r_f/R_{\rm{vir}}$ for each halo. This allows us to stack the relaxation relation by simply taking the mean and standard deviation of the relaxation ratio and mass ratio at each of the 20 shells. While the physical size of the shell differs from halo to halo, we ensure that the smallest shell has a radius of at least 10 times the force smoothing length of the simulation. In \figref{fig:relx-results-simple-ch:z0main}, we show this stacked relaxation relation in large coloured markers, where the colour denotes the relaxed radius of the shell; and the error bar shown in red corresponds to the statistical error in the estimate of the mean value. By comparing with the small markers of the same colour, we can see that there is significant scatter not only in the relaxation ratio but also in the mass ratio at fixed $r_f/R_{\rm{vir}}$ across haloes in each sample.

To assess the level of systematic error introduced by our default choice of stacking technique, we also tested an alternate stacking definition, wherein we interpolate the relaxation relation of individual haloes to obtain the relaxation ratio at fixed values of mass ratios and stack them by ignoring the value of corresponding relaxed radii. However, this stacking method ignores radius information completely; we discuss the consequences of this later in \secref{sec:results-rad-dep-qadiab-ch:z0main}.

\begin{figure}
    \centering
    \includegraphics[width=\linewidth,trim={0 0 0cm 0},clip]{plots/indv_relx-reln_all.pdf}
    \caption{Relaxation relation for 4 different samples of haloes selected by mass from the IllustrisTNG simulations. The large coloured circles denote the stacked relaxation ratio and total mass ratio at 20 different shells, whose radii are indicated by the colour. Small coloured markers joined by grey lines show the relaxation relation of a few randomly chosen individual haloes in each of the samples. The black curves denote the radius-independent stack of the relaxation relation for each sample (see text). The quasi-adiabatic relaxation model \eqn{eq:chi-linear-ch:sims} with $q=0.68$ and $q=0.33$ are shown by the dot-dashed and dashed purple lines, respectively, in each panel.} 
    \label{fig:relx-results-simple-ch:z0main}
\end{figure}




\section{Trend in relaxation relation with halo mass}
\label{sec:results-mass-ch:z0main}
As can be already noted in \figref{fig:relx-results-simple-ch:z0main}, the relaxation relation shows very different behaviour at different mass scales. In this section, we focus on the stacked relation (using our default stacking definition) and study how it varies as a function of unrelaxed halo mass. For this, we consider nine mass bins starting from $\log (M/\Mh) = 10$ to $14$ in steps of $0.5$ dex. We list the colour labels used for these mass bins in \figref{fig:mass_bin_label-ch:z0main}; this colour-coding will be used in all subsequent plots. None of the five simulations considered, simultaneously provides a sufficiently large sample of cluster-scale haloes and well-resolved low-mass haloes.
In the IllustrisTNG suite, we use the smallest box TNG50 to study haloes with mass $10^{10} \Mh < M < 10^{12} \Mh$, whereas we use TNG100 and TNG300 to study haloes with mass $10^{11} \Mh < M < 10^{12.5} \Mh$ and $10^{12} \Mh < M < 10^{14} \Mh$ respectively. At those mass bins where multiple IllustrisTNG boxes provide halo samples, the smaller box provides a smaller sample but with better resolution. For computational ease, we limit the size of each sample to be $\leq500$ haloes, as we find that the statistics are well-converged with this number.

\begin{figure}
    \centering
    \includegraphics[width=0.69\linewidth]{plots/Mass_bin_labels.pdf}
    \caption{Representative colours we use to denote each of the halo mass bins.}
    \label{fig:mass_bin_label-ch:z0main}
\end{figure}

By repeating the procedure described in \secref{sec:results-1-ch:z0main}, we obtain both the fixed-radius (default) stack and radius-independent (alternate) stack of the relaxation relation for each of the halo samples taken from IllustrisTNG at these nine mass bins (see \emph{left panel} of \figref{fig:fit-view-mass-indep-ch:z0main}). 
For reference, note that the case of no relaxation would correspond to a horizontal line at unity in this plot.
Relaxation is strongest for Milky Way-scale haloes, as indicated by the small values of the relaxation ratio for $M\sim10^{12}\Mh$; we discuss the physical implications of this result later. 
Note that the simple quasi-adiabatic relaxation model \eqn{eq:chi-linear-ch:sims} with $q=0.68$ \citep{2015JCAP...12..049S} fails to explain the relaxation relation for any of the halo masses considered; however this model with $q=0.33$ is reasonably close to the relaxation relation at %
$M\sim 10^{13} \Mh$. 
And while the quadratic model proposed by Abadi et al. (2010) \citep{2010MNRAS.407..435A} matches with the relaxation relation of $10^{12.5} \Mh$ haloes in IllustrisTNG, this is possibly a coincidence given that this model was built using zoom simulation of haloes in the mass range $10^{11.5} $-$ 10^{12} \Mh$, which show a very different relaxation relation in IllustrisTNG.\footnote{The simulation used by Abadi et al. (2010) \citep{2010MNRAS.407..435A} also suffered from overcooling due to the lack of feedback effects, so that the mass ratios attained much smaller value for shells at the same radii as compared to IllustrisTNG.}

\begin{figure}
\centering
\includegraphics[width=0.49\linewidth]{plots/fit_view_M_T.pdf}
\includegraphics[width=0.49\linewidth]{plots/fit_view_M_E.pdf}
\caption{The stacked relation between relaxation ratio and mass ratio as a function of halo mass in IllustrisTNG (left panel) and EAGLE (right panel) simulations. Here the points and solid lines represent two different stacking methods as in Fig.~\ref{fig:relx-results-simple-ch:z0main}. The colour-coding follows Fig.~\ref{fig:mass_bin_label-ch:z0main}.} 
\label{fig:fit-view-mass-indep-ch:z0main}
\end{figure}








We also take six samples of haloes from the EAGLE simulation, in mass bins $\log (M/\Mh) = 10.5, 11,11.5$  from the small, high-resolution L25 box and in mass bins $12, 12.5, 13$ from the main L100 box. Here too, the $q=0.68$ model fails for all masses, but $q=0.33$ model works reasonably for  $M\sim10^{13} \Mh$ haloes (see the \emph{right panel} of \figref{fig:fit-view-mass-indep-ch:z0main}). We find that, despite having a different galaxy formation model, the relaxation relation for haloes found in the primary EAGLE run L100 is consistent with the results from IllustrisTNG. 
IllustrisTNG samples reach lower values of the relaxation ratio and mass ratio than EAGLE because of the better resolution available. For $M_{200}=10^{12} \Mh$, the mean relaxation relation shown in \figref{fig:fit-view-mass-indep-ch:z0main}, does not seem to be very different between IllustrisTNG and EAGLE $L100$, atleast not anymore than the difference between different boxes of the IllustrisTNG. However, the haloes from EAGLE $L25$ simulation shows a unique behaviour where the relaxation ratio increases with decrease in mass ratio in the innermost regions. We expect that this might be due to the fact that the EAGLE reference model required recalibration at this higher resolution. In a future work we will explore how the dark matter response depends on such variations in the baryonic prescription.


\section{Parametrised model of quasi-adiabatic relaxation}
\label{sec:results-rad-dep-qadiab-ch:z0main}
In this section, we model the relaxation relations discussed above, with a focus on conveniently quantifying this response across a wide range of halo masses.



\subsection{Expectations from simulation measurements}
\label{subsubsec:sim-relax-ch:z0main}
In both IllustrisTNG and EAGLE, for all masses other than $10^{13} \Mh$, the simple quasi-adiabatic relaxation model \eqn{eq:chi-linear-ch:sims} fails to explain the measured relation with any value of $q$. 
As seen in Fig.~\ref{fig:fit-view-mass-indep-ch:z0main}, an important aspect of this mismatch is caused by the model's requirement that shells which hold their baryonic mass fixed (i.e., for which $M_i/M_f=1$) must necessarily also hold their radius fixed ($r_f/r_i=1$), and vice-versa. The measurements, however, show substantial offsets in the relaxation ratio from unity for shells with $M_i/M_f = 1$, and also substantial offsets in the mass ratio from unity for shells with $r_f/r_i=1$, across nearly the entire range of halo mass. One way of understanding this effect physically is due to feedback-related baryonic outflows: a particular shell which maintains its radius after relaxation ($r_f/r_i=1$), could still lose its baryonic mass due to outflows, resulting in $M_i/M_f > 1$ \citep[][]{2022MNRAS.511.3910F}. Alternatively, the interplay between cooling-related condensation (which increases baryonic mass in a given shell) and feedback-related outflows (which decrease baryonic mass) could result in a situation where the baryonic mass after relaxation retains its initial value despite an overall relaxation, e.g. due to approximate angular momentum conservation, leading to $r_f/r_i < 1$ while $M_i/M_f=1$. These trends are visible  in  Fig.~\ref{fig:fit-view-mass-indep-ch:z0main}  for haloes with $M<10^{13}\Mh$. Fig.~\ref{fig:relx-results-simple-ch:z0main} shows that the former trend ($M_i/M_f > 1$ when $r_f/r_i=1$) occurs in the halo outskirts ($r_f\sim R_{\rm vir}$) and the latter ($r_f/r_i < 1$ when $M_i/M_f=1$) in the inner halo ($r_f\lesssim0.3\,R_{\rm vir}$), for $M<10^{13}\Mh$. 
On the other hand, more massive haloes show little to no relaxation in both inner halo (where there is net baryonic inflow, $M_i/M_f < 1$) and outer halo (where there is net baryonic outflow, $M_i/M_f > 1$).

To account for such effects, we expand the simple quasi-adiabatic relaxation model \eqn{eq:chi-linear-ch:sims} by adding 
a null offset parameter $q_0$:
\begin{align}
    \label{eq:chi-linear-q0-ch:z0main}
    \frac{r_f}{r_i} - 1 &= q_1 \left( \frac{M_i(r_i)}{M_f(r_f)} - 1 \right) + q_0\,.
\end{align}
With this model, the ratio of angular momenta of the dark matter particles in approximately circular orbits before and after relaxation can be expressed simply as follows (with $L_i$ and $L_f$ denoting the angular momenta of the unrelaxed and relaxed shell, respectively),
\begin{align}
\left( \frac{L_f}{L_i} \right)^2 &= \frac{M_f}{M_i} \frac{r_f}{r_i}\\
&= \frac{M_f}{M_i} \left[ q_1 \left( \frac{M_i}{M_f} - 1 \right) + q_0 + 1 \right]\\
\label{eq:Lf-Li-ratio-ch:z0main}
&= (1 + q_0 - q_1) \frac{M_f}{M_i} + q_1
\end{align}
For example, the special case $q_0=-(1-q_1)$ can be thought of as a natural generalisation of the original adiabatic relaxation model, because in this case we have $L_f/L_i = \sqrt{q_1}$, relating $q_1$ directly to angular momentum loss or gain.
Below, however, we will see that there is no simple relation between $q_1$ and $q_0$ for generic measurements in the simulations. In general, then, one can only say that a particular shell has gained or lost angular momentum when the value of $(1 + q_0 - q_1) (M_f/M_i)$ is, respectively, larger or smaller than $1-q_1$.

However, the above holds true only when the dark matter particles are in circular orbits. When galactic processes lead to changes in the baryonic mass profile, even the dark matter particles in circular orbit can go into elliptical orbits \citep[see, e.g.][]{2005ApJ...634...70S}. For example, when there is a sudden expulsion of gas due to feedback events, the total mass enclosed decreases and the particles start moving radially outward. During this period the mass ratio $M_i/M_f$ can become greater than one and still have no relaxation (i.e. $r_f/r_i=1$) as discussed in the start of this section.

While this extended linear model can describe the relaxation relation at few other halo masses (see for example $10^{12.5}\Mh$ halos in both left and right panel of \figref{fig:fit-view-mass-indep-ch:z0main}), even this model fails at many halo masses.
Moreover, we have checked that there is no simple polynomial model favoured by standard information criteria such as AICC \citep[][]{2007MNRAS.377L..74L}
to describe the relaxation relation at all masses.
Rather,  we find that, if we 
simply elevate $q_0$ and $q_1$ in \eqn{eq:chi-linear-q0-ch:z0main} to functions of $r_f/R_{\rm vir}$, then this model
can be applied at all the mass scales that we consider. For this, we need the relaxation relation at fixed relaxed radius, which we obtain as follows. We measure the relaxation ratio and mass ratio of shells having fixed $r_f/R_{\rm vir}$ for all haloes in a selected sample, then stack them in bins
of mass ratio at each spherical shell separately. 
We find that \eqn{eq:chi-linear-q0-ch:z0main}
is consistent with the relaxation relation of all halo masses considered, where we infer the values of $q_0$ and $q_1$ at each $r_f$ using standard least squares fitting (the reduced $\chi^2$ values are always close to unity, for all masses and radial shells).

\begin{figure}
    \centering
    \includegraphics[width=0.48\linewidth]{plots/fit_show_rf_M_T50_M11.pdf}
    \includegraphics[width=0.48\linewidth]{plots/fit_show_rf_M_T50_M12.pdf}
    \includegraphics[width=0.48\linewidth]{plots/fit_show_rf_M_T100_M12.5.pdf}
    \includegraphics[width=0.48\linewidth]{plots/fit_show_rf_M_T300_M14.pdf}
    \includegraphics[width=0.48\linewidth]{plots/fit_show_rf_M_E25_M11.pdf}
    \includegraphics[width=0.48\linewidth]{plots/fit_show_rf_M_E100_M12.5.pdf}
    \caption{Relaxation relation stacked separately at 5 different radii indicated by color in six samples of haloes selected by mass from IllustrisTNG and EAGLE simulations. We also show linear polynomial fit to this relation, following \eqn{eq:chi-linear-q0-ch:z0main} with the best-fit values for the parameters $q_0$ and $q_1$ at each of the selected radii for each of the six halo samples.} %
    \label{fig:rf-fit-show-ch:z0main}
\end{figure}

In \figref{fig:rf-fit-show-ch:z0main} we show the measured relaxation relation for six different sample of haloes at shells of selected radii, compared with the best-fit model \eqn{eq:chi-linear-q0-ch:z0main} for each case; the model clearly describes these measurements extremely well. We already noted from \figref{fig:fit-view-mass-indep-ch:z0main}, that the haloes in the small volume EAGLE simulation with the reference model shows a different relaxation behaviour. This is also apparent in \figref{fig:rf-fit-show-ch:z0main}, between the $10^{11} \Mh$ haloes from that L25 simulation (last row, left panel) and the haloes of the same mass from IllustrisTNG (first row, left panel); however they both follow the linear relaxation relation at fixed radii. An interesting feature to note is the dramatic change in slope $q_1$ for the most massive haloes (middle row, right panel), with $q_1$ changing sign as one moves outwards through the halo. This also qualitatively explains the non-monotonicity and multi-valued nature of the default stacks in the lower right panel of Fig.~\ref{fig:relx-results-simple-ch:z0main}.


\begin{figure}
    \centering
    \includegraphics[width=0.49\linewidth]{plots/fit_params_rf_M_T.pdf}
    \includegraphics[width=0.49\linewidth]{plots/fit_params_rf_M_E.pdf}
    \caption{Linear quasi-adiabatic relaxation model parameters as a function of the radius of relaxed halo at different halo masses in IllustrisTNG \emph{(upper panel)} and EAGLE \emph{(lower panel)}. The colour-coding follows Fig.~\ref{fig:mass_bin_label-ch:z0main}. See text for details.}
    \label{fig:rf-fit-params-ch:z0main}
\end{figure}


\subsection{Modelling the radial dependence of the relaxation relation}
As described above, we obtained the best fit parameters $q_0$ and $q_1$ of the relaxation relation at each relaxed radius $r_f$, this is shown in \figref{fig:rf-fit-params-ch:z0main} as a function of $r_f/R_{\rm vir}$. 

For haloes of mass $M<10^{13}\Mh$, the $q_1$ parameter increases monotonically with radius and this dependence can be modelled as 
\begin{align}
q_1 (r_f) = q_{10} + q_{11} \log \left( r_f/R_{\rm vir} \right) \,,
\label{eq:q1(r_f)-ch:z0main}
\end{align}
where $q_{10}$ and $q_{11}$ are constants.
The parameter $q_0$, on the other hands, remains relatively constant at small negative values for each halo sample.
This means the factor $(1 + q_0 - q_1)$ starts with a positive value in the inner halo and becomes negative in the outer halo, inverting the relationship between change in angular momentum and mass ratio (see equation~\ref{eq:Lf-Li-ratio-ch:z0main}). And due to $q_0$ being small in magnitude, the radius at which this transition happens roughly satisfies the condition $L_f/L_i=1$. 

For cluster-scale haloes, this simple monotonic dependence of $q_1(r_f)$ is replaced with oscillatory behaviour. In fact, some of the peaks in $q_1$ correspond to $q_1\approx1$; combined with $|q_0|\ll1$, this indicates that these peaks are shells which nearly perfectly conserve angular momentum. (E.g., this happens at $r_f/R_{\rm vir}\simeq0.5\,(0.2)$ for $M=10^{13.5}\,(10^{14})\Mh$.)
This is in strong contrast to \figref{fig:fit-view-mass-indep-ch:z0main} where the slope of the relaxation relation represented by a globally defined $q_1$ (e.g., equation~\ref{eq:chi-linear-q0-ch:z0main} without explicit $r_f$ dependence in the parameters) is close to zero for these haloes, indicating maximum deviation from the adiabatic relaxation. This complicated behaviour in the cluster-scale haloes could be due to the presence of substructures. The $q_0$ parameter now also shows interesting behaviour; it is only slightly negative in the outer halo but becomes close to zero in the inner halo for these haloes ($q_0$ was relatively constant with more negative value for less massive haloes).
 

Excluding those cluster scale haloes, we propose a three parameter model as an extension to the quasi-adiabatic relaxation model, where the relaxation ratio depends linearly on the mass ratio as in \eqn{eq:chi-linear-q0-ch:z0main}, however the slope of this relationship has explicit logarithmic dependence on the radius:
\begin{align}
\label{eq:q3-model-ch:z0main}
\frac{r_f}{r_i} - 1 &=  \left[ q_{10} + q_{11} \log \left( \frac{r_f}{R_{\rm vir}} \right) \right] \left( \frac{M_i(r_i)}{M_f(r_f)} - 1 \right) + q_0
\end{align}
Using this model we can quantify the response of these haloes to galaxy formation;
in \figref{fig:3-param-mass-only-ch:z0main}, we show these 3 parameters estimated as a function of halo mass for IllustrisTNG and EAGLE. We see that each of the parameters is nearly mass-independent for $M\lesssim10^{12}\Mh$, showing significant trends with mass only above $M\gtrsim10^{12.5}\Mh$ (with the exception of $10^{10.5} \Mh$ in EAGLE). This simplified but accurate relaxation model can be of great use in modelling the rotation curves of low-mass and Milky Way-like galaxies \citep{2021MNRAS.503.4147P,2021MNRAS.507..632P}, which we will explore in future work.

\begin{figure}
    \centering
    \includegraphics[width=.7\linewidth]{plots/fit_param_q3s_M_TE.pdf}
    \caption{Fitting values for the three parameters namely $q_{0}$, $q_{10}$ and $q_{11}$ in radially dependent quasi-adiabatic relaxation model described by equation~\ref{eq:q3-model-ch:z0main} as a function of halo mass in the three IllustrisTNG simulations and the two EAGLE simulations.}
    \label{fig:3-param-mass-only-ch:z0main}
\end{figure}


\section{Properties beyond halo mass}
\label{sec:dep-on-hal-gal-props-ch:z0main}
In this section, we study our parametrised model of the response of dark matter in the halo as a function of halo and galaxy properties beyond halo mass. This is motivated by the fact that there is a significant scatter in the relaxation relation (see Fig.~\ref{fig:relx-results-simple-ch:z0main}), even within halo samples selected by mass. Such a study would also have implications for halo assembly bias and similar environmental correlations predicted in the $\Lambda$CDM framework  \citep[see, e.g., the discussion in][]{2021arXiv211200026P}. 


For the simulations from TNG suite, we focus on the following halo properties in addition to their mass. For the haloes in the gravity-only runs, we define the concentration as $c=2.1626 \times R_{\rm vir}/R_{V_{\rm{max}}}$, where $R_{V_{\rm{max}}}$ is the radius at which the rotation curve attains its peak. If the sphericalised mass profile of these haloes had a perfectly NFW form, this concentration would be exactly equal to the standard NFW concentration defined in terms of NFW scale radius \citep[see equation 5 of ][]{1996ApJ...462..563N}. 
This definition of concentration is convenient since it does not require any statistical fit to the measured halo profile.
For the haloes in hydrodynamic simulations, we consider 
three different properties, namely, \textbf{gas fraction} $(f_g)$ of the whole FOF group, and the \textbf{stellar mass fraction} $(f_{\ast})$ and specific star formation rate \textbf{(SSFR)} of the central subhalo associated to each of them. 
\begin{align}
    f_{g} = \frac{M_{g}^{\rm{F}}}{M^{\rm{F}}}\,; \quad
    f_{\ast} = \frac{M_{\ast}^{\rm{S}}}{M^{\rm{S}}}\,; \quad {\rm SSFR} = \frac{\rm{SFR}}{M_{\ast}^{\rm{S}}}\,.
\label{eq:galpropdefs-ch:z0main}
\end{align}
Here $M^{\rm{F}}$ ($M_{g}^{\rm{F}}$) denotes the total mass of all (gas) particles in the FOF group, whereas $M^{\rm{S}}$ ($M_{\ast}^{\rm{S}}$) denotes the total mass of all (stellar) particles in the associated central subhalo. 
Finally, SFR denotes the sum of star formation rate of all gas cells in the central subhalo. The above definitions allow us to track the response of dark matter to the gas content of the full halo and the stellar content and activity of its central galaxy. We have also checked that using the stellar content and activity of the full halo leads to qualitatively similar results as when using the central galaxy alone. All these four halo/galaxy properties show overall trend with halo mass, however, there is also a considerable scatter even at fixed mass scale, this is illustrated in \figref{fig:halo-prop-pers-data-ch:z0main}. 

\begin{figure}[htbp]
    \centering
    \includegraphics[clip,trim={0.2cm 1.18cm 0.25cm 0.3cm},width=0.4\linewidth]{plots/percentile_data_M-cs_T.pdf}
    \includegraphics[clip,trim={0.2cm 1.18cm 0.25cm 0.3cm},width=0.4\linewidth]{plots/percentile_data_M-fs1_T.pdf}
    \includegraphics[clip,trim={0.2cm 0cm 0.25cm 0.3cm},width=0.4\linewidth]{plots/percentile_data_M-ssfr1_T.pdf}
    \includegraphics[clip,trim={0.2cm 0cm 0.25cm 0.3cm},width=0.4\linewidth]{plots/percentile_data_M-fg_T.pdf}
    \caption{The 10th, 50th and 90th percentile of the four different halo properties in each of the sample selected by mass from three different cosmological boxes of the IllustrisTNG.
    This includes the concentration ($c$) of the unrelaxed halo, and the stellar fraction ($f^{\ast}$), specific star formation rate (SSFR) and gas fraction ($f_g$) of the hydrodynamical halo, all of which are defined in main text.} %
    \label{fig:halo-prop-pers-data-ch:z0main}
\end{figure}

For all subsamples selected by a secondary halo/galaxy property at fixed halo mass, we use the 3-parameter model discussed above in the mass range $10^{10}$ to $10^{12.5}$ $\Mh$, 
while for cluster-scale haloes, where our 3-parameter model fails, we directly compare the linear quasi-adiabatic relaxation model parameters $q_0$ and $q_1$ as a function of the scaled halo-centric radius $r_f/R_{\rm vir}$. The results for low-mass (massive) haloes are shown in Fig.~\ref{fig:fit-fit-func-q-ch:z0main} (Fig.~\ref{fig:fit-func-rf-13514-ch:z0main}).


For reference, the upper panels of Fig.~\ref{fig:fit-fit-func-q-ch:z0main} show the best-fit values of $q_0$, $q_{10}$ and $q_{11}$ as a function of halo mass alone for haloes with $M\leq10^{13}\Mh$,
which repeat the corresponding curves in Fig.~\ref{fig:3-param-mass-only-ch:z0main}. The upper panels of Fig.~\ref{fig:fit-func-rf-13514-ch:z0main} similarly repeat the results for $q_0(r_f)$ and $q_1(r_f)$ for the mass bins $M=10^{13},10^{13.5},10^{14}\Mh$ from Fig.~\ref{fig:rf-fit-params-ch:z0main}. By displaying both, the full radial dependence as well as the 3-parameter description for the mass bin $10^{13}\Mh$, we can assess the reliability of the latter around the mass scale where it begins to fail. We repeat this for subsamples split by secondary halo/galaxy properties below.



\subsection{Dependence on unrelaxed halo concentration}
Unrelaxed haloes at fixed mass, as found in gravity-only simulations are known to have universal mass profiles characterised by their concentration, together with their mass.
As can already be noted in \figref{fig:halo-prop-pers-data-ch:z0main}, this NFW concentration is correlated with the halo mass \citep[see e.g. ][]{2006ApJ...652...71W,2007MNRAS.378...55M,2015ApJ...799..108D,2017MNRAS.468.2984P}.
In order to isolate the effect of concentration on the response, we define \textbf{concentration significance} $(c_s)$.
\begin{align}
c_s = \left(\log c- \log \bar{c}(M)\right)/\sigma\,. \nonumber
\end{align}
Here we use the median $\bar{c}(M)$ and scatter $\sigma$ of the concentration-mass relation as given by Diemer et al. (2019) \citep{2019ApJ...871..168D} and computed with the COLOSSUS code \citep[][]{2018ApJS..239...35D}.

Then we select haloes from each of the mass bins in three separated $c_s$ percentile bins $(10\pm10, ~50\pm10 ~\&~ 90\pm10)$  and compute the relaxation relation as described in previous sections for each of those samples. 
We find that $q_0$ shows strong dependence on the concentration, with more concentrated haloes having higher value of $q_0$ at most halo masses with $M \leq 10^{13} \Mh$ (see second row in \figref{fig:fit-fit-func-q-ch:z0main} and second row, first column in \figref{fig:fit-func-rf-13514-ch:z0main}). This can be understood in terms of the formation time of the halo, since concentration is correlated with the formation time. More concentrated haloes, that have formed earlier might have had enough time for the dark matter to respond to the baryonic feedback, and hence there is less offset. On the other hand, $q_{10}$ or $q_{11}$ display a more complex dependence at low mass, with no clear monotonic trend. Meanwhile, for cluster-scale haloes, $q_1$ shows a very different behaviour as a function of $r_f$ at different halo concentrations (see second row of \figref{fig:fit-func-rf-13514-ch:z0main}). We leave a fuller exploration of these trends, particularly their dependence on substructure properties, to future work.



\begin{figure}
    \centering
    \includegraphics[width=0.32\linewidth]{plots/fit_param_q0_M_T.pdf}
    \includegraphics[width=0.32\linewidth]{plots/fit_param_q10_M_T.pdf}
    \includegraphics[width=0.32\linewidth]{plots/fit_param_q11_M_T.pdf}
    \includegraphics[width=0.32\linewidth]{plots/fit_param_q0_M-cs_T.pdf}
    \includegraphics[width=0.32\linewidth]{plots/fit_param_q10_M-cs_T.pdf}
    \includegraphics[width=0.32\linewidth]{plots/fit_param_q11_M-cs_T.pdf}
    
    \includegraphics[width=0.32\linewidth]{plots/fit_param_q0_M-fs1_T.pdf}
    \includegraphics[width=0.32\linewidth]{plots/fit_param_q10_M-fs1_T.pdf}
    \includegraphics[width=0.32\linewidth]{plots/fit_param_q11_M-fs1_T.pdf}
    
    \includegraphics[width=0.32\linewidth]{plots/fit_param_q0_M-ssfr1_T.pdf}
    \includegraphics[width=0.32\linewidth]{plots/fit_param_q10_M-ssfr1_T.pdf}
    \includegraphics[width=0.32\linewidth]{plots/fit_param_q11_M-ssfr1_T.pdf}
    
    \includegraphics[width=0.32\linewidth]{plots/fit_param_q0_M-fg_T.pdf}
    \includegraphics[width=0.32\linewidth]{plots/fit_param_q10_M-fg_T.pdf}
    \includegraphics[width=0.32\linewidth]{plots/fit_param_q11_M-fg_T.pdf}
    \caption{Radially dependent quasi-adiabatic relaxation model parameters $q_{0}$, $q_{10}$ and $q_{11}$ estimated as a function of halo properties in IllustrisTNG simulations. In the top row panels, only halo mass dependence is shown; whereas in the next three rows, we show the dependence on halo concentration, stellar mass fraction, specific star formation rate and gas fraction in terms of percentiles respectively.} %
    \label{fig:fit-fit-func-q-ch:z0main}
\end{figure}


\subsection{Dependence on baryonic halo properties}
We now shift our focus to hydrodynamical halo properties. In this regard, we study the response as a function of the three halo properties defined above; namely the specific star formation rate (SSFR) at current redshift $z=0$ and the total stellar mass fraction ($f_{\ast}$) and gas fraction ($f_g$) at redshift $z=0$ which respectively represent the integrated star formation activity and gas content of the central subhalo. At each halo mass bin, we take three subsamples selected by bins of percentiles in $f_{\ast}$, SSFR and $f_g$, in a similar fashion as with concentration significance. %

From the third column of \figref{fig:fit-fit-func-q-ch:z0main}, we note that $f_\ast$ does not affect the relaxation response significantly; in particular the $q_0$ parameter is relatively least dependence on $f_{\ast}$, compared to other halo properties. This is consistent with the fact that the $q_0$ parameter clearly converges between three TNG boxes (see upper panel of \figref{fig:fit-fit-func-q-ch:z0main}), despite large differences in $f_\ast$ with resolution (see upper right panel of \figref{fig:halo-prop-pers-data-ch:z0main}).
On the other hand, we can see a clear trend  in $q_0$ parameter with SSFR (see the first column in the third row of \figref{fig:fit-fit-func-q-ch:z0main}); at a given halo mass, the $q_0$ value is closer to zero when the star formation activity is lower. To recall, $q_0 \simeq 0$ would mean no offset in the relaxation relation, and in that case for shells having no relaxation, the mass ratio is unity indicating that the enclosed baryonic mass also remains same. This result is consistent with our argument in \secref{sec:results-rad-dep-qadiab-ch:z0main} that the $q_0<0$ is caused by the recent baryonic outflows due to feedback which is lower in these low mass haloes when SSFR is low.
From the first panel in the last row of \figref{fig:fit-fit-func-q-ch:z0main}, we can see a similar trend in $q_0$ with gas fraction $f_g$ of the halo; this is likely due to the fact that the FOF haloes with more gas have relatively higher active star formation with larger recent baryonic outflows. However, even the halos with similar SSFR and different gas fraction may show different relaxation behaviour. In a future work, we will study such effects using hydrodynamic simulations with different baryonic prescriptions, that produces haloes with same $f_g$ but very different SSFR and vice versa.
On the other hand, for the cluster-scale haloes shown in \figref{fig:fit-func-rf-13514-ch:z0main}, the $q_0$ parameter does not vary significantly with any of the halo property that we considered.

Turning to $q_1$, while we see no clear dependence of its constituent parameters $q_{10}$ and $q_{11}$ on the hydrodynamical halo properties of low-mass haloes in \figref{fig:fit-fit-func-q-ch:z0main}, for cluster-scale haloes we do see a strong, albeit complex, dependence of $q_1$ on SSFR (see the third row of \figref{fig:fit-func-rf-13514-ch:z0main}). Like the case of the concentration significance, it will be interesting in future work to understand the physical mechanisms driving some of the stronger correlations of the halo response with properties such as SSFR and $f_g$ seen above.




\subsubsection*{Dependence in the cluster scale}
\label{sec:depend_hal_gal_props_cluster}
The effect of different halo and galaxy properties on the response of the dark matter is presented here for clusters. The relaxation parameters $q_0(r)$ and $q_1(r)$ is shown in \figref{fig:fit-func-rf-13514-ch:z0main} as our three parameter description fails for these cluster scale haloes (see section \ref{sec:dep-on-hal-gal-props-ch:z0main} for details). Notice that unlike low mass haloes, the relaxation in these clusters doesn't show any clear monotonic trend with the set of halo and galaxy properties explored in this chapter. Notice that the features in radial dependence of these relaxation parameters is shifted between these subsample of haloes. For example, the less concentrated haloes shows a peak in $q_{1}$ at a slightly larger halo-centric distance than the more concentrated haloes among $10^{14}$. We present these results here for completeness, and more investigations are needed especially regarding the mergers and substructure distribution to interpret these results.  

\begin{figure*}
    \centering
    \includegraphics[width=0.32\linewidth]{plots/fit_params_rf_M_T_13.pdf}
    \includegraphics[width=0.32\linewidth]{plots/fit_params_rf_M_T_13.5.pdf}
    \includegraphics[width=0.32\linewidth]{plots/fit_params_rf_M_T_14.pdf}
    
    \includegraphics[width=0.32\linewidth]{plots/fit_params_rf_M-cs_T_13.pdf}
    \includegraphics[width=0.32\linewidth]{plots/fit_params_rf_M-cs_T_13.5.pdf}
    \includegraphics[width=0.32\linewidth]{plots/fit_params_rf_M-cs_T_14.pdf}
    
    \includegraphics[width=0.32\linewidth]{plots/fit_params_rf_M-fs1_T_13.pdf}
    \includegraphics[width=0.32\linewidth]{plots/fit_params_rf_M-fs1_T_13.5.pdf}
    \includegraphics[width=0.32\linewidth]{plots/fit_params_rf_M-fs1_T_14.pdf}
    
    \includegraphics[width=0.32\linewidth]{plots/fit_params_rf_M-ssfr1_T_13.pdf}
    \includegraphics[width=0.32\linewidth]{plots/fit_params_rf_M-ssfr1_T_13.5.pdf}
    \includegraphics[width=0.32\linewidth]{plots/fit_params_rf_M-ssfr1_T_14.pdf}
    
    \includegraphics[width=0.32\linewidth]{plots/fit_params_rf_M-fg_T_13.pdf}
    \includegraphics[width=0.32\linewidth]{plots/fit_params_rf_M-fg_T_13.5.pdf}
    \includegraphics[width=0.32\linewidth]{plots/fit_params_rf_M-fg_T_14.pdf}
    
    \caption{Similar to upper panel of \figref{fig:rf-fit-params-ch:z0main} with cluster-scale haloes further split by other properties. In top row, only halo mass dependence is shown, whereas in the next three rows, we further show the dependence on halo concentration, stellar mass fraction, specific star formation rate and gas fraction in terms of percentiles respectively.} 
    \label{fig:fit-func-rf-13514-ch:z0main}
\end{figure*}





\section{Applications}
\label{sec:applic-ch:z0main}
In this section, we briefly discuss a few (potential) applications of our analysis.

\subsection{Baryonification schemes}

The results above show that the response of a halo's dark matter content to the galaxy and gas evolving in it depends not only on the integrated properties of the halo and galaxy (such as mass, concentration, etc.) but also on halo-centric distance, even at fixed mass ratio. This is in stark contrast to analytical approximations employed in the literature which typically use simplified  relations between the relaxation ratio and mass ratio, ignoring the radial dependence. These analytical approximations are now commonly employed in baryonification schemes to predict the total matter power spectrum for a given cosmological model using only the results of gravity-only $N$-body simulations \citep{2015JCAP...12..049S,2018MNRAS.480.3962C,2021MNRAS.503.3596A}. Our results above can directly impact such predictions by modifying the small-scale (deep 1-halo regime) behaviour of the power spectrum. 

For example, to model the effect of baryons in low- and intermediate-mass haloes ($\lesssim10^{13}\Mh$), we advocate the use of our fitting function \eqn{eq:q3-model-ch:z0main} for the relaxation relation, with parameters set to $q_0\simeq-0.05$, $q_{10}\simeq1.1$ and $q_{11}\simeq0.5$,\footnote{We have tested that the relaxed mass profile predicted by such a generic model agrees reasonably with the simulation; see \secref{sec:mass-prof-demo-ch:z0main} and \secref{sec:apndx-demo-ch:z0main} for a more accurate prediction} which gives a good description of the results of both IllustrisTNG and EAGLE haloes (see Fig.~\ref{fig:3-param-mass-only-ch:z0main}). For larger (cluster-sized) haloes, the response is still accurately described by the relation \eqn{eq:chi-linear-q0-ch:z0main}, but with more complex behaviour for the parameters $q_1(r_f)$ and $q_0(r_f)$, which presently needs to be accounted for numerically (see, e.g., Fig.~\ref{fig:fit-func-rf-13514-ch:z0main}). We discuss this further in \secref{sec:conclusion-ch:z0main}.









\subsection{Mass profiles}
\label{sec:mass-prof-demo-ch:z0main}
The primary utility of an analytical model (or fitting function) such as \eqn{eq:chi-linear-q0-ch:z0main} for the relaxation relation is to be able to predict the relaxed dark matter profile $M_f^d(r_f)$ of a halo which has responded to its baryonic content. The procedure for obtaining this profile is straightforward \citep[see, e.g., Appendix A of][]{2021MNRAS.503.4147P}: the relaxation relation is solved iteratively using the unrelaxed mass profile and the baryonic mass profile as inputs, until a converged answer for $M_f^d(r_f)$ is achieved.\footnote{In some cases, when these input profiles can be described by simplified analytical forms, a fully analytical expression for the relaxed dark matter profile can also be obtained \citep[see, e.g., Appendix A of][]{2021MNRAS.507..632P}.} In our case, the procedure to obtain the relaxed dark matter profile using \eqn{eq:chi-linear-q0-ch:z0main} works identically. The additional radial dependence of $q_1$ and $q_0$ is not an issue, since the radius $r_f$ itself is used as the control variable in solving for the enclosed mass.

\begin{figure}[htbp]
    \centering
    \includegraphics[width=0.84\textwidth]{plots/Mass-prof_with_demo_accu-1.pdf}
    \caption{\emph{(Top row:)} For the haloes in IllustrisTNG simulations, the mean radial mass profiles are shown in bins of halo mass for the baryonic component (dash-dotted curves) and dark matter component in hydrodynamic (dashed curves) and gravity-only (dotted curves) runs.  The relaxed dark matter mass profile predicted as described in \secref{sec:mass-prof-demo-ch:z0main} is shown by solid curves. The colour-coding follows Fig.~\ref{fig:mass_bin_label-ch:z0main}; for clarity we use two panels to show the averaged mass profiles for the nine mass bins. 
    \emph{(Bottom row:)} The ratio of the relaxed dark matter mass profile predicted by our model to that from the hydrodynamic simulation is shown by solid curves. For comparison, the corresponding ratio for quasi-adiabatic relaxation model with $q=0.33$ is shown by dashed curves and the ratio of dark matter mass profile between gravity-only simulation to the full hydrodynamic simulation is shown by dotted curves, representing the case of no relaxation. } 
    \label{fig:demo-fit-ch:z0main}
\end{figure}

As an example, we compare the relaxed profiles predicted by this procedure -- using the unrelaxed and baryonic mass profiles and the fits to the relaxation relation \eqn{eq:chi-linear-q0-ch:z0main} as inputs -- with the dark matter profiles actually measured in the hydrodynamical simulations for the same haloes.
For simplicity, in this analysis we ignore the dependence of the dark matter response on halo properties other than the mass;
we use $q_0$ and $q_1$ as a function of $r_f/R_{\rm{vir}}$ as shown in the \figref{fig:rf-fit-params-ch:z0main} for each halo mass bin. In the upper panel of Fig.~\ref{fig:demo-fit-ch:z0main}, we show this estimated mass profile along with the actual mass profiles found in the IllustrisTNG simulation.
For comparison, we also show the results of replacing the relaxation relation with simpler approximations from the literature (while still using the unrelaxed and baryonic mass profiles from the simulation as inputs in the iterative procedure).
Our model produces significantly better estimates of the relaxed dark matter profile, especially in the inner halo we obtain an order of magnitude better accuracy in comparison to such simple models (see lower panel of Fig.~\ref{fig:demo-fit-ch:z0main}). Below in \secref{sec:apndx-demo-ch:z0main}, we show that even the simple three parameter model gives a reasonably good prediction of the relaxed mass profile, while also being easier to incorporate into the existing procedures that use an adiabatic relaxation model. 



\subsection{Using the three parameter model}
\label{sec:apndx-demo-ch:z0main}
Here we show the relaxed mass profile predicted by our 3-parameter model \eqn{eq:q3-model-ch:z0main} for the haloes with mass, $M\leq10^{13}\Mh$. We follow a similar procedure as described above in \secref{sec:mass-prof-demo-ch:z0main}; once again we consider the response as a function of only the halo mass and ignore the dependence on other halo properties. For a given halo we obtain the values of the three parameters namely $q_0$, $q_{10}$ and $q_{11}$ by simply interpolating the fitted parameters shown in \figref{fig:3-param-mass-only-ch:z0main} as a function of mass. We find that accounting for radial dependence through a simple \eqn{eq:q3-model-ch:z0main}, gives a mass profile that is within $10\%$ of the simulation even upto $2 \%$ of virial radii (see upper left panel of \figref{fig:relxn_models_compare-ch:z0main}) and within $3 \%$ for the low mass haloes. The upper right panel of \figref{fig:relxn_models_compare-ch:z0main}, shows the corresponding profiles assuming a mass independent fixed values for the parameters $q_0\simeq-0.05$, $q_{10}\simeq1.1$ and $q_{11}\simeq0.5$. In addition to these, we also compare with mass profiles predicted by the standard adiabatic relaxation of Blumenthal et al. (1986) \citep{1986ApJ...301...27B} and few other relaxation models from Gnedin et al. (2004) \citep{2004ApJ...616...16G}, Paranjape $\&$ Sheth (2021) \citep{2021MNRAS.507..632P} and Cautun et al. (2020) \citep{2020MNRAS.494.4291C} for the IllustrisTNG haloes.

\begin{figure*}
    \centering
    \includegraphics[width=0.85\linewidth]{plots/relxn_model_comparison.pdf}
    \caption{Ratio of the relaxed dark matter mass profile predicted by various models to the dark matter profile found in the hydrodynamical simulation IllustrisTNG is shown by solid curves. Here the mass profiles are stacked across multiple haloes selected by their mass and the color coding follows \figref{fig:mass_bin_label-ch:z0main}. Results from our three parameter model \eqn{eq:q3-model-ch:z0main} is shown in the upper panel, while the left panel accounts for mass dependence in the parameters, the right panel assumes fixed values namely $q_0\simeq-0.05$, $q_{10}\simeq1.1$ and $q_{11}\simeq0.5$. In the rest of the panels, the corresponding ratio is shown for the mass profiles predicted by few existing models as mentioned in the \secref{sec:apndx-demo-ch:z0main}. The result from this work, as shown in bottom panel of \figref{fig:demo-fit-ch:z0main} is shown in dashed curves for reference.}
    \label{fig:relxn_models_compare-ch:z0main}
\end{figure*}


\subsection{Rotation curves}
Since our model can predict relaxed mass profiles using unrelaxed and baryonic mass profiles as inputs, it can also predict rotation curves of galaxies using the same inputs, along with some assumptions regarding the geometry of various mass components. 
The interpretation of observed rotation curves and related statistics such as the radial acceleration relation, using data from spatially resolved spectroscopy of nearby galaxies, forms a key aspect of discussions in the literature regarding the nature of gravity at galactic scales \citep[e.g.,][]{lms16b,lmsp17}. 

In the $\Lambda$CDM context, such studies typically model the relaxed dark matter profile using a generalised NFW profile, with or without a core, but unconnected to the baryonic mass \citep[e.g.,][]{llms20}. Previous work has suggested that the use of a parametrised model of dark matter \emph{response}, rather than the relaxed profile itself, should lead to more robust results \citep{2021MNRAS.507..632P,pscs21}. For example, it is known that the use of smooth NFW-like profiles does not produce formally good fits in cases where the observed rotation curve shows oscillatory behaviour. Rather, these oscillations in the rotation curve  correlate with similar oscillations seen in the measured baryonic mass profiles \citep[see, e.g., figs.~4 and~6 of][]{llms20}.\footnote{There could also be additional biases induced by various simplifying modelling assumptions regarding, e.g., circularity of orbits and disk thickness, which must be accounted for especially in the context of cored versus cuspy inner halo profiles \citep[see, e.g., the discussion in][]{roper+22}.} It is then reasonable to speculate that a model which smoothly parametrises the physics of the dark matter response, rather than the profile of dark matter itself, might account for such correlations naturally. More generally, such a model is more physically motivated than one which directly parametrises the dark matter profile itself. 

In a future work, we plan to confront observed rotation curves for low-mass systems with the 3-parameter relaxation model presented above. Our specific results for the values of these parameters in IllustrisTNG and EAGLE can then provide useful priors for the statistical comparison with data.












\section{Conclusion}
\label{sec:conclusion-ch:z0main}
In this chapter, we have explored in detail the response of the dark matter content of a halo to the galaxy and gas it hosts. Understanding and accurately modelling this response is important for a number of applications including baryonification schemes for small-scale power spectrum emulation, rotation curve modelling, %
constraining the nature of dark matter using inner halo mass profiles, etc. 

Using haloes and galaxies identified in the IllustrisTNG and EAGLE simulations and matched to their gravity-only counterparts, our analysis demonstrates that the simplified analytical schemes used thus far to model the dark matter response \citep[e.g.,][]{1986ApJ...301...27B,2010MNRAS.407..435A,2015JCAP...12..049S} are inadequate in describing its detailed behaviour across a variety of halo and galaxy types. Specifically, we showed that the dark matter response, or relaxation relation (see equation~\ref{eq:qAR1}), which connects the relaxation ratio $r_f/r_i$ to the mass ratio $M_i/M_f$ between unrelaxed (gravity-only) and relaxed (hydrodynamical) haloes, explicitly depends on halo-centric distance $r_f$ in the relaxed halo, in addition to being sensitive to a number of halo and galaxy properties including halo mass, halo concentration, stellar and gas mass fraction, and specific star formation rate. These effects, especially the dependence on halo-centric distance, have been typically neglected by existing quasi-adiabatic relaxation models. 

We presented a simple, physically motivated extension (equation~\ref{eq:chi-linear-q0-ch:z0main}) of the existing models which accurately captures the dark matter response over 4 orders of magnitude in halo mass ($10^{10}\lesssim M/(\Mh)\lesssim 10^{14}$) and $\sim2$ orders of magnitude in relative halo-centric distance ($0.02\lesssim r_f/R_{\rm vir}\leq1$). Apart from an explicit radial dependence of the relaxation relation (e.g., equation~\ref{eq:q3-model-ch:z0main} for low-mass haloes), a second novelty of our model is the inclusion of a parameter $q_0$ which characterises feedback-induced offsets seen in the relaxation relation measured in IllustrisTNG and EAGLE haloes in which, e.g., shells that do not show an overall change in radius ($r_f/r_i\simeq1$) nevertheless have $M_i/M_f>1$ (indicating loss of baryonic material). The existing quasi-adiabatic relaxation models do not allow for the existence of such shells, which are however captured well by our new null-offset parameter $q_0$ (see \secref{subsubsec:sim-relax-ch:z0main} for a detailed discussion).
We argued that our results could have a significant impact on the applications listed above.

Our analysis also raises some interesting questions, which we briefly discuss before concluding. We noted in \secref{sec:results-rad-dep-qadiab-ch:z0main} that, unlike low-mass haloes whose relaxation relation is well-described by \eqn{eq:q3-model-ch:z0main}, the radial dependence of the relaxation parameters $q_1$ and $q_0$ in \eqn{eq:chi-linear-q0-ch:z0main} for haloes with $M\gtrsim10^{13}\Mh$ shows non-trivial features and oscillations that are not easily captured by simple fitting functions (see Fig.~\ref{fig:fit-func-rf-13514-ch:z0main}). These features, which typically occur in the halo outskirts, are likely due to the presence of substructure or recent mergers, which would generically lead to a disturbed dynamical state of the halo. Here we have not attempted to model these features; it will be interesting in the future to systematically study the dependence of these features on substructure fraction, merger history, locations of shocks, etc.

At the other extreme, in the inner halo of low-mass systems, it is very interesting to ask whether the simple quasi-adiabatic relaxation prescription we have calibrated here can naturally lead to cored inner dark matter profiles. Previous attempts at coupling the relaxed dark matter profile to baryonic physics using simple prescriptions have focused on introducing a baryonic dependence of the parameters describing the dark matter profile itself \citep[e.g.,][]{2014MNRAS.441.2986D}. Our approach, on the other hand, parametrises the \emph{physics} of the dark matter response, and it will be interesting to see whether this leads to more robust results for cored inner profiles. For such an exercise, it will also be important to understand the dependence of our calibrated parameters on technical choices defining the sub-grid physics models used in the simulations, which can significantly impact the formation of cores \citep{bfln18}. We investigate the role of such sub-grid astrophysical models in \chapref{chap:physvar_z01}.

Finally, building a more in-depth understanding of our results will need a physical, preferably analytical, model. One possibility is to use the self-similar approximation \citep[][]{1984FillmoreGoldreich,1985ApJS...58...39B,2015LauNagaietal,2016ShiDMLamCDM} to model the combined dynamical evolution of dark matter, gas and stars in a halo. We report the results of such a study in chapter \ref{chap:self-sim-relxn}.

 
% \section*{Acknowledgments}
% We thank Nishant Singh, Nishikanta Khandai and Kandaswamy Subramanian for useful discussions in the early phases of this work.
% We thank the anonymous referee for useful comments that improved the clarity of the presentation.
% We gratefully acknowledge the use of high performance computing facilities at IUCAA (\url{http://hpc.iucaa.in}). This work made extensive use of the open source computing packages NumPy \citep{vanderwalt-numpy},\footnote{\url{http://www.numpy.org}} SciPy \citep{scipy},\footnote{\url{http://www.scipy.org}} Matplotlib \citep{hunter07_matplotlib},\footnote{\url{https://matplotlib.org/}} Pandas \citep[][]{reback2020pandas},\footnote{\url{https://pandas.pydata.org/about/}} Schwimmbad \citep{schwimmbad},\footnote{\url{https://joss.theoj.org/papers/10.21105/joss.00357}} H5py,\footnote{\url{https://www.h5py.org/}} Colossus \citep{colossus},\footnote{\url{http://www.benediktdiemer.com/code/colossus/}}  Jupyter Notebook\footnote{\url{https://jupyter.org}} and Code-OSS.\footnote{\url{https://github.com/microsoft/vscode}}

% \section*{Data availability}
% The IllustrisTNG simulations are publicly available at \url{https://www.tng-project.org/}. The EAGLE simulations are publicly available at \url{https://icc.dur.ac.uk/Eagle/}. 


% \bibliography{references}

% \appendix












% \section{Locally linear relaxation relation}
 % 

\chapter[Role of astrophysical modeling and epoch]{Role of astrophysical modeling on dark matter halo relaxation response at redshifts $z=0$ and $z=1$}
\label{chap:physvar_z01}

Advancements in computational cosmology have produced state-of-the-art hydrodynamical simulations of cosmological volumes with realistic galaxies (see, e.g., OWLS \citep{2010MNRAS.402.1536S}, Illustris \citep{2014MNRAS.445..175G}, FIRE \citep{2014MNRAS.445..581H}, EAGLE \citep{2015MNRAS.446..521S}, Horizon-AGN \citep[][]{2017MNRAS.467.4739K}, SIMBA \citep[][]{2019MNRAS.486.2827D}, IllustrisTNG \citep{2019ComAC...6....2N}). However, many sub-galactic astrophysical processes, such as star formation, are not resolved by these simulations. Instead, they rely on subgrid prescriptions with parameters calibrated to a set of empirical observations. Ambiguities in the modeling and the calibrated quantities have led to inconsistent results. Additionally, these simulations are computationally expensive to perform, which is why dark matter haloes have typically been studied using gravity-only simulations. Nevertheless, the impact of baryonic processes on the dark matter can be significant and must be accounted for before comparing against observations.

Early studies of individual haloes modeled the response of the radial distribution of dark matter to galaxy formation as adiabatic relaxation \citep[][]{osti6457593,1984MNRAS.211..753B,1986ApJ...301...27B,1987ApJ...318...15R}. However, this idealistic model rarely predicts the dark matter distribution within haloes in hydrodynamical cosmological simulations \citep[e.g.,][]{2004ApJ...616...16G,2010MNRAS.407..435A}. Moreover, the halo response was found to vary widely across haloes and in different simulations \citep[][]{2004ApJ...616...16G,2006PhRvD..74l3522G,2010MNRAS.402..776P,2010MNRAS.406..922T,2010MNRAS.405.2161D,2010MNRAS.407..435A,2011MNRAS.414..195T,2016MNRAS.461.2658D,2019A&A...622A.197A,2022MNRAS.511.3910F,2023Velmani&Paranjape}, leading to the development of various models of the response, some of which are direct extensions of adiabatic relaxation \citep[e.g.,][]{2004ApJ...616...16G,2006PhRvD..74l3522G,2010MNRAS.407..435A}. \AP{chapter heading at top left part of the page is showing `Introduction'? this may affect all chapters.}~\PV{Fixed them}

In \chapref{chap:z0_main}, we have demonstrated that introducing an additional dependence on the halo-centric distance makes the relation between $r_f/r_i$ and $M_i/M_f$ linear across a variety of haloes in both IllustrisTNG and EAGLE simulations at the present epoch ($z=0$). Based on those results, we have presented a simple prescription for computing the relaxed dark matter mass profiles. In the first part of this chapter, we investigate if such a relaxation model can also be used at an earlier redshift.

Previous works have shown that feedback from various galactic processes strongly influences halo relaxation \citep{2011MNRAS.414..195T}. These effects primarily affect the offset in the relaxation relation, quantifying the excess relaxation experienced by the dark matter \citep{2023Velmani&Paranjape}. We have argued in \chapref{chap:z0_main}, that this could be the reason for the small deviations in the relaxation offset across haloes with different star formation activities. In this chapter, we systematically the role of various astrophysical processes in the galaxy, such as feedback, in mediating the relaxation response of the halo using simulations with variations in the feedback implementation.

This chapter is organized as follows. In \secref{sec:res-itng-z01}, we explore the relaxation response at an earlier redshift in the IllustrisTNG reference simulation. This allows us to understand how the different galactic processes occurring at earlier times affect dark matter halo relaxation. Then in \secref{sec:res-physvar-CAMELS}, we systematically study the relaxation response as a function of the feedback related parameters in the IllustrisTNG model using the set of simulations from CAMELS as described in \secref{sec:sims-CAMELS}. 
Following this, we investigate the role of a variety of astrophysical processes in the EAGLE physics variation simulations described in \secref{sec:sims-EAGLE} and conclude in \secref{sec:conclusion-ch:physvar}.

\section{Early epoch in IllustrisTNG simulations}
\label{sec:res-itng-z01}

In this section, we investigate the relaxation response in the radial distribution of dark matter in the main simulations of IllustrisTNG simulations at an earlier redshift of $z = 1$ and compare it to the present redshift of $z = 0$. Haloes at both these redshifts were identified and matched between hydrodynamical and the corresponding gravity-only runs as described in \secref{sec:hals}. Additionally, we use the SubLink merger tree catalogues to trace the most massive progenitor haloes at $z=1$ of the haloes considered at $z=0$. While present epoch haloes are sampled only by their masses at the present time, we consider three different methods for sampling the early epoch haloes. This results in four distinct sets of halo samples:

\begin{enumerate}
    \item $z=1$ haloes sampled by their masses at $z=1$.
    \item $z=1$ haloes sampled by the masses of their descendants at $z=0$. Note that not all haloes with a given mass at the present time have valid progenitors with the same mass at $z=1$.
    \item $z=1$ haloes sampled by their masses at $z=1$, but the mass bins are defined by the median masses of the most massive progenitors of the $z=0$ haloes.
    \item $z=0$ haloes sampled by their masses at $z=0$.
\end{enumerate}

In each case, the mass bins are defined as described in \secref{sec:results-mass-ch:z0main} considering the mass of the gravity-only halo. The representative colors to represent this halo mass is shown in \figref{fig:mass_bin_label-z01}. Additionally, this figure indicates the peak heights ($\nu$) corresponding to these halo masses at both $z=0$ and $z=1$. These values correspond to the rarity of haloes with that mass at that redshift with rarer haloes having larger values of $\nu$. This can be used to identify mass of the halo at $z=1$ that will have same rarity as $z=0$ halo of a given mass. 

For each of these four sets of halo samples, the average relaxation relations ($r_f/r_i$ vs $M_i/M_f$) are shown in \figref{fig:fit-view-mass-indep}. The bottom right panel, reproduces the \figref{fig:fit-view-mass-indep-ch:z0main} and shown here for comparison. We find that, for a given halo mass, relaxation is usually stronger at the earlier epoch (top left panel) compared to the present time $z=0$ (bottom right panel). In both cases, the trend in relaxation with halo mass is similar, with the strongest relaxation observed in $10^{12} \Mh$ haloes. Interestingly, cluster-scale haloes with masses of $10^{14} \Mh$ at redshift $z=1$ exhibit significant relaxation, unlike clusters of similar size at the present time.

The progenitors at redshift $z=1$ of the same haloes found at $z=0$ show even stronger  relaxation, especially in Milky Way-scale and larger haloes, as shown in the top right panel of \figref{fig:fit-view-mass-indep}. The relaxation follows the simple quasi-adiabatic model \eqref{eq:chi-linear-ch:sims} with $q=0.33$ among larger cluster-scale ($10^{14} \Mh$) haloes, while group-scale haloes are consistent with the second-order polynomial relation proposed by Abadi et al. (2010) \cite{2010MNRAS.407..435A}. In the bottom left panel, the relaxation relation is shown for all haloes within a narrow mass bin around the median mass of the progenitors of the haloes selected at $z=0$. Notice that the relaxation relation shifts further lower in Milky Way-scale haloes and smaller clusters with the inclusion of those additional haloes.

\begin{figure}[htbp]
\centering
\includegraphics[width=0.49\linewidth]{plots/Mass_bin_labels_z.pdf}
\caption{Representative colors denoting each of the halo mass bins. The numbers in the figure indicate the corresponding values of peak heights $\nu$ at redshifts $z=0$ and $z=1$.}
\label{fig:mass_bin_label-z01}
\end{figure}

\begin{figure*}
\centering
\includegraphics[width=0.48\linewidth]{plots/fit_view_M_T_snap049.pdf}
\includegraphics[width=0.48\linewidth]{plots/fit_view_M_T_snap049_smpl98.pdf}
\includegraphics[width=0.48\linewidth]{plots/fit_view_M_T_snap049_smpl98_allHalsMrange.pdf}
\includegraphics[width=0.48\linewidth]{plots/fit_view_M_T_snap098.pdf}
\caption{The stacked relation between relaxation ratio and mass ratio as a function of halo mass in IllustrisTNG at $z=0$ \emph{(bottom right panel)} and $z=1$ \emph{(other panels)}. In the top right panel, relaxation is shown at $z=1$ for the progenitors of the haloes selected at $z=0$. In the second row, left panel, relaxation is shown at different mass bins at $z=1$, indicated by corresponding mass bins at $z=0$. Points represent stacks over fixed halo-centric distances, and solid lines represent stacks over fixed mass ratios. The color-coding follows Fig.~\ref{fig:mass_bin_label-z01}. The quasi-adiabatic relaxation model \eqref{eq:chi-linear-ch:sims} with $q=0.68$ and $q=0.33$ are shown by the dot-dashed and dashed purple lines, respectively, in each panel.}
\label{fig:fit-view-mass-indep}
\end{figure*}



\begin{figure}[htbp]
\centering
\includegraphics[width=0.48\linewidth,trim={0.5cm 0 0 0},clip]{plots/fit_params_rf_M_T_snap049.pdf}
\includegraphics[width=0.48\linewidth,trim={0.5cm 0 0 0},clip]{plots/fit_params_rf_M_T_snap049_smpl98.pdf}
\includegraphics[width=0.48\linewidth,trim={0.5cm 0 0 0},clip]{plots/fit_params_rf_M_T_snap049_smpl98_allHalsMrange.pdf}
\includegraphics[width=0.48\linewidth,trim={0.5cm 0 0 0},clip]{plots/fit_params_rf_M_T_snap098.pdf}
\caption{Linear quasi-adiabatic relaxation model parameters $q_1$ and $q_0$ as a function of the halo-centric distance at different halo masses in IllustrisTNG at $z=0$ \emph{(bottom right panel)} and $z=1$ \emph{(other panels)}. In the top right panel, relaxation is shown at $z=1$ for the progenitors of the haloes selected at $z=0$. In the second row, left panel, relaxation is shown at different mass bins at $z=1$, indicated by corresponding mass bins at $z=0$. The color-coding follows Fig.~\ref{fig:mass_bin_label-z01}.}
\label{fig:rf-fit-params}
\end{figure}

In \chapref{chap:z0_main}, we proposed a locally linear model of the relaxation relation as follows:
\begin{align}
\label{eq:chi-linear-q0-ch:physvar}
\frac{r_f}{r_i} - 1 &= q_1(r_f) \left[ \frac{M_i(r_i)}{M_f(r_f)} - 1 \right] + q_0(r_f),.
\end{align}
We have tested this relation with our halo samples at $z=1$ and found it to hold reasonably well. \red{For each halo sample, at each halo-centric distance $r_f$, the relationship between mass ratio and relaxation ratio across all haloes is best fitted by a linear curve with the slope and offset parameters namely $q_1(r_f)$ and $q_0(r_f)$. The radial profiles of these two relaxation parameters are shown for the four set of halo populations in \figref{fig:rf-fit-params} with the color-coding given by \figref{fig:mass_bin_label-z01} for the associated halo mass.}

\red{Notice that the universality in these relaxation profiles extends to much larger mass haloes ($\sim 10^{13.5} \Mh$) at $z=1$ (top left panel) compared to $z=0$ (bottom right panel). This is despite the fact that such massive haloes are even rarer at $z=1$ than at $z=0$, as indicated by the value of $\nu$ in \figref{fig:mass_bin_label-z01}. For all the halo samples selected at $z=0$, their traced progenitor populations at $z=1$ show nearly universal relaxation profiles, with the exception of $q_0(r_f)$ in the inner halo (shown in the top right panel). This universality is even more apparent among the halo populations selected in narrow mass bins at $z=1$ based on the median masses of the progenitor populations (see bottom left panel).}

\red{The less universal and relatively noisier profiles in direct progenitors might be due to the inclusion of low mass fast accreting haloes together with high mass slow accreting haloes or simply a systematic issue such as a few misidentified progenitors. This difference between the direct progenitor populations and the haloes selected by the median progenitor masses seems to be more noticeable in the relaxation profiles than in the radially independent relaxation relations. For example, the black curves (corresponding to $10^{14} \Mh$) are noticeably different especially in inner regions between top left and bottom right panels in \figref{fig:rf-fit-params}, however, the overall relaxation relation shown in \figref{fig:fit-view-mass-indep} is well-fitted by the linear $q=0.33$ model in both the panels. While the exact reason is not yet clear, we note that the radially-dependent relaxation might be more sensitive to this difference. \AP{not sure about this... to me the black curves in bottom left and top right look very similar, except in the innermost parts of the halo}~\PV{Edited for clarity} Another interesting thing to note is that $q_1(r_f)$ shows a small non-monotonic behavior in the outer haloes at all masses in $z=1$ (e.g., see lower left panel) that is not seen at $z=0$ (lower right panel). This could indicate some relaxation effects that have already happened by $z=1$, happens later in the outermost regions.}

The offset parameter $q_0$ shown in the lower sub-panels is relatively uniform across the halo especially among the low mass haloes. The \figref{fig:fit-fit-func-q} shows the mean of $q_0$ in all four sets of halo samples. The $q_0$ parameter is usually more negative across all $z=1$ halo populations compared to $z=0$, indicating a stronger relaxation offset. Additionally, the values are more universal with halo mass at $z=1$. \red{Recall that a negative value of $q_0$ is expected to be a result of recent feedback outflows (see \secref{sec:results-rad-dep-qadiab-ch:z0main} for a detailed discussion). These feedback outflows are produced by a combination of AGN and stellar feedbacks. In the high mass haloes, powerful AGN feedbacks at earlier epochs lead to significant suppression in the star formation activity reducing the stellar feedbacks at present epoch. The reduction in the magnitude of $q_0$ from $z=1$ to $z=0$ in the high mass haloes could be simply due to the reduction in overall feedback among those haloes.}
\AP{some physics discussion needed to explain differences seen between different samples}~\PV{Added discussion}



\begin{figure}[htbp]
\centering
\includegraphics[width=0.6\linewidth]{plots/fit_param_q0_M_T_z01.pdf}
\caption{Mean of the radially dependent quasi-adiabatic relaxation offset, $q_{0}$ as a function halo mass in the four sets of halo samples indicated by color.}
\label{fig:fit-fit-func-q}
\end{figure}










\section{Varying feedback strengths using CAMELS simulations}
\label{sec:res-physvar-CAMELS}
In this section, we present the role of various feedback parameters prescribed in the IllustrisTNG simulations using the set of CAMELS simulations performed with the IllustrisTNG model. This includes a set of 41 hydrodynamical simulations, with one replicating the reference TNG model in a smaller cosmological volume, and 10 simulations each by varying 4 different feedback parameters as described in \secref{sec:sims-CAMELS}. To recall, this includes two supernovae feedback parameters ($A_{\mathrm{SN1}}$ and $A_{\mathrm{SN2}}$) and another two AGN feedback parameters ($A_{\mathrm{AGN1}}$ and $A_{\mathrm{AGN2}}$).

Due to the limited resolution and the smaller volumes of the CAMELS simulation, among all the mass bins shown in \figref{fig:mass_bin_label-z01}, we consider only $10^{11} \Mh$, $10^{11.5}$, and $10^{12} \Mh$ at both redshifts $z=0$ and $z=1$. Even in these mass bins, we consider only the outer well-resolved regions of the haloes. Also, since the cosmological volume is smaller than even the smallest TNG50 simulation, we have only a smaller sample of haloes at each of these halo mass bins. We find this sample size insufficient to estimate the radially-dependent relaxation parameters at each of these mass bins. We consider the following two approaches to alleviate this issue.

\subsection*{Intercepts in the Relaxation Relation}

The intercepts of the relaxation relation, given by the relation between $M_i/M_f-1$ and $r_f/r_i-1$, already provide interesting information about the relaxation. For example, the y-intercept denotes the offset in the relaxation ratio $r_f/r_i$ from unity for the shells having a mass ratio of unity $M_i/M_f=1$. Similarly, the x-intercept denotes the offset in the mass ratio $M_i/M_f$ from unity for the shells having a relaxation ratio $r_f/r_i = 1$.

In a given sample of haloes, we denote the average x and y intercepts as $q_x$ and $q_y$ respectively. For example, if we consider the Milky Way scale haloes at redshift $z=0$, the relaxation relation shown by the green curves in the lower right panel of \figref{fig:fit-view-mass-indep} indicates that $q_x$ will be positive and $q_y$ will be negative. In general, we expect that feedback effects would lead to larger $q_x$ and more negative $q_y$.

Due to the limited resolution of the CAMELS simulation, the outer well-resolved regions in most haloes didn't have a mass ratio less than unity. This makes the y-intercept of the relaxation relation available only in a smaller number of haloes. This makes our estimation of $q_y$ very noisy, and hence we only interpret the parameter $q_x$ in these simulations. This parameter $q_x$ is presented in \figref{fig:camels-qx0} as a function of the astrophysical parameters in the CAMELS TNG set of simulations at both redshifts $z=0$ and $z=1$.

\subsection*{Wider mass bin}

We found that the radially-dependent relaxation parameters are usually more uniform across a wider range of halo masses. We leverage this to consider haloes in a wider mass bin from $10^{11} \Mh$ to $10^{12} \Mh$, which gives a sufficient number of haloes to obtain the radially-dependent relaxation model parameters. However, still, the radial range is not sufficient to accurately model the radial dependence of the slope parameter $q_1(r_f)$, so we only investigate the offset parameter $q_0$, defined as the mean of the $q_0(r_f)$. This parameter $q_0$ is usually negative, and it is expected to be more negative when the offset produced by overall feedback effects is stronger. This model-dependent offset parameter is presented in \figref{fig:camels-q0q1} at both redshifts $z=0$ and $z=1$.

\begin{figure}[htbp]
\centering
% \includegraphics[width=0.325\linewidth]{plots/CAMELS_I_qx0_sn18_11.pdf}
% \includegraphics[width=0.325\linewidth]{plots/CAMELS_I_qx0_sn18_11.5.pdf}
% \includegraphics[width=0.325\linewidth]{plots/CAMELS_I_qx0_sn18_12.pdf}
% \includegraphics[width=0.325\linewidth]{plots/CAMELS_I_qx0_sn33_11.pdf}
% \includegraphics[width=0.325\linewidth]{plots/CAMELS_I_qx0_sn33_11.5.pdf}
% \includegraphics[width=0.325\linewidth]{plots/CAMELS_I_qx0_sn33_12.pdf}
\includegraphics[width=\linewidth]{plots/CAMELS_I_qx0.pdf}
\caption{Relaxation offset parameter $q_x$ \AP{axis labels say $q_{0}^x$ while text uses $q_x$}\PV{updated plot} as a function of the baryonic astrophysical feedback parameters in haloes found in CAMELS-TNG at three different halo masses. Top: $z=1$, Bottom: $z=0$}
\label{fig:camels-qx0}
\end{figure}

\begin{figure}[htbp]
\centering
\includegraphics[width=0.49\linewidth]{plots/CAMELS_I_q0_sn18.pdf}
% \includegraphics[width=0.49\linewidth]{plots/CAMELS_I_q1_sn18.pdf}
\includegraphics[width=0.49\linewidth]{plots/CAMELS_I_q0_sn33.pdf}
% \includegraphics[width=0.49\linewidth]{plots/CAMELS_I_q1_sn33.pdf}
\caption{Relaxation offset parameter $q_0$ as a function of the baryonic feedback parameters in CAMELS-TNG. Left: $z=1$, Right: $z=0$.}
\label{fig:camels-q0q1}
\end{figure}

\subsection*{Discussion}

We find that among all haloes investigated, the feedback strength parameters $A_{\mathrm{SN1}}$ and $A_{\mathrm{AGN1}}$ have a strong influence on the relaxation, with $q_x$ typically increasing monotonically when increasing these parameters. On the other hand, 
%whereas 
the wind speed parameters $A_{\mathrm{SN2}}$ and $A_{\mathrm{AGN2}}$ have negligible effect on the relaxation characterized by both $q_x$ and $q_0$.


Recall that when varying only a wind speed parameter, it affects the burstiness of the feedback outflows while keeping the overall flux constant. Suppose the deviations in the relaxation relation from the idealized adiabatic model, quantified by non-zero offset parameters, are caused by the transfer of angular momentum between the dark matter particles and the baryonic particles. In that case, one may expect that the nature of the baryonic feedback quantified by the wind speed parameters will have a significant influence on the value of $q_x$. However, our results suggest otherwise.

% These results suggest that the overall feedback outflows controlled by the 
\red{Recall that in \secref{subsubsec:sim-relax-ch:z0main}, we have argued that the relaxation offset is a reflection of the fact that the dark matter shells have not yet expanded in response to the recent feedback gas outflows. These results suggest that the time taken for the dark matter shells to expand are sufficiently long enough, that only the overall gas outflow is relevant irrespective of the speed and burstiness. We investigate these timescales in \chapref{chap:dynam-relxn}.} 
\AP{explain what this means. why is time delay relevant?} \AP{briefly describe how you infer this consistency}\PV{Edited for clarity}

We find that the AGN feedback strength generally has a stronger influence on the relaxation among the high-mass haloes, whereas the supernova feedback strength has a stronger influence in lower-mass haloes. This is consistent with our expectations that AGN feedbacks dominate in more massive haloes. However, at all masses, AGN feedback starts dominating the value of $q_x$ \red{when their strengths are set to be higher than the reference model of TNG} \AP{on what basis are these unrealistic? do we even care whether they are unrealistic or not? why mention this word?}; this is indicated by the green curves in \figref{fig:camels-qx0}. \red{This suggests that the relative importance of AGN and supernovae feedback on the dark matter relaxation depends strongly on their implementation.}

\AP{this para seems disconnected from previous one.}\PV{Added discussion and edited for clarity}
\AP{which part of which plot shows me the effect on these slightly inner regions, and why?}
\AP{are you talking about $q_0$ or $q_x$?}
\red{While $q_x$ which characterizes the relaxation offset predominantly in the outer halo,} is always larger with the stronger implementation of AGN feedback at both \( z=0 \) and \( z=1 \), \red{the value of $q_0$, which is averaged over both inner and outer halo, shows a different trend.}  Notice, in \figref{fig:camels-q0q1}, that the stronger AGN feedback implementation leads to a weaker relaxation offset at \( z=0 \) and a stronger offset at \( z=1 \).  We interpret this as a consequence of the overall reduction in total feedback at \( z=0 \) due to the suppression of star formation caused by higher AGN feedbacks in the past. This led to reduction in the relaxation offset in the inner haloes, \red{the outer halo has not yet responded by $z=0$}. These results highlight the significance of feedback mechanisms in building a physical understanding of dark matter halo relaxation.

% We find that at $z=1$, the magnitude of $q_0$ is larger with stronger AGN feedback indicated by larger $A_{\rm{AGN1}}$. However, this trend is absent at $z=0$. This could be due to a strong suppression in star formation rate at the present epoch due to the stronger AGN feedback in the past. \AP{did you just forget to delete this para? it seems like a shorter and less informative version of the previous para. what is the point of this para?}









\section{Role of astrophysical modeling in EAGLE simulations}
\label{sec:res-physvar-eagle}

In this section, we present insights from small boxes of EAGLE simulations on the role of different feedback mechanisms on the relaxation response. These physics variation simulations from the EAGLE suite are described in \secref{sec:sims-EAGLE}. Again, due to the significantly smaller box size and resolution, we only consider haloes in three different mass bins centered at $10^{10.5} \Mh$, $10^{11} \Mh$, and $10^{11.5} \Mh$ as discussed in \secref{sec:res-physvar-CAMELS}. In these narrow mass bins, the mean relaxation relation obtained by stacking independent of the halo-centric distance is shown in \figref{fig:EAGLE-rad-indep} for these physics variation simulations. We find that the deviation in the relaxation across different astrophysical modeling is usually much smaller than the differences from one halo mass to the other. However, notice that the gas equation of state has a strong influence on the relaxation relation, especially among low-mass haloes. In particular, stiffer equations of state lead to a very large $r_f/r_i$ indicating a stronger expansion of the dark matter shells in response to galaxy formation.

In \figref{fig:EAGLE-rad-dep}, we present the radially dependent relaxation parameters for a larger sample of haloes in a wider range of halo masses from $10^{10.5} \Mh$ to $10^{11.5} \Mh$. These results also reflect the strong influence of the equation of state of gas on the relaxation response of dark matter haloes. Additionally, these results highlight the effect of supernovae feedback modeling. In particular, the $q_0(r_f)$ is more negative, indicating a stronger offset in the haloes found in the simulation with stronger supernova feedback implementation (brown curve vs. rose curve).    

\begin{figure}[htbp]
\centering
\includegraphics[width=0.32\linewidth]{plots/eagle_physvar_rad_indep_relxn_reln_MiMf_10.5.pdf}
\includegraphics[width=0.32\linewidth]{plots/eagle_physvar_rad_indep_relxn_reln_MiMf_11.pdf}
\includegraphics[width=0.32\linewidth]{plots/eagle_physvar_rad_indep_relxn_reln_MiMf_11.5.pdf}
\caption{Relaxation relation in the physics variation EAGLE simulations for haloes in the mass bins from $10^{10.5} \Mh$ to $10^{11.5} \Mh$. Here colors represent the specific simulation with a variation in the baryonic physics prescription.}
\label{fig:EAGLE-rad-indep}
\end{figure}

\begin{figure}[htbp]
\centering
\includegraphics[width=0.7\linewidth]{plots/fit_params_rf_M_E_physvar_fatmass_uniradb.pdf}
\caption{Radially-dependent relaxation parameters for low-mass haloes from $10^{10.5} \Mh$ to $10^{11.5} \Mh$ as a function of the halo-centric distance in the physics variation EAGLE simulations. Here colors represent the specific simulation with a variation in the baryonic physics prescription.}
\label{fig:EAGLE-rad-dep}
\end{figure}


\section{Conclusion}
\label{sec:conclusion-ch:physvar}

In this chapter, we investigated the influence of astrophysical modeling on the relaxation response of dark matter haloes at different epochs, specifically focusing on \( z=0 \) and \( z=1 \). The analysis is divided into three main parts, each shedding light on the role of various astrophysical processes in shaping the dark matter content of haloes.

% \subsubsection*{1. Early Epoch in IllustrisTNG Simulations}
We began by examining the relaxation response at an earlier redshift (\( z=1 \)) in the IllustrisTNG simulations using three distinct sets of halo samples, which highlight the variations in relaxation across different halo masses. Our study reveals that dark matter relaxation tends to be stronger (smaller \( r_f/r_i \)) at the earlier epoch compared to the present among haloes of the same mass. This is even more prominent among the progenitors of present epoch haloes. Notably, we observe that cluster-scale haloes at \( z=1 \) show significant relaxation (\( r_f/r_i < 1 \)) that is also a function of the change in the enclosed mass (\( M_i/M_f \)). This is in contrast to similar haloes at the present epoch, where \( r_f/r_i \) stayed close to unity on average irrespective of the value of \( M_i/M_f \).

We also find that the locally linear quasi-adiabatic relaxation model is a good description of the relaxation relation at this earlier epoch, demonstrating its robustness in capturing the dark matter response across redshifts. Moreover, the parameters of the radially dependent relaxation are found to be more universal across a much wider range of masses at \( z=1 \). For example, the progenitors of even the most massive clusters are well characterized by the simple three-parameter model of relaxation that was developed with a focus on galactic-scale haloes at \( z=0 \). \red{This suggests that the deviation from the three-parameter model is a result of late-time events such as mergers in the cluster scale haloes.}\AP{speculate on why this is the case.}

% \subsubsection*{2. Variation in Astrophysical Feedback Using CAMELS Simulations}
Next, we explored variations in astrophysical feedback strengths within the IllustrisTNG model using simulations from the CAMELS project, which varies four different feedback parameters: two for stellar feedback and two for AGN feedback. Our analysis shows that the parameters controlling the energy flux of the feedback have a significant impact on the relaxation of dark matter at different epochs. In contrast, the parameters governing the speed and burstiness of feedback have negligible effects on the halo relaxation response. \red{This further strengthens our argument that the relaxation offset is caused by the dark matter shells that have not yet responded to the recent feedback outflows.} \AP{mention this is consistent with discussion of meaning of $q_0$}

We find that variations in stellar feedback strengths have a larger impact among dwarf galaxy-scale haloes, while variations in AGN feedback parameters exert a stronger influence on Milky Way-scale haloes. Notably, the relaxation offset in the outer well-resolved regions is stronger at the present epoch than at \( z=1 \), contrasting with results from the inner regions explored in the IllustrisTNG simulations in the first part of this chapter.

The stronger implementation of AGN feedback tends to result in greater relaxation at both \( z=0 \) and \( z=1 \) in the outer regions of the haloes. However, in the slightly inner regions, stronger AGN feedback implementation leads to a weaker relaxation offset at \( z=0 \) and a stronger offset at \( z=1 \). We interpret this as a consequence of the overall reduction in total feedback at \( z=0 \) due to the suppression of star formation caused by higher AGN feedbacks in the past. These results highlight the significance of feedback mechanisms in building a physical understanding of dark matter halo relaxation.

% \subsubsection*{3. Role of Astrophysical Models in the EAGLE Simulations}
Finally, we assessed the impact of different astrophysical models in the EAGLE simulations. Supernova feedback strengths show a similar trend to that observed in the CAMELS simulations. Additionally, we find that the gas equation of state has the strongest effect on the relaxation response of dark matter, particularly among haloes hosting dwarf galaxies.

Overall, this chapter underscores the intricate relationship between baryonic processes and dark matter halo relaxation, illustrating the variations that arise due to different astrophysical models and redshifts. % 

\chapter{Dynamics of the response of dark matter halo to galaxy evolution in IllustrisTNG}
\label{chap:dynam-relxn}

We present the dynamical evolution of the dark matter's relaxation response to galaxies and their connection to the astrophysical properties as simulated in the IllustrisTNG suite of cosmological hydrodynamical simulations.
% This study investigates the dynamical evolution of dark matter's response to galaxies within populations of haloes simulated using the IllustrisTNG cosmological volumes. By comparing haloes from hydrodynamical and gravity-only simulations, we examined the correlation between relaxation quantities and various halo/galaxy properties. 
Our results show that the radially-dependent linear relaxation relation model from our previous work is applicable at least from redshift \(z=5\). We focus on the offset parameter \(q_0\), which characterizes the relaxation of dark matter shells without changing the enclosed mass.
%
We perform multiple time-series analyses to determine the possible causal connections between the relaxation mechanism and astrophysical processes such as star formation and associated feedback processes, as well as feedback due to active galactic nuclei.
%
We show that star formation activity significantly influences the halo relaxation response throughout its evolutionary history, with essentially immediate effects in the inner haloes and delayed effects of 2 to 3 Gyr in the outer regions. Metal content shows a weaker connection to relaxation than star formation rates, but the accumulated wind from feedback processes exhibits a stronger correlation.
%
These findings enhance our understanding of halo relaxation mechanisms. 
%and improve models of halo profiles in baryonification procedures and semi-analytic galaxy formation models. 
Our estimates of the time-scales relevant for dark matter relaxation can potentially improve the description of halo profiles in existing baryonification schemes and semi-analytical galaxy formation models. Our results also show how
the relaxation response of dark haloes can probe the evolutionary history of the galaxies they host.

\section{Introduction}
\label{sec:intro}
% \begin{itemize}
%     \item Introduction to dark matter haloes, galaxies and their interplay
%     \item Intro to hydrodynamical cosmo simulations
%     \item Literature review on studies of relaxation
%     \item Brief outline of Velmani \& Paranjape (2023) \cite{2023Velmani&Paranjape} in the context of relaxation literature and build the motivation for this paper.
%     \item In this work, we study the redshift evolution of the relaxation of dark matter haloes to galaxy formation. 
%     \item Relation between star formation rate feedback and relaxation literature.
% \end{itemize}





In the standard paradigm of cosmology, galaxies form within the gravitationally collapsed structures called haloes, primarily made of dark matter that doesn't interact with baryonic galaxies except through gravity \citep[][]{wr78}. Formation and the evolution of these haloes and the galaxies they host are crucial in both cosmological and astrophysical studies. %These dark haloes are usually studied using cosmological N-body simulations, where they are formed by gravitational collapse over initial overdensity peaks. 
In the $\Lambda$CDM paradigm, dark matter haloes form through the gravitational collapse of initial density fluctuations \citep{1974ApJ...187..425P,2002PhR...372....1C}. Properties of these haloes, such as their triaxial shapes \citep{1988ApJ...327..507F} and universal mass profiles (NFW profiles; \citep{1996ApJ...462..563N,1997ApJ...490..493N}), have been well-characterized in gravity-only simulations. However, the presence of baryonic matter introduces additional complexities. The gravitational coupling between baryons and dark matter can significantly alter the spatial distribution and evolution of the latter, necessitating a detailed study of this interaction. 

High-resolution cosmological hydrodynamical simulations, which incorporate detailed feedback processes from supernovae and active galactic nuclei (AGN), provide a more nuanced understanding of halo dynamics (e.g., OWLS \citep{2010MNRAS.402.1536S}; Illustris \citealp{2014MNRAS.445..175G}; FIRE \citep{2014MNRAS.445..581H}; EAGLE \citep{2015MNRAS.446..521S}). These simulations reveal that feedback mechanisms can significantly alter the inner density profiles of dark matter haloes, potentially resolving the discrepancy between observed dark matter cores and the cuspy profiles predicted by gravity-only simulations \citep{2014Natur.506..171P}. However, the degree of transformation depends on factors such as the time-scale and frequency of feedback events \citep{2012MNRAS.421.3464P,2014ApJ...793...46O}.

The response of dark matter haloes to galaxy formation includes two main aspects: contraction or expansion towards the centre and changes in their triaxial shape \citep[][]{2004ApJ...616...16G,2006PhRvD..74l3522G,2010MNRAS.402..776P,2010MNRAS.406..922T,2010MNRAS.405.2161D,2010MNRAS.407..435A,2011MNRAS.414..195T,2016MNRAS.461.2658D,2019A&A...622A.197A,2022MNRAS.511.3910F,2023Velmani&Paranjape}. 
%\AP{briefly mention earlier work on the subject by other authors and include citations.}
In this study, we focus on the former aspect, examining spherically averaged mass profiles.
In a previous work \cite{2023Velmani&Paranjape}, we examined this impact of baryonic processes on dark matter haloes, focusing on the radial halo profiles in IllustrisTNG and EAGLE simulations across a wide range of halo masses at redshift $z \sim 0$. We found that simple adiabatic contraction models, which assume spherical symmetry and no shell crossing, often fall short of accurately describing the complex response of dark matter to baryonic effects observed in high-resolution hydrodynamical simulations.

However, by empirical modelling within the context of quasi-adiabatic relaxation, simple relations were found to describe the nature of the halo relaxation response. 
Building on our previous study, the present work aims to systematically investigate the dynamical evolution of dark matter halo relaxation and its connection to the evolution of halo and galaxy properties. In this paper, we use data from the IllustrisTNG simulation project towards this goal. By comparing the properties of haloes in the baryonic simulations of this suite with their counterparts in the corresponding gravity-only simulations, we characterize the relaxation behaviour of a variety of haloes across a wide range of masses over a time from redshift $z \sim 5$ to $z \sim 1$.

The structure of this paper is as follows.
% In \secref{sec:methods}, we describe the simulations and methodologies employed in our study. In \secref{sec:results}, we present our findings, highlighting the impact of baryonic processes on the dynamical evolution of dark matter haloes. We conclude with a summary of our key findings in \secref{sec:conclusion}. Throughout this paper, $\ln$ and $\log$ denote the natural and base-10 logarithms, respectively. 
We start with a brief description of the numerical simulations and the construction of a catalogue of matched haloes along with their evolutionary tracks in \secref{sec:simhals}. Then, in \secref{sec:methods-relchar}, we characterize the relaxation response in the context of a quasi-adiabatic framework and study their evolution in halo populations. This is followed in \secref{sec:methods-stat} by a statistical exploration of the connection between halo properties such as star formation rate and metallicity and their role in mediating the halo relaxation response. This analysis especially focuses on the rank correlations between these quantities not only across haloes in the populations but also over time. %\AP{abrupt end of para}
Finally, we conclude with a summary of our key findings and their applications in \secref{sec:conclusion}. Throughout this paper, $\ln$ and $\log$ denote the natural and base-10 logarithms, respectively.






% \newpage
% \section{Simulations and Techniques}
% \label{sec:methods}
% In this section, we start with a brief description of the numerical simulations and the construction of a catalogue of matched haloes along with their evolutionary tracks in \secref{sec:simhals}. 
% Along with an outline of the quasi-adiabatic relaxation modelling, we define our parameters that quantify some aspects of the relaxation of haloes in \secref{sec:methods-relchar}. Then, in \secref{sec:methods-stat}, we describe our statistical methods in analysing the connection between the evolution of relaxation along with other halo and galaxy properties.


\section{Simulations and haloes}
\label{sec:simhals}
In this work, we use all three different cosmological volumes of the publicly available IllustrisTNG simulations at their highest resolution \citep{2019ComAC...6....2N}; details of these simulations are outlined in \secref{sec:methods-itng}. Then, we describe in \secref{sec:methods-halopairsel} the identification and selection of haloes across a wide range of masses along with their corresponding matched halo simulated without any baryonic astrophysics. We need evolving haloes to study the dynamical relaxation; these are constructed using the merger trees consisting of the progenitor haloes matched across time as described in \secref{sec:methods-tracehals}. Finally, in \secref{sec:hal-gal-props}, % \AP{missing ref}
we explore some key properties of the halo (and its central galaxy) and their overall evolution in the population. 
% Their role in mediating the halo relaxation response is presented in \secref{sec:methods-stat}. 
% \AP{why isn't this para in the last para of the introduction? you are anticipating all the way to section 4!}

\subsection{IllustrisTNG simulations}
\label{sec:methods-itng}
IllustrisTNG provides hydrodynamical simulations of three different cosmological volumes namely TNG50, TNG100, and TNG300, with periodic box sizes of $35 \Mpch$, $75 \Mpch$, and $200 \Mpch$, respectively, consistent with the cosmology from Planck collaboration \citep[][]{2016A&A...594A..13P}. In each box, we utilise the highest resolution runs with dark matter mass resolution of $5.4 \times 10^5 \Msun$, $8.8 \times 10^{6} \Msun$ and $7.0 \times 10^7 \Msun$ respectively from the smallest TNG50 to largest TNG300; along with their corresponding gravity-only simulation of same volumes. Initial conditions for these cosmological boxes were constructed using the \textsc{N-GenIC} code \citep[][]{2015ascl.soft02003S} with the Zel'dovich approximation \citep[][]{1970A&A.....5...84Z} at $z = 127$. All these simulations employed the \textsc{arepo} code \citep[][]{2020ApJS..248...32W} utilizing a moving mesh approach defined by Voronoi tessellation \citep[][]{2010MNRAS.401..791S} for the hydrodynamics. These simulations incorporate a state-of-the-art subgrid prescription for the major baryonic processes such as cooling, star formation, stellar, and AGN feedback along with cosmic magnetic field \citep[][]{2017MNRAS.465.3291W,2018MNRAS.473.4077P}. In this work, we utilise data from redshift $z=0$ all the way to redshift $z=5$ for both hydrodynamical and corresponding gravity-only runs.

\subsection{Halo selection}
\label{sec:methods-halopairsel}
Haloes in these simulations were identified using a friend-of-friends (FoF) algorithm \citep[see][for specifics]{2016A&C....15...72M,2019ComAC...6....2N}, and the gravitationally bound substructures within these FoF group haloes were found using \textsc{subfind} code \citep{2001MNRAS.328..726S} and identified as subhaloes. While there can be more than one subhalo within a given FoF halo, only the central subhalo encloses the halo centre, which is defined by the co-moving position of the minimum gravitational potential. The radius of the sphere around this centre enclosing a mean matter density that is 200 times the cosmological critical density is defined as the `virial' radius $R_{\rm vir}\equiv R_{\rm 200c}$ of a given FoF group halo; while the total mass enclosed within this radius quantifies the mass of the halo $M\equiv M_{200c}$.

Following our previous work \citep{2023Velmani&Paranjape}, we match the haloes from the full hydrodynamic simulations with the haloes in the corresponding gravity-only runs performed with the same seed for initial conditions. This gives a set of matched halo pairs showing the strongest overlap in their proto-haloes among the nearby haloes of similar sizes found by the KD-tree algorithm. From this catalogue, we select populations of matched halo pairs by the logarithmic mass of the gravity-only halo $(\log(M/\Mh))$ in bins centred at $11.5, 12, 12.5, 13, 13.5, 14$ with a bin width of 0.3 at redshift $z=0.01$. While the small volume TNG50 offers well-resolved low-mass haloes $10^{11.5} \Mh$ and $10^{12} \Mh$, the TNG100 cosmological box gives $10^{12.5} \Mh$ and $10^{13} \Mh$ haloes and the largest volume TNG300 provides an adequate number of cluster-scale haloes of masses $10^{13.5} \Mh$ and $10^{14} \Mh$. 






% \begin{figure}
%     \centering
%     \includegraphics[width=0.49\linewidth,trim={5.2cm 0 0 0},clip]{plots/Mass_bin_labels.pdf}
%     \caption{Representative colours we use to denote each of the halo mass bins.}
%     \label{fig:mass_bin_label}
% \end{figure}
   

\subsection{Tracing evolutionary history}
\label{sec:methods-tracehals}
In cosmological simulations, it is particles that are evolved in cosmological volumes, and hence haloes need not be tracked explicitly. Rather, the haloes found at different snapshots in time are matched to construct the evolutionary tracks of the haloes. In this work, we identify the central subhaloes pairs corresponding to each pair of FoF group halos at $z\sim 0$ in our catalogue. These central subhaloes are traced back along the most massive progenitor branch using merger trees constructed by the \textsc{SubLink} code \citep{2015RodriguezGeneletalSubLink}. Then, the corresponding host FoF group haloes are considered as the progenitors of the matched haloes in our catalogue at $z \sim 0$. This gives us a catalogue of evolving pairs of haloes from redshift $z \sim 5$ to $z \sim 0$ corresponding to a cosmological look-back time of $12.63$ Gyr.

From this catalogue, we remove all those haloes that had a major merger during this period with a dark matter mass ratio above four. Further, we also remove those pairs of evolving haloes that are either separated by more than their virial radius or have sizes differing by more than $25\% $ at any given redshift. This leaves us with a sample of evolving matched haloes for around $50\%$ to $80\%$ of the populations of all haloes in our $z \sim 0$ catalogue. For these halo pairs, we study the relaxation by comparing the spherically averaged radial distribution of matter within their virial radii; this is described in  \secref{sec:methods-relchar}. Before that, we explore some of the key properties of these haloes in \secref{sec:hal-gal-props} that are relevant in this work.
% \PV{Mean evolution of some halo and galaxy properties plot}

\subsection{Halo and galaxy properties}
\label{sec:hal-gal-props}
When it comes to the role of galactic astrophysical processes in mediating the halo relaxation, feedback processes are known to play a significant role \cite{2011MNRAS.414..195T,2023Velmani&Paranjape}. To understand this, we look into the properties of the halo that quantify various galactic processes involved in producing the feedback. 
% In this regard, we first focus on the star formation rate since newly formed massive stars are among the major sources of the feedback. These feedback processes inject metals into the surrounding gas and contribute to galactic winds. In addition to this, accretion onto AGN also produce powerful winds.
% \AP{do we want to comment on AGN feedback? you have already mentioned that we will look at group/cluster-sized halos.}\PV{Adding details below}
% 
% 
In this work, we primarily focus on the  following quantities: 
\begin{itemize}
    \item \textbf{SFR}: The net star formation rate of the central subhalo. The feedback associated with stellar winds and supernovae of massive stars follows star formation activity.
    \item $\mathbf{Z^{\star}_{\rm{O}}}$: The mean metallicity of oxygen in the star-forming regions within the central subhalo. Oxygen is a key tracer of chemical enrichment and feedback processes.
    \item $\mathbf{M_{\rm{Wind}}}$: The total mass of wind within the halo group. In addition to the stellar feedback, accretion onto AGN can also produce powerful winds. These winds eventually lose their mass and energy as they travel through the gas.
    \item $\mathbf{M_{\rm{O}}}$: The total oxygen content within the halo group in the gas component; This quantity probes the stellar feedback over a longer time, even after the suppression of $\mathbf{Z^{\star}_{\rm{O}}}$ due to inflowing metal-poor gas.
    % Another tracer of the stellar feedback. Unlike $\mathbf{Z^{\star}_{\rm{O}}}$, this quantity is less sensitive  }
    % \item Total Mass of Wind ($M_{\rm{Wind}}$): Galactic winds carry away material (including metals) from the galaxy. Understanding the mass of these winds helps us grasp the impact of feedback on the overall system.
\end{itemize}
All these quantities are defined for the hydrodynamic halo of each matched group halo pair in our catalogue. The evolution history of these quantities in the populations of haloes having the same final mass is shown in \figref{fig:evolution-hal-gal-props}. While low mass haloes at $z \sim 0$ are currently at their peak star formation, it is significantly suppressed in the high mass haloes.
% \st{A detailed study of these quantities and their interplay with halo relaxation response is presented in the section} \ref{sec:methods-stat}. \AP{you keep anticipating things 2 sections in advance, it is quite distracting. pls fix this everywhere.}

\begin{figure}[htbp]
\centering
\includegraphics[width=\linewidth]{plots/dynam_relxn/hal_gal_props_evolve.pdf}
\caption{Evolution history of the average star formation rate (SFR, top left panel), mean oxygen metallicity in star-forming regions ($Z^{\star}_{\rm{O}}$, top right panel), the mass of oxygen in gas ($M_{\rm{O}}$, bottom left panel) and mass of wind ($M_{\rm{Wind}}$, bottom right panel) in halo populations selected by their total mass at redshift $z\sim 0$ in the gravity-only run.}
\label{fig:evolution-hal-gal-props}
\end{figure}

\section{Characterizing Relaxation Response}
\label{sec:methods-relchar}
% \subsubsection{Mass Profiles}
The overall relaxation response of dark matter in a halo, such as an expansion or contraction in response to galaxy formation, is usually analysed through changes in the spherically averaged mass profiles. In our catalogue of matched evolving pairs of haloes, while the hydrodynamic ones with galaxies include this relaxation response, the gravity-only counterparts provide the corresponding unrelaxed dark haloes. These radial dark matter mass profiles are obtained as the cumulative sum of the mass contributed by all dark matter particles within concentric spherical shells. In the case of the gravity-only halo, we consider the cosmic dark matter fraction of the mass in each particle of the gravity-only halo to be contributing to the dark matter. Characterizing the relaxation from these profiles is done using the quasi-adiabatic relaxation framework as described below.

\subsection{Quasi-adiabatic relaxation framework}
\label{sec:methods-adiab}
All the relaxation response on cold dark matter happens entirely through gravitational encounters with the baryons. In particular, it is an integrated effect of the flow of baryonic mass in the past due to galactic processes such as inflows and feedback. The quasi-adiabatic relaxation is a physically motivated framework to model the change in the spherically averaged dark matter distribution at a given time as a function of just the spherically averaged baryonic distribution at that same time. This baryonic profile includes all the mass other than the dark matter, including the mass in gas cells assigned by a Gaussian kernel to concentric spherical shells. 

Early works made further assumptions that the halo is spherical and that the dark particles maintain their radial ordering while responding adiabatically to the flow of the baryonic particles across them \citep[][]{1986ApJ...301...27B}. Suppose a dark matter particle at radius $r_i$ in the unrelaxed
% \AP{unrelaxed?}
halo ends up in radius $r_f$ in the relaxed halo, then the enclosed dark matter within spheres of those radii will be equal.
% For a spherical shell of radius $r_i$ enclosing a dark matter mass $M_i^d(r_i)$ 
% This assumes that the 
% Following the procedure of our previous work \citep{2023Velmani&Paranjape}
 % Additionally, we incorporate baryonic mass profiles in modelling the dark matter response. 
% Throughout this study, we define concentric shells by their radii, where the mass enclosed is the mass within the sphere bounded by each shell. 
% We have verified that our results are stable to variations in the choice of kernel size.
% The impact of galaxy formation on the dark halo is expected to be primarily an adiabatic relaxation of dark matter particle orbits in response to baryon condensation. 
% We discuss this simplified model and study more complex effects, such as the impact of baryonic feedback processes, in the following sections.
% Assuming that the dark matter halo is spherical and doesn't undergo shell crossing while baryons condense towards the centre, the adiabatic relaxation of any given dark matter shell is determined by the change in baryonic mass within that shell. 
% Consider a shell enclosing a \emph{dark matter} mass $M_i^d(r_i)$ in radius $r_i$ in the unrelaxed halo. After relaxation, the radius of the shell changes to $r_f$. By definition, the dark matter mass $M_f^d(r_f)$ enclosed in $r_f$ in the relaxed halo is simply
\be 
M_f^d(r_f) = M_i^d(r_i)\,.
\label{eq:DMmass}
\ee
However, the \emph{total} mass enclosed within those spheres is not necessarily equal $M_i(r_i) \neq M_f(r_f)$ due to the flow of baryonic mass. When the dark matter particles conserve angular momentum, maintaining nearly circular orbits, this change in total mass enclosed must be consistent with the amount of relaxation,
% then the amount of relaxation of the shell is completely determined by the change in this total mass within the shell
\citep[][]{1986ApJ...301...27B},
\begin{align}
    r_i \,M_i(r_i) = r_f \,M_f(r_f) %
    \implies 
\frac{r_f}{r_i} = \frac{M_i(r_i)}{M_f(r_f)}\,. 
\label{eq:AR}
\end{align}
The quasi-adiabatic relaxation framework is an empirical extension to this idealised scenario and considers the relaxation ratio $r_f/r_i$ as a function of the mass ratio $M_i/M_f$.
\begin{align}
\frac{r_f}{r_i} &= 1 + \chi \left( \frac{M_i(r_i)}{M_f(r_f)} \right) 
\label{eq:qAR}
\end{align}
In a simple extension, the baryonification procedures in \cite{2015JCAP...12..049S,2021MNRAS.503.4147P} include dark matter response as a quasi-adiabatic relaxation with $\chi(y) = q (y-1)$. However, a variety of quasi-adiabatic models have been proposed \citep{2023Velmani&Paranjape}.  

\subsection{Relaxation offset}
% \subsubsection{Model dependent}
In our previous work \cite{2023Velmani&Paranjape}, we found that 
% Following Velmani \& Paranjape (2023) \cite{2023Velmani&Paranjape} the 
the relaxation relation follows a locally linear relation. 
\begin{align}
    \label{eq:chi-linear-q0}
    \frac{r_f}{r_i} - 1 &= q_1(r_f) \left[ \frac{M_i(r_i)}{M_f(r_f)} - 1 \right] + q_0(r_f)\,.
\end{align}
For a given halo sample, at each $r_f$, the relation between mass ratio and relaxation ratio across all the haloes is fitted by a linear curve to obtain the parameters $q_0(r_f)$ and $q_1(r_f)$. It was also found that the $q_0(r_f)$ is usually roughly uniform from the inner to the outer haloes. In this work, we have tested that this radial dependent linear model is a good description of the halo relaxation over their evolution history back to at least redshift $z=5$ in populations of all six halo masses considered and at halo centric distance from the virial radius to at least $5 \%$ of that radius. To characterize the halo-to-halo variation in the relaxation offset, we define the following relaxation quantity for each halo in the population. 
\begin{align}
\label{eq:def-q0hal}
q_0 |_{h}(r_f) &\equiv \frac{r_f}{r_i} - 1 - q_1(r_f) \left[ \frac{M_i(r_i)}{M_f(r_f)} - 1 \right]\,.
\end{align}
This quantity $q_0 |_{h}(r_f)$ is usually uniform within each of the haloes; We primarily focus on its mean in this work, denoted simply as $q_0$ for that halo. We call this a relaxation offset parameter since it characterizes the amount of relaxation of dark matter shells that have no change in the total enclosed mass. In addition to this, we also study the mean $q_0(r)$ in the inner ($\sim 10\%~R_{\rm{vir}}$) and the outer ($\sim 50\%~R_{\rm{vir}}$) halo defined as $q_0^{\rm{in}} ~\& ~q_0^{\rm{out}}$ respectively.

\begin{figure}[htbp]
\centering
\includegraphics[width=\linewidth]{plots/dynam_relxn/hal_relxn_offset_evolve.pdf}    \caption{Dynamical evolution of the relaxation offset parameters $q_0$ (left panel) and $q_y$ (right panel) averaged in halo populations selected by their mass indicated by the same colour coding as in \figref{fig:evolution-hal-gal-props}}
\label{fig:evolution-hal-reln-offset}
\end{figure}
% \subsubsection{Model independent}
In addition to this model parameter $q_0$, we also define a somewhat model-agnostic quantity to characterize just the offset in the relaxation. For each matched halo, the y-intercept of the relaxation relation is given by the relation between $M_i/M_f-1$ and $r_f/r_i-1$ for each matched halo. This parameter, denoted as $q_y$, is the offset in the relaxation ratio $r_f/r_i$ from unity for the shells having a mass ratio of unity $M_i/M_f=1$. The evolution history of these relaxation offset parameters for our halo populations is shown in \figref{fig:evolution-hal-reln-offset}. We find that the average evolution of $q_0$ and $q_y$ are similar for all haloes. The relaxation offset starts with smaller values and reaches peak magnitudes before approaching zero, consistent with the $z \sim 0$ results from our previous work \citep{2023Velmani&Paranjape}. We have also checked that they show good correlation across haloes, especially $q_y$ shows good correlation with the $q_0^{\rm{in}}$. Hence, in this study, we will primarily focus the relaxation offset parameters $q_0(r)$,  $q_0^{\rm{in}} ~\& ~q_0^{\rm{out}}$.
% While the average evolution suggests that this offset parameter peaks following the peak star formation period, 





\section{Statistical analysis}
\label{sec:methods-stat}
The average evolution of the relaxation parameters shown in \figref{fig:evolution-hal-reln-offset} and other properties shown in \figref{fig:evolution-hal-gal-props} already suggest that the relaxation offset is strongest following the peak star formation period. However, we must study their evolution in individual haloes to further understand the role of different astrophysical processes. This additional information can be captured using correlations not only across haloes in the population but also over time between these quantities. In the \secref{sec:sfr-metallicity} below, we develop this methodology as we explore the connection between star formation rate and metallicity ($Z^{\star}_{\rm{O}}$) through these correlations. In the following (sub)section, we use this method to study the connection between relaxation and various galactic processes. 

\begin{figure}[bhtp]
\centering
\includegraphics[width=\linewidth]{plots/dynam_relxn/Spea_correl_betw_SFR-Z(O)_SFreg.pdf}
\caption{Correlation across haloes between star formation rate and metallicity ($Z^{\star}_{\rm{O}}$) in halo populations selected by their final masses at redshift $z\sim 0$. In the images, the colour at a given $(t[\rm{SFR}],t[Z^{\star}_{\rm{O}}])$ represents the Spearman rank correlation coefficient between the SFR at time $t[\rm{SFR}]$ and metallicity at time $t[Z^{\star}_{\rm{O}}]$ across haloes for each pair of times.}
\label{fig:dynam-correl-sfr-ZOsfr-img}
\end{figure}

\subsection{Star formation rate and metallicity connection}
\label{sec:sfr-metallicity}
As a warm-up, and as a means to showcase our method, we first present the correlation between the $Z^{\star}_{\rm{O}}$ 
% \AP{use the already introduced notation $Z_O$ or whatever, instead of spelling out in words each time} 
and the SFR for haloes selected by their mass at redshift $z\sim 0$. 
In the simulation, metals are added to the gas by various feedback processes such as stellar winds and supernovae explosions that are dominantly produced by the short-lived massive stars. Hence, the metallicity is expected to rise following periods of star formation activity. To characterize this information, we focus on the Spearman rank correlations between the SFR and $Z^{\star}_{\rm{O}}$ at different times.

For a given population of $N$ haloes, we have the values of these two quantities for individual haloes tracked in distinct time steps $t_i$ with $i=1,2,\ldots,T$ separated by a uniform time interval $\Delta t = 157$ Myr. Using this, we construct the $T \times T$ matrix of Spearman correlation coefficients denoted by $\rho^s_{ij} [ \rm{SFR}, Z^{\star}_{\rm{O}}]$. Each element $ij$ of this matrix corresponds to the correlation between SFR at time $t_i$ with $Z^{\star}_{\rm{O}}$ at time $t_j$. (Note that the matrix is not expected to be symmetric.) Suppose $\rm{SFR}(n,t_i)$ and $Z^{\star}_{\rm{O}}(n,t_i)$ denote the corresponding quantities in the $n^{\rm{th}}$ halo at time $t_i$ in a population of $N$ haloes, then,
% 
\begin{align}
% \rho^s_{ij} [\rm{SFR}, Z^{\star}_{\rm{O}}] &= \rho^s [~\rm{SFR}(n \in [1,N],t_i), ~Z^{\star}_{\rm{O}}(n \in [1,N],t_j)~] \\%= \rho [R_{\rm{SFR}}(t_i), R_{Z^{\star}_{\rm{O}}}(t_j)]\\
% \intertext{\PV{Alternative notation}}
\rho^s_{ij} [\rm{SFR}, Z^{\star}_{\rm{O}}] &= \rho^s \left[~\rm{SFR}(\{1 \ldots N\},t_i), ~Z^{\star}_{\rm{O}}(\{1 \ldots N\},t_j)~\right] \\
\intertext{where the Spearman correlation $\rho^s$ is defined as the Pearson correlation ($\rho$) of the ranks ($R$), which in turn can be expressed as the covariance normalized by their standard deviations ($\sigma$).}
\rho^s_{ij} [\rm{SFR}, Z^{\star}_{\rm{O}}] &= \rho \left[~R[~{\rm{SFR}}(\{1 \ldots N\},t_i)~], ~R[~Z^{\star}_{\rm{O}}(\{1 \ldots N\},t_j)~]~\right]\\
&= \frac{\text{cov}\left(R[~{\rm{SFR}}(\{1 \ldots N\},t_i)~], R[~Z^{\star}_{\rm{O}}(\{1 \ldots N\},t_j)~]~\right)}{\sigma\left(R[{\rm{SFR}}(\{1 \ldots N\},t_i)]~\right) \times \sigma\left(R[Z^{\star}_{\rm{O}}(\{1 \ldots N\},t_j)]~\right)}\\
% &= \frac{ (1/N) \sum_{n=1}^{N} R_{\rm{SFR}}(n,t_i) \times  R_{Z^{\star}_{\rm{O}}}(n,t_j) - \mu_{1 \leq n \leq N} [ R_{\rm{SFR}}(n,t_i) ]  \times \mu_{1 \leq n \leq N} [R_{Z^{\star}_{\rm{O}}}(n,t_j)] }{ \sigma_{1 \leq n \leq N} [ R_{\rm{SFR}}(n,t_i) ]  \times \sigma_{1 \leq n \leq N} [R_{Z^{\star}_{\rm{O}}}(n,t_j)] }\\
% \intertext{More explicitly, \PV{Below equation can be removed} \AP{no, it is important. just combine it into the prev eqn, removing the text `More explicitly'}} 
&= \frac{  (1/N) \sum_{n=1}^{N} R_{\rm{SFR}}(n,t_i) \times  R_{Z^{\star}_{\rm{O}}}(n,t_j) - \mu_{n} [ R_{\rm{SFR}}(n,t_i) ]  \times \mu_{n} [R_{Z^{\star}_{\rm{O}}}(n,t_j)] }
{ \sigma_{n} [ R_{\rm{SFR}}(n,t_i) ]  \times \sigma_{n} [R_{Z^{\star}_{\rm{O}}}(n,t_j)] }
\end{align}
% 
where $R_{\rm{SFR}}$ and $R_{Z^{\star}_{\rm{O}}}$ denote the corresponding ranks, while $\mu_n$ and $\sigma_n$ represent the mean and standard deviation, respectively computed across haloes ($n=1$ to $N$) at each time step. This matrix is depicted in \figref{fig:dynam-correl-sfr-ZOsfr-img} for the six populations of our evolving haloes selected by their final masses at $z \sim 0$; We find blue patches of strong positive correlation mostly near the diagonal. 
% \st{We find that there is a strong positive correlation between them at all halo masses indicated by the blue patches near the diagonal.} 
% \AP{i don't understand this argument. there are also red patches. and the diagonal is not always blue, e.g. for M=12.0. so at this point in the text how can you justify an overall positive correlation at all masses and all time lags?} \PV{It was a typo, I meant to say usually instead of overall} 
More specifically, there is usually a stronger positive correlation between the star formation rate at a given time and the $Z^{\star}_{\rm{O}}$ in the near future time. This is expected since the metals are primarily produced by the feedback that follows the star formation activity. 

Another thing to note is the strong negative correlation between the star formation rate of the haloes at later times with metallicity at a very early time and vice versa. 
% We have checked that this is also seen in the auto-correlation of the SFR 
% \st{This is also seen in the auto-correlation of the SFR and} \AP{i guess you are not showing this anywhere? in which case say something like `we have checked that this is also seen...'} \PV{We have checked but anyways I think it is not important to mention here.}
This happens at a characteristic time scale associated with the peak star formation period for that population of haloes. 
For example, in $10^{13} \Mh$ haloes, this transition period happens at a cosmic time of $2$ Gyr; this is indicated by the transition between blue and red patches. 
% \st{the haloes that were forming stars at a higher rate \underline{ended up}} \AP{are you guessing or do you have proof? if guessing then qualify with something like `likely ended up'} with lower star formation activity at later times. 
%(see \figref{fig:dynam-correl-sfr-ZOsfr-img}).
% 
While the metallicity is initially raised by the feedback events, the same feedback pushes inner metal-rich gas into the circumgalactic medium (CGM). On the other hand, the inflowing metal-poor gas from CGM lowers the metallicity while being crucial to feed the star formation activity at a later time. 
This leads to the inverse relationship between star formation rate and metallicity at fixed stellar mass reported in the literature \citep[][]{2011DaveFinlatorOppenheimer,2018TorreyVogelsberger_etal_SFRZ}; and our results are consistent with it.
% The gas inflow from CGM is usually followed by an enhanced star formation activity.
% The x-axis shows the time in scale factor at which the SFR is 
% In order to understand the halo and galaxy properties that are involved in mediating the relaxation, we look at the statistical correlation between the observed properties of those haloes in the simulation as a function of time. There are two different approaches 

\subsubsection{Correlation at fixed time intervals}
\label{sec:halhal-corr-sfrZ}
% \PV{Entire section rewritten} \\
We now focus on using the correlations to explore the repetitive nature of galactic processes. As a preliminary approach, we quantify this using the average of the Spearman rank correlation matrix $\rho^s_{ij}$ along the diagonals parallel to the black line shown in each panel of the \figref{fig:dynam-correl-sfr-ZOsfr-img}. This essentially represents the time-averaged correlation across haloes between SFR and $Z^{\star}_{\rm{O}}$ at a fixed earlier or later time, defined as:
%\AP{what exactly should the reader look at in the figure?} 
\begin{align}
\rho^s_h(\tau)[\rm{SFR}, Z^{\star}_{\rm{O}}] &= \avg{ \rho^s [ \rm{SFR}(t), Z^{\star}_{\rm{O}}(t+\tau)] }_{t}  
% \\ &=
% \begin{cases}
% \sum_{t_i=t_1}^{t_T-\tau} \rho^s [\rm{SFR}(t_i), Z^{\star}_{\rm{O}}(t_i+\tau)] \qquad \text{if } \tau \geq 0\\
% \sum_{t_i=\tau}^{t_T} \rho^s [\rm{SFR}(t_i), Z^{\star}_{\rm{O}}(t_i+\tau)] \qquad \text{if } \tau < 0
% \end{cases}
\end{align}
This quantity is computed at distinct time steps $(\tau=m \Delta t)$ for integer $m = 0,\pm 1,\ldots,\pm T$ using the matrix $\rho^s_{ij}$ as,
\begin{align}
\label{eq:def-hal-correl}
\rho^s_h(\tau = m \Delta t) 
&= \avg{ \rho^s_{i,i+m} }_i \\
&= \begin{cases}
\sum_{i=1}^{T-m} \rho^s_{i,i+m}/(T-m) \qquad \text{if } m \geq 0 \\
\sum_{i=1-m}^{T} \rho^s_{i,i+m}/(T-m) \qquad \text{if } m < 0
\end{cases}
\end{align}
Similar to the standard Spearman correlation coefficient, this quantity takes on a value between $-1$ and $+1$ at each $\tau$, indicating the strength of the correlation from being strongly negative to positive.
% As shown in \figref{fig:dynam-correl-sfr-ZOsfr-timeshift-func-halhal}, this quantity depicts that there is generally a stronger positive correlation between SFR with the future $Z^{\star}_{\rm{O}}$ in our halo population. This is clearly noticeable for the lowest (green curves) and highest halo masses (black and purple curves).
% Note that while this quantity averages over time, only the halo-to-halo correlation between the quantities is considered. This ignores the correlation between temporal variations in the star formation rate and metallicity. To capture that, we need to consider the correlation over time in their evolutionary tracks.
As shown in \figref{fig:dynam-correl-sfr-ZOsfr-timeshift-func-halhal}, this analysis reveals that the positive correlation is strongest between the SFR and the immediate future $Z^{\star}_{\rm{O}}$ as expected. Also, notice that on average, there is a stronger positive correlation $\rho^s_h$ for $\tau>0$ than $\tau<0$; This is particularly evident for the lowest (green curves) and highest halo masses (black and purple curves).
While there is a relatively stronger positive correlation at intermediate masses (blue curves), it is similar with both positive and negative time lags. This is a consequence of the fact that the $\rho^s_h(\tau)$ focuses only on the halo-halo correlation. Although $\rho^s_h(\tau)$ involves summation over both haloes and their evolutionary time steps, the ranks are computed independently at each of the time steps. Hence, it does not capture the correlation between temporal variations in the SFR and $Z^{\star}_{\rm{O}}$ of individual haloes, especially when the corresponding halo-to-halo variations are relatively stronger. In the following section, we present another quantity that can directly probe the correlations over time.


\begin{figure}[htbp]
\centering
\includegraphics[width=.6\linewidth]{plots/dynam_relxn/Spea_correl_vs_shift_betw_SFR-Z(O)_SFreg_halhalcorr.pdf}
\caption{Spearman correlation $\rho^s_h(\tau)$ across haloes between SFR and metallicity ($Z^{\star}_{\rm{O}}$) with a time lag. This is shown for the six populations of haloes selected by their final masses at redshift $z\sim 0$; Colour coding follows \figref{fig:evolution-hal-gal-props}.}
\label{fig:dynam-correl-sfr-ZOsfr-timeshift-func-halhal}
\end{figure}

\subsubsection{Time Correlations}
\label{sec:time-correl-sfrZ}
% \PV{Entire section rewritten} \\
In order to further understand the connection between the sequences of galactic processes, we study time correlations that capture the temporal variations along the evolutionary tracks of each halo.
% The halo-halo correlation might not capture the relevant information if the sample variance is either much smaller or larger than the temporal variations. 
Consider an individual halo along with its evolutionary track, say $n^{\rm{th}}$ halo in a population of N haloes. We characterize the time correlation through the Spearman correlation as a function of time lag, $\rho^s_t |_{n}(\tau)[\rm{SFR}, Z^{\star}_{\rm{O}}]$ defined as follows.
\begin{align}
\rho^s_t |_{n}(\tau)[\rm{SFR}, Z^{\star}_{\rm{O}}] &= \rho^s [ ~\rm{SFR}(n,t), Z^{\star}_{\rm{O}}(n,t+\tau) ~] \quad \text{where } ~t, t+\tau \in [t_1,t_T].
\end{align}
This quantity is also computed at distinct time steps ($\tau = m \Delta t$) for integer $m=0, \pm1, \ldots, \pm T$ as:
\begin{align}
\label{eq:def-time-correl-nth-halo}
\rho^s_t |_{n}(\tau = m \Delta t) &= 
\begin{cases}
\rho^s [ ~\rm{SFR}(n,\{t_1 \ldots t_{T-m}\}), ~Z^{\star}_{\rm{O}}(n,\{t_{1+m} \ldots t_T\}) ~]  \qquad \text{if } m \geq 0 \\
\rho^s [ ~\rm{SFR}(n,\{t_{-m+1} \ldots t_T\}), ~Z^{\star}_{\rm{O}}(n,\{t_1 \ldots t_{T+m}\}) ~] \qquad \text{if } m < 0
\end{cases}.
\end{align}

Alternatively, cross-correlation is also a characterization of the correlation over time in that halo. In particular, consider the cross-correlation function between the normalized ranks of SFR and $Z^{\star}_{\rm{O}}$ over its evolutionary history given by, 
\begin{align}
\rho^s_{t'} |_{n}(\tau = m \Delta t) 
% &= \sum_{} R^{\rm{norm}}_{\rm{SFR}}(n,t_i) \times R^{\rm{norm}}_{Z^{\star}_{\rm{O}}}(n,t_i)
&= \begin{cases}
\sum_{i=1}^{T-m} R^{\rm{norm}}_{\rm{SFR}}(n,t_i) \times R^{\rm{norm}}_{Z^{\star}_{\rm{O}}}(n,t_{i+m})/(T-m) \qquad \text{if } m \geq 0 \\
\sum_{i=1-m}^{T} R^{\rm{norm}}_{\rm{SFR}}(n,t_i) \times R^{\rm{norm}}_{Z^{\star}_{\rm{O}}}(n,t_{i+m})/(T-m) \qquad \text{if } m < 0
\end{cases}
\end{align}
where the normalized ranks are defined as
\begin{align}
R^{\rm{norm}}_{\rm{SFR}}(n,\{t_1 \ldots t_T\}) &= \frac{R[\rm{SFR}(n,\{t_1 \ldots t_T\})] - \avg{R[\rm{SFR}(n,\{t_1 \ldots t_T\})]}}{\sigma[R[\rm{SFR}(n,\{t_1 \ldots t_T\})]]},\\
R^{\rm{norm}}_{Z^{\star}_{\rm{O}}}(n,\{t_1 \ldots t_T\}) &= \frac{R[Z^{\star}_{\rm{O}}(n,\{t_1 \ldots t_T\})] - \avg{R[Z^{\star}_{\rm{O}}(n,\{t_1 \ldots t_T\})]}}{\sigma[R[Z^{\star}_{\rm{O}}(n,\{t_1 \ldots t_T\})]]}.
\end{align}
% These two quantities would agree at zero time-lag $\tau=0$, they ma

Both $\rho^s_t$ and $\rho^s_{t'}$ probe the time correlations, but they are not necessarily equal. To illustrate the difference, consider the $n^{\rm{th}}$ halo's evolutionary history over a period of 12 billion years. For a time-lag of 3 billion years between SFR and $Z^{\star}_{\rm{O}}$, $\rho^s_{t} |_{n}(\tau=3~\rm{Gyr})[\rm{SFR}, Z^{\star}_{\rm{O}}]$ is the Spearman correlation coefficient between SFR data from 12 billion years ago to 3 billion years ago and the metallicity from 9 billion years ago to the present. Here, the ranks, mean, and variance of the SFR are computed on the subset of data from 12 billion years ago until 3 billion years ago. As the average SFR evolves over time, the mean computed on this subset may differ from the mean SFR across the entire duration of 12 billion years. Therefore, the correlation function of the normalized ranks ($\rho^s_{t'} |_{n}(\tau=3~\rm{Gyr})[\rm{SFR}, Z^{\star}_{\rm{O}}]$) does not have to be equal to the actual Spearman correlation ($\rho^s_{t} |_{n}(\tau=3~\rm{Gyr})[\rm{SFR}, Z^{\star}_{\rm{O}}]$). The corresponding sample mean values of these two quantities of time correlations in each of the halo populations are defined as follows:
\begin{align}
\label{eq:def-time-correl}
\rho^s_{t}(\tau)[\rm{SFR}, Z^{\star}_{\rm{O}}] = \avg{ \rho^s_{t} |_{n}(\tau)[\rm{SFR}, Z^{\star}_{\rm{O}}] }_n, \\ 
\rho^s_{t'}(\tau)[\rm{SFR}, Z^{\star}_{\rm{O}}] = \avg{ \rho^s_{t'} |_{n}(\tau)[\rm{SFR}, Z^{\star}_{\rm{O}}] }_n.
\end{align}
% This is in contrast to $\rho^s_h(\tau)$ quantifying the time-averaged sample correlation across haloes in each population. 
% To address this, we define \(\rho^s_t(\tau)\), which quantifies the correlation over time along the evolutionary tracks of each halo, averaged across the population.

These time correlations ($\rho^s_t ~\&~ \rho^s_{t'}$) are shown by the solid and the dashed curves in the left panel of \figref{fig:dynam-correl-sfr-ZOsfr-timeshift-func-all} in each of our halo populations. While they agree at $\tau=0$, they deviate significantly as the time lag becomes larger than the timescales over which the local mean of the quantities evolve. In this work, we will only focus on the time lags where these two quantities are similar, as marked in the \figref{fig:dynam-correl-sfr-ZOsfr-timeshift-func-all}. In that case, we will consider the Spearman correlation as a function of time $\rho^s_t$; It takes on a value between $-1$ and $+1$, similar to the standard Spearman correlation coefficient. 
\begin{figure}[htbp]
\centering
\includegraphics[width=.49\linewidth]{plots/dynam_relxn/Spea_correl_vs_shift_betw_SFR-Z(O)_SFreg_timecorr.pdf}
\includegraphics[width=.49\linewidth]{plots/dynam_relxn/Spea_correl_vs_shift_betw_SFR-Z(O)_SFreg_fullcorr.pdf}
\caption{Spearman rank correlation coefficients, $\rho^s_t$ (left panel) and $\rho^s_f$ (right panel) as a function of the time lag between SFR and $Z^{\star}_{\rm{O}}$ for haloes selected by their final masses at redshift $z\sim 0$. Additionally, in the left panel the dashed lines indicate cross-correlation $\rho^s_{t'}$ as defined in the main text. Colour coding follows \figref{fig:evolution-hal-gal-props}. The horizontal lines indicate the time lag range that we interpret physically.}
\label{fig:dynam-correl-sfr-ZOsfr-timeshift-func-all}
\end{figure}
%\AP{unfortunately i am not following anything explained in this subsection 4.1.2.  e.g., based on what is written i do not even know whether the quantity shown in fig 5 is calculated in the same way or a different way than the one shown in fig 4. as a referee i might be tempted to reject the paper since the method is too obscure. it will be better to introduce distinct notation for the output of the two calculations and explicit equations showing exactly what is calculated and displayed, rather than just using english words. in the text please explicitly use the notation (which can be identified with some specific equation) rather than words like `full correlation' and `time correlation'. you had explained to me in person way better than what is written here.}

% Now, let us consider the correlation between star formation rate and metallicity ($Z^{\star}_{\rm{O}}$) in our evolving halo populations. The sample average of the Spearman (time) correlation function of individual haloes is shown in the left panel of \figref{fig:dynam-correl-sfr-ZOsfr-timeshift-func-all}.

Notice that the positive time correlation $\rho^s_t>0$ is predominantly seen only with positive time lags $\tau>0$ even at the intermediate halo masses (see blue curves in \figref{fig:dynam-correl-sfr-ZOsfr-timeshift-func-all}). In general, at all halo masses considered, the SFR is correlated with the then future metallicity over a duration of atleast $2~\rm{Gyr}$. In addition, the time correlation also shows some more interesting information. It shows a significant anti-correlation between SFR and the \textbf{past} metallicity ($Z^{\star}_{\rm{O}}$) in most of our halo populations. This can again be interpreted in terms of lowering metallicity by the gas inflow from CGM, which also feeds the future star formation activity. 

The time correlation $\rho^s_t$ focuses on the temporal variations, but it doesn't consider the sample variation in the halo population. We define the full correlation, $\rho^s_f(\tau)$ by simultaneously correlating both across the sample of haloes and over their evolutionary tracks as follows:
\begin{align}
\rho^s_f (\tau)[\rm{SFR}, Z^{\star}_{\rm{O}}] &= \rho^s [ ~\rm{SFR}(n,t), Z^{\star}_{\rm{O}}(n,t+\tau) ~] \quad \text{where } n \in [1,N] ~\&~ t, t+\tau \in [t_1,t_T].
\end{align}
When computed at distinct time steps ($\tau = m \Delta t$) for integer $m=0, \pm1, \ldots, \pm T$, we have,
\begin{align}
\label{eq:def-full-corr}
\rho^s_f (\tau = m \Delta t) &= 
\begin{cases}
\rho^s [ ~\rm{SFR}(\{1 \ldots N\}, ~\{t_1 \ldots t_{T-m}\}), ~Z^{\star}_{\rm{O}}(\{1 \ldots N\}, ~\{t_{1+m} \ldots t_T\}) ~]  \qquad \text{if } m \geq 0 \\
\rho^s [ ~\rm{SFR}(\{1 \ldots N\}, ~\{t_{-m+1} \ldots t_T\}), ~Z^{\star}_{\rm{O}}(\{1 \ldots N\}, ~\{t_1 \ldots t_{T+m}\}) ~] \qquad \text{if } m < 0
\end{cases}.
\end{align}
% \AP{show the explicit summation that is used in the code, similar to previous explicit expressions}
This considers $(T-m) \times N$ values of both SFR and $Z^{\star}_{\rm{O}}$ at each time lag $\tau = m \Delta t$; Suppose the ranks are denoted as $R^m_{\rm{SFR}}(n,t_i)$ and $R^m_{Z^{\star}_{\rm{O}}}(n,t_i)$, then for a positive value of $m$,
\begin{align}
\rho^s_f (\tau = m \Delta t) &= \frac{ \sum_{i=1}^{T-m} \sum_{n=1}^{N} R^m_{\rm{SFR}}(n,t_i) \times  R^m_{Z^{\star}_{\rm{O}}}(n,t_j) - \mu_{n,t} [ R_{\rm{SFR}}(n,t_i) ]  \times \mu_{n,t} [R_{Z^{\star}_{\rm{O}}}(n,t_j)] }
{ N(T-m) \times \sigma_{n,t} [ R_{\rm{SFR}}(n,t_i) ]  \times \sigma_{n,t} [R_{Z^{\star}_{\rm{O}}}(n,t_j)] }
\end{align}
where the mean ($\mu_{n,t}$), standard deviation ($\sigma_{n,t}$) and the ranks are computed over the $(T-m) \times N$ values,
\begin{align}
R^m_{\rm{SFR}}(\{1 \ldots N\}, ~\{t_1 \ldots t_{T-m}\}) &= R[ ~\rm{SFR}(\{1 \ldots N\}, ~\{t_1 \ldots t_{T-m}\}) ~]\\
R^m_{Z^{\star}_{\rm{O}}}(\{1 \ldots N\}, ~\{t_{1+m} \ldots t_T\}) &= R[ ~Z^{\star}_{\rm{O}}(\{1 \ldots N\}, ~\{t_{1+m} \ldots t_T\}) ~]
\end{align}
This full 2D Spearman correlation coefficient $\rho^s_f[\rm{SFR},Z^{\star}_{\rm{O}}]$ is presented in the right panel of the \figref{fig:dynam-correl-sfr-ZOsfr-timeshift-func-all}.
Once again, we only consider the range of time-lags marked in the figure at each of the halo masses. These are obtained by comparing $\rho^s_f(\tau)$ with the cross-correlation of normalized ranks computed on the entire duration and across all haloes in the population. We find that the full correlation $\rho^s_f$ tends to be stronger than the time correlation $\rho^s_t$, while also including the correlations in temporal variations unlike $\rho^s_h$ defined in \eqref{eq:def-hal-correl}.
% This will be useful later when studying the correlation between other quantities, especially when the underlying correlations are weaker. 
% Once again, we identify the range of time lags where the 
% \begin{align}
% \rho^s_f[\rm{SFR},Z](\tau) = \sum_{t_i=t_1}^{t_f-\tau} \sum_{n \in N} \rm{SFR}(n,t) \times Z(n,t+\tau)
% \end{align}


\subsubsection{Time lag analysis}
\label{sec:time-lag-analysis}
Spearman correlations ($\rho^s_h, \rho^s_t, ~\&~ \rho^s_f$) as a function time lag ($\tau$) as shown in figures \ref{fig:dynam-correl-sfr-ZOsfr-timeshift-func-halhal} and \ref{fig:dynam-correl-sfr-ZOsfr-timeshift-func-all} depict a qualitative picture of the sequences of galactic processes. In this section, we quantify the typical timescales associated with those processes. In this regard, we compute the mean time lag between SFR and metallicity weighed by their correlation strengths. This is done by considering the function of Spearman correlation coefficients in two regions of time lags: the correlation era $\rho_{+}$ and the anti-correlation era $\rho_{-}$.
\begin{gather}
\rho_{+}(\tau) = 
\begin{cases} 
\rho^s_f(\tau) & \text{if } \rho^s_f(\tau) > 0 \\ 
0 & \text{else} 
\end{cases}, \qquad
\rho_{-}(\tau) = 
\begin{cases} 
-\rho^s_f(\tau) & \text{if } \rho^s_f(\tau) < 0 \\
0 & \text{else} 
\end{cases}
\end{gather}
Then, for both the correlation and anti-correlation, we obtain the correlation weighted mean time lag ($\tau_{\pm}$), the mean correlation strength $A_{\pm}$ in that era and the time duration $\Delta\tau_{\pm}$
% \AP{i modified the notation from $d\tau_{\pm}$ to $\Delta\tau_{\pm}$. pls chk that it is changed everywhere.}
over which the correlation is significant.
\begin{gather}
\Delta\tau_{+} = \int_{\rho(\tau)>0} w d\tau , \qquad \Delta\tau_{-} = \int_{\rho<0} w d\tau   \\
A_{+} = \int w(\tau) \rho_{+}(\tau) d\tau / \Delta\tau_{+}, \qquad A_{-} = \int w(\tau) \rho_{-}(\tau) d\tau / \Delta\tau_{-} \\
\tau_{+} = \frac{1}{A_{+} \Delta\tau_{+}} \int \tau w(\tau) \rho_{+}(\tau) d\tau , \qquad \tau_{-} = \frac{1}{A_{-}\Delta\tau_{-}} \int \tau w(\tau) \rho_{-}(\tau) d\tau     
\end{gather}
Here, the weight function $w$ is taken to be the fraction of the entire data between $z \sim 5$ and $z \sim 0$ involved in computing the correlation coefficient at a given time lag. In general, the values of $\tau_{\pm}$ are physical only when the corresponding correlation strengths $A_{\pm}$ and the durations $\Delta \tau_{\pm}$ are both significant. Also note that when correlating between two quantities say $X$ and $Y$ a positive time lag $\tau_{+}[X,Y]$ (or $\tau_{-}[X,Y]$) indicate that the positive (or negative) correlation is seen with $Y$ lagging behind $X$. Hence the mean time lags are anti-symmetric with respect to the quantities, $\tau_{\pm}[X,Y]= - \tau_{\pm}[Y,X]$; However, the strengths and durations of correlations are symmetric, $A_{\pm}[X,Y]= A_{\pm}[Y,X]$ and $\Delta \tau_{\pm}[X,Y]= \Delta \tau_{\pm}[Y,X]$.

\begin{figure}[htbp]
\centering
% \includegraphics[width=.49\linewidth]{plots/dynam_relxn/shift_betw_SFR-Z(O)_SFreg_timecorr.pdf}
\includegraphics[width=.49\linewidth]{plots/dynam_relxn/shift_betw_SFR-Z(O)_SFreg_fullcorr.pdf}
\caption{In the top panel, correlation weighted mean time lags $\tau_{\pm}$ between SFR and $Z^{\star}_{\rm{O}}$ in the six halo populations selected by their final masses at redshift $z\sim 0$ are shown. The corresponding strengths $A_{\pm}$ and the durations $\Delta \tau_{\pm}$ of the correlation are shown in the middle and bottom panels, respectively.}
% Only the time correlations are considered in the left panel, while the right panel considers both sample and time correlation. \PV{shall one of them or combine them}}
\label{fig:dynam-correl-sfr-ZOsfr-timeshift-all}
\end{figure}

The mean time lags ($\tau_{\pm}$), the effective strengths ($A_{\pm}$) and the durations ($\Delta\tau_{\pm}$) of the correlation between the SFR and metallicity $(Z^{\star}_{\rm{O}})$ are shown in the \figref{fig:dynam-correl-sfr-ZOsfr-timeshift-all} for all our halo populations. 
% \AP{sign convention needs to be clarified, namely, the meaning of positive and negative values of $\tau$ when correlating quantity X with quantity Y. later you are using $\tau[X,Y]$, so why not just introduce that notation here and use it to clarify sign convention?}
The positive correlation between SFR and future $Z^{\star}_{\rm{O}}$ is depicted by positive values of $\tau_{+}[\rm{SFR}, Z^{\star}_{\rm{O}}]$. At low masses $(<10^{12.5}\Mh)$ this time lag is as large as $\tau_{+}=\sim 2 ~\rm{Gyr}$. At higher masses, though, the time lags are smaller, along with relatively lower correlation strengths. This is likely due to the strong AGN feedback redistributing the metals from the star-forming regions. In addition, the negative correlation identified qualitatively in the previous section can also be seen in the \figref{fig:dynam-correl-sfr-ZOsfr-timeshift-all} with large negative time-lags ($\tau_{-} \sim -3~ \rm{Gyr}$) in some halo populations (shown by dashed curves).
%\AP{weren't you going to explain that one should only seriously consider signals in regions where the correlation strength is high and the interval $\Delta\tau$ is significantly away from zero?}


\subsection{Interplay of halo relaxation with galactic processes}
\label{sec:main-res-halgal-relxn}
In this section, we study the correlations between the relaxation offset and various other properties of the halo/galaxy. To quantify the strength of the relaxation offset we define $Q_0 \equiv -q_0$; similarly we also define $Q_0^{\rm{in}}\equiv -q_0^{\rm{in}} \quad \& \quad  Q_0^{\rm{out}}\equiv -q_0^{\rm{out}}$ to characterize the strength of the relaxation offset in the inner and the outer regions of the halo respectively. Following the procedure described in the previous section, we primarily focus on the correlation-weighted time lags and their effective correlation strengths in this analysis.

\begin{figure}[htbp]
\centering
% \includegraphics[width=.49\linewidth]{plots/dynam_relxn/Spea_correl_vs_shift_betw_q0-SFR_fullcorr.pdf}
\includegraphics[width=.69\linewidth]{plots/dynam_relxn/shift_betw_multi-SFR_fullcorr.pdf}
% \includegraphics[width=.49\linewidth]{plots/dynam_relxn/Spea_correl_vs_shift_betw_q0(r)_in-SFR_fullcorr.pdf}
% \includegraphics[width=.49\linewidth]{plots/dynam_relxn/shift_betw_q0(r)_in-SFR_fullcorr.pdf}
% \includegraphics[width=.49\linewidth]{plots/dynam_relxn/Spea_correl_vs_shift_betw_q0(r)_out-SFR_fullcorr.pdf}
% \includegraphics[width=.49\linewidth]{plots/dynam_relxn/shift_betw_q0(r)_out-SFR_fullcorr.pdf}
\caption{In the top panel, correlation weighted mean time lags ($\tau_{\pm}$) between relaxation offset strengths ($Q_0, Q_0^{\rm{in}}, Q_0^{\rm{out}}$) against the star formation rate (SFR) in the six halo populations selected by their final masses at redshift $z\sim 0$ are shown. The corresponding strengths ($A_{\pm}$) and the durations ($\Delta \tau_{\pm}$) of the correlation are shown in the middle and bottom panels, respectively.}
\label{fig:dynam-correl-q0-SFR-timeshift-func}
\end{figure}

Let us start by studying the correlations between relaxation offset strengths and the SFR shown in the \figref{fig:dynam-correl-q0-SFR-timeshift-func}. In each case, a positive time lag of positive or negative correlation would mean that the corresponding correlation is seen with SFR lagging behind the offset strength; On the other hand, a negative time lag would indicate that the offset strengths lag behind the SFR with that correlation. We find that at all masses $Q_0$ shows a significant correlation with SFR over a long duration of $\Delta \tau_{+}\sim 6$ Gyr with $\tau_{+}$ close to zero. This is more prominent at high masses with effective correlation strength $A_{+}$ as high as $0.35$ at $10^{13.5} \Mh$. This is consistent with our previous work \citep{2023Velmani&Paranjape} where we have found that the relaxation offset to be stronger among haloes with higher specific star formation rate at redshift zero. In addition, we also see that at all masses except $10^{11.5}\Mh$, the mean time lag of positive correlation is negative with a magnitude as large as 1 Gyr in intermediate-mass haloes. This suggests that the higher star formation activity is followed by an enhanced relaxation offset. This means the past SFR is a better predictor of the current relaxation offset.

Also notice that at all masses, $Q_0^{\rm{in}}$ shows a stronger correlation and with a higher time lag $\tau_{+}[Q_0^{\rm{in}},\rm{SFR}] > \tau_{+}[Q_0,\rm{SFR}]$. On the other hand, $Q_0^{\rm{out}}$ shows a significantly weaker correlation but with time lags $\tau_{+}[Q_0^{\rm{out}},\rm{SFR}] < \tau_{+}[Q_0,\rm{SFR}]$. This suggests that the astrophysical processes associated with star formation activity produce relaxation offset starting from the inner to the outer halo. The outer halo shows this positive correlation only for a shorter duration while negative correlation are stronger and seen for a longer duration ($\Delta \tau_{-}\sim 6 ~\rm{Gyr}$). With $\tau_{-}[Q_0^{\rm{out}},\rm{SFR}]>0$, this negative correlation with SFR is mostly seen over past relaxation offset in the outer halo.

% However, the inner relaxation offset is correlated with SFR in the more recent past compared to the outer relaxation offset, which is correlated to much earlier SFR. This is depicted in the figure by a more negative value of $\tau_{+}[\rm{SFR},Q_0^{\rm{out}}]$ compared to the $\tau_{+}[\rm{SFR},Q_0^{\rm{in}}]$ for high mass haloes. Also, we find this positive correlation to be much weaker in low-mass haloes than in high-mass haloes.

\begin{figure}[htbp]
\centering
% \includegraphics[width=.49\linewidth]{plots/dynam_relxn/shift_betw_SFR-Z(O)_SFreg_fullcorr.pdf}
% \includegraphics[width=.49\linewidth]{plots/dynam_relxn/Spea_correl_vs_shift_betw_q0-Z(O)_SFreg_fullcorr.pdf}
\includegraphics[width=.69\linewidth]{plots/dynam_relxn/shift_betw_multi-Z(O)_SFreg_fullcorr.pdf}
% \includegraphics[width=.49\linewidth]{plots/dynam_relxn/Spea_correl_vs_shift_betw_q0(r)_in-Z(O)_SFreg_fullcorr.pdf}
% \includegraphics[width=.49\linewidth]{plots/dynam_relxn/shift_betw_q0(r)_in-Z(O)_SFreg_fullcorr.pdf}
% \includegraphics[width=.49\linewidth]{plots/dynam_relxn/Spea_correl_vs_shift_betw_q0(r)_out-Z(O)_SFreg_fullcorr.pdf}
% \includegraphics[width=.49\linewidth]{plots/dynam_relxn/shift_betw_q0(r)_out-Z(O)_SFreg_fullcorr.pdf}
\caption{In the top panel, correlation weighted mean time lags ($\tau_{\pm}$) between relaxation offset strengths ($Q_0, Q_0^{\rm{in}}, Q_0^{\rm{out}}$) and SFR against the metallicity ($Z^{\star}_{\rm{O}}$) in the six halo populations selected by their final masses at redshift $z\sim 0$ are shown. The corresponding strengths ($A_{\pm}$) and the durations ($\Delta \tau_{\pm}$) of the correlation are shown in the middle and bottom panels, respectively.}
\label{fig:dynam-correl-q0-ZOsfr-timeshift-func}
\end{figure}

Moreover, the above result suggests that the effect of star formation activity on the relaxation offset is more immediate on the inner halo compared to the outer halo. This is consistent with the argument that the feedback from massive, short-lived stars that follow the star formation activities is causing the offset in the relaxation.  However, this also means that a measure of the recent feedback processes should correlate even more strongly with the relaxation offset strength. To test this, we have studied the correlations between the relaxation offset strengths and metallicity of oxygen in the star-forming regions ($Z^{\star}_{\rm{O}}$); this is shown in \figref{fig:dynam-correl-q0-ZOsfr-timeshift-func}

% the rate of increase in the oxygen mass in gas $M_{\rm{O}}' \equiv d M_{\rm{O}}/ dt$; this is shown in the \figref{fig:dynam-correl-q0-dMOFoF-timeshift-func}.


We find that the relaxation offset strength does show a positive correlation with $Z^{\star}_{\rm{O}}$, albeit weaker than the correlation found with SFR. 
% and also with an earlier time of $2 \rm{Gyr}$ 
With the exception of the highest halo masses, the positive correlation is first seen in the inner halo followed by the outer halo. This is depicted by 
\begin{align}
\tau_{+}[Q_0^{\rm{in}},Z^{\star}_{\rm{O}}] > \tau_{+}[Q_0^{\rm{out}},Z^{\star}_{\rm{O}}] > \tau_{+}[Q_0,Z^{\star}_{\rm{O}}]
\end{align}
Also, the correlations are relatively weaker for high-mass haloes where the correlation between SFR and $Z^{\star}_{\rm{O}}$ is also weaker. In $10^{13} \Mh$ haloes, there is a significant correlation with SFR ($A_{+}[\rm{SFR},Z^{\star}_{\rm{O}}] \sim 0.35$), but interestingly $Q_0$ correlates with much earlier $Z^{\star}_{\rm{O}}$ of over 4 Gyr in the past, while it correlates with SFR at just 1 Gyr in the past on average.
% we have \tau_{+}[Q_0^{\rm{out}},Z^{\star}_{\rm{O}}] close to - 4 Gyr indicating th
% \begin{algin}
% \tau_{+}[Q_0^{\rm{out}},Z^{\star}_{\rm{O}}] > \tau_{+}[Q_0^{\rm{out}},\rm{SFR}]
% \end{algin}
This may be because the feedback alone does not mediate the relaxation offset. At least we can say that the metallicity is not strongly associated with the feedback processes related to star formation activity involved in producing relaxation offset.

Since the metallicity in the star forming regions is also affected by the inflowing gas and redistribution by AGN feedbacks, we also study correlations with the overall rise in metals. In particular, let us now focus on the correlations between relaxation offset strengths and the rise in total oxygen mass in the gas content of the entire halo ($M_{\rm{O}}'$).  While $Z^{\star}_{\rm{O}}$ is an integrated quantity, $M_{\rm{O}}'$ tracks instantaneous increase in the metals. However, $M_{\rm{O}}'$ also shows only a weaker correlation with the relaxation offset compared to correlations with SFR (see \figref{fig:dynam-correl-q0-dMOFoF-timeshift-func}). We find that the loss of metals in the gas to the newly formed stars dominates over the metal enrichment by feedback. Hence SFR is negatively correlated with the future $M_{\rm{O}}'$ 
depicted by positive values of $\tau_{-}[\rm{SFR},M_{\rm{O}}']$.
% \AP{are you sure the negative sign isn't just because you switched sign convention compared to fig 6? i.e. are you actually plotting $\tau_{-}[M_{\rm{O}}',\rm{SFR}]$?} \PV{I have verified and the sign convention is same as fig 6; In both cases it is SFR first and then metallicity,  $\tau_{\pm}[\rm{SFR},Z^{\star}_{\rm{O}}]$ and $\tau_{-}[\rm{SFR},M_{\rm{O}}']$}

% Due to the We have then studied the correlations with the metallicity $Z^{\star}_{\rm{O}}$ of oxygen in the star-forming regions. We have already seen that this metallicity is an integrated quantity, and it has been correlated with the star formation rate over a long time. 
% However, this metallicity ($Z^{\star}_{\rm{O}}$) also shows only a weaker correlation with the relaxation offset compared to correlations with SFR (see \figref{fig:dynam-correl-q0-ZOsfr-timeshift-func}). 

Finally, we have also studied the correlations with the wind mass $M_{\rm{Wind}}$; this is presented in \figref{fig:dynam-correl-q0-Mwind-timeshift-func}. We find that the wind mass correlates with relaxation offset strengths in the future, affecting the inner halo earlier than the outer halo. We also find that the relaxation offset strength is correlated (although weakly) with the wind mass at a much later time in the future in low-mass haloes compared to high-mass haloes. We do see a similar trend in the correlation between wind mass and SFR; wind mass positively correlates with the star formation activity at a later time for lower mass haloes. However, we note that for the $10^{11.5}\Mh$ haloes, the wind mass correlates with the star-forming activity nearly 3.5 Gyr later. On the other hand, it correlates with the relaxation offset strength at just 2 Gyr in the future. Also, this correlation between wind mass and the relaxation offset is slightly stronger than the correlation we saw with SFR. This suggests that the wind mass is likely more closely associated with the relaxation offset than the star formation rate itself.

%\AP{what is the main result of this section? there is a collection of plots but no clear message. i had asked for typical numbers for the lag times in Gyr. pls report these wherever appropriate in the text for the main correlations studied here, because it is very difficult to read these off from the plots.}

% Since 

% metallicity of oxygen in the star-forming $Z^{\star}_{\rm{O}}$.
%  The total mass of wind within the FoF group halo $M_{\rm{Wind}}$.
%  The total oxygen content within the FoF group halo $M_{\rm{O}}$.

% In the \figref{fig:dynam-correl-q0-dMOFoF-timeshift-func}, , 


\begin{figure}[htbp]
\centering
% \includegraphics[width=.49\linewidth]{plots/dynam_relxn/Spea_correl_vs_shift_betw_q0-dM(O)_FoF_fullcorr.pdf}
\includegraphics[width=.69\linewidth]{plots/dynam_relxn/shift_betw_multi-dM(O)_FoF_fullcorr.pdf}
% \includegraphics[width=.49\linewidth]{plots/dynam_relxn/Spea_correl_vs_shift_betw_q0(r)_in-dM(O)_FoF_fullcorr.pdf}
% \includegraphics[width=.49\linewidth]{plots/dynam_relxn/shift_betw_q0(r)_in-dM(O)_FoF_fullcorr.pdf}
% \includegraphics[width=.49\linewidth]{plots/dynam_relxn/Spea_correl_vs_shift_betw_q0(r)_out-dM(O)_FoF_fullcorr.pdf}
% \includegraphics[width=.49\linewidth]{plots/dynam_relxn/shift_betw_q0(r)_out-dM(O)_FoF_fullcorr.pdf}
% \includegraphics[width=.49\linewidth]{plots/dynam_relxn/shift_betw_SFR-dM(O)_FoF_fullcorr.pdf}
\caption{In the top panel, correlation weighted mean time lags ($\tau_{\pm}$) between relaxation offset strengths ($Q_0, Q_0^{\rm{in}}, Q_0^{\rm{out}}$) and SFR against the rise in the oxygen mass in the gas ($M_{\rm{O}}'$) in the six halo populations selected by their final masses at redshift $z\sim 0$ are shown. The corresponding strengths ($A_{\pm}$) and the durations ($\Delta \tau_{\pm}$) of the correlation are shown in the middle and bottom panels, respectively.} 
% \AP{has the sign convention for correlation with SFR changed? why is $\tau_+ < 0$ for the purple solid line here but $>0$ in fig 6?}\PV{Same sign convention but fig 6 is for the metal fraction while this is for the change in the amount of metal and this quantity gets affected by absorption of metals into the stars}}
\label{fig:dynam-correl-q0-dMOFoF-timeshift-func}
\end{figure}




\begin{figure}[htbp]
\centering
% \includegraphics[width=.49\linewidth]{plots/dynam_relxn/Spea_correl_vs_shift_betw_q0-Wind_FoF_fullcorr.pdf}
\includegraphics[width=.69\linewidth]{plots/dynam_relxn/shift_betw_multi-Wind_FoF_fullcorr.pdf}
% \includegraphics[width=.49\linewidth]{plots/dynam_relxn/Spea_correl_vs_shift_betw_q0(r)_in-Wind_FoF_fullcorr.pdf}
% \includegraphics[width=.49\linewidth]{plots/dynam_relxn/shift_betw_q0(r)_in-Wind_FoF_fullcorr.pdf}
% \includegraphics[width=.49\linewidth]{plots/dynam_relxn/Spea_correl_vs_shift_betw_q0(r)_out-Wind_FoF_fullcorr.pdf}
% \includegraphics[width=.49\linewidth]{plots/dynam_relxn/shift_betw_q0(r)_out-Wind_FoF_fullcorr.pdf}
% \includegraphics[width=.49\linewidth]{plots/dynam_relxn/shift_betw_SFR-Wind_FoF_fullcorr.pdf}
\caption{In the top panel, correlation weighted mean time lags ($\tau_{\pm}$) between relaxation offset strengths ($Q_0, Q_0^{\rm{in}}, Q_0^{\rm{out}}$) and SFR against the mass in wind within the halo ($M_{\rm{Wind}}$) in the six halo populations selected by their final masses at redshift $z\sim 0$ are shown. The corresponding strengths ($A_{\pm}$) and the durations ($\Delta \tau_{\pm}$) of the correlation are shown in the middle and bottom panels, respectively.}
\label{fig:dynam-correl-q0-Mwind-timeshift-func}
\end{figure}


% \begin{figure}[htbp]
% \centering
% \includegraphics[width=.79\linewidth]{plots/dynam_relxn/shift_betw_multi-SFR_fullcorr.pdf}
% \includegraphics[width=.79\linewidth]{plots/dynam_relxn/shift_betw_multi-dM(O)_FoF_fullcorr.pdf}
% \caption{Correlation weighted mean time lag between model relaxation offset parameters $Q_0, Q_0^{\rm{in}}, Q_0^{\rm{out}}$ and other halo properties. Halo samples are selected by their final masses at redshift $z\sim 0$. Colour coding follows \figref{fig:evolution-hal-gal-props}. \PV{Merged plots} \AP{this is impossible to decipher. there are too many curves and the labels are illegible.}}
% \label{fig:dynam-correl-mainall-timeshift-func}
% \end{figure}

% \begin{figure}[htbp]
% \centering
% \includegraphics[width=.79\linewidth]{plots/dynam_relxn/shift_betw_multi-Z(O)_SFreg_fullcorr.pdf}
% \includegraphics[width=.79\linewidth]{plots/dynam_relxn/shift_betw_multi-Wind_FoF_fullcorr.pdf}
% \caption{Correlation weighted mean time lag between model relaxation offset parameters $Q_0, Q_0^{\rm{in}}, Q_0^{\rm{out}}$ and other halo properties. Halo samples are selected by their final masses at redshift $z\sim 0$. Colour coding follows \figref{fig:evolution-hal-gal-props}. \PV{Merged plots} \AP{this is impossible to decipher. there are too many curves and the labels are illegible.}}
% \label{fig:dynam-correl-mainall-timeshift-func}
% \end{figure}


% Key results q0 inner and outer are poorly correlated. However, the mean q0 is strongly correlated with qy. q0 and SFR show an interesting correlation that is more seen in inner q0 but not in outer q0. Correlation weighted time lag.

% \begin{itemize}
%     \item Correlation between metallicity and q0 outer halo shows a clear time offset in haloes of mass $\sim 10^{13.5} \Mh$.
% \end{itemize}

% \begin{table}[htbp]
% \centering
% \begin{tabular}{c|c|c|c}%
% \hline
% quantity & other quantity & halo mass & observation\\
% \hline &&&\\
% Z(O) & SFR & $10^{14}$ & SFR correlates strongly with metallicity  \\
%  & & to $10^{11.5}$ & especially at a slightly later time. (see figures \ref{fig:dynam-correl-sfr-ZOsfr-img},\ref{fig:dynam-correl-sfr-ZOsfr-timeshift-func-all}) \\
% Z(O) & q0 & $10^{14}$ & slightly negative correlation\\
% & & $10^{13.5}$ & \\
% \hline 
% \end{tabular}
% \caption{Results}
% \label{tab:results}
% \end{table}

% \subsection{Correlation between star formation rate and metallicity}

% \begin{figure}[htbp]
% \centering
% \includegraphics[width=.49\linewidth]{plots/dynam_relxn/Spea_correl_vs_shift_betw_SFR-Wind_sbhl_timecorr.pdf}
% \includegraphics[width=.49\linewidth]{plots/dynam_relxn/Spea_correl_vs_shift_betw_SFR-Wind_sbhl_fullcorr.pdf}
% \caption{Correlation between star formation rate and gas mass in the wind for haloes selected by their final masses at redshift $z\sim 0$.}
% \label{fig:dynam-correl-sfr-windsbhl-timeshift-func-all}
% \end{figure}







% Here we present the correlation between relaxation parameter $Q_0$ and star formation rate for haloes selected by their mass at redshift $z\sim 0$. 
% \begin{itemize}
%     \item We find that the inner q0 shows a good correlation with the increase in the total oxygen in the gas content of the halo and the star formation rate (SFR).
%     \item Meanwhile, outer q0 correlates with the metallicity of oxygen in the star-forming regions.
% \end{itemize}






\section{Conclusion}
\label{sec:conclusion}

This study explored the dynamical evolution of dark matter's response to galaxies within populations of haloes simulated using the IllustrisTNG cosmological volumes. By constructing a detailed population of haloes and tracing their evolutionary tracks, we characterized the relaxation response of these haloes. This was performed by comparing haloes from hydrodynamical simulations, which include subgrid prescriptions for various astrophysical processes, with corresponding haloes from gravity-only simulations. Using a catalogue of evolving matched haloes, we examined the correlation between relaxation quantities and other halo/galaxy properties to elucidate their roles in mediating the relaxation response.

Firstly, we find that the radially-dependent linear relaxation relation model proposed in our previous work is applicable even at earlier redshifts, at least from redshift $z=5$. In this work, we have primarily studied the offset parameter $q_0$ in the relaxation relation that characterizes the amount of relaxation of the dark matter shells with no change in the enclosed mass. In a given population of haloes selected by their final mass, this offset is on average stronger during the peak star formation among those haloes. 
Our findings reveal that star formation activity significantly influences the offset in the halo relaxation response  over the entire evolutionary history of the haloes. While this connection with SFR is immediate on the relaxation in the inner haloes, it is seen 2 to 3 billion years later on average in the outer regions of both Milky Way scale haloes and halo groups. %\AP{refer to typical lags mentioned in the prev section and compare with literature on SFR-metallicity.} 
%
We also found that simple tracers of the stellar feedback processes through metal content only show a weaker connection with the relaxation than SFR itself. However, the wind accumulated from various feedback processes did have a stronger connection with the relaxation.% While we expected that a  if the feedback processes traced by metals alone does not fully account for this offset. Notably, we observed a stronger correlation between the relaxation offset and the overall wind in the halo produced by various feedback processes. 

These insights enhance our understanding of the mechanisms driving halo relaxation and contribute to the development of more accurate models of halo profiles in baryonification procedures. 
% This has an important application barynification procedures and semi-analytic galaxy formation models. 
For example, the knowledge of time lags can, in princple, allow modelling the observed dark matter distribution in large surveys such as Euclid with fewer parameters by exploiting correlations across different redshift bins.
And in semi-analytic galaxy formation models, this %will
may allow simple time-dependent transformation procedures to incorporate the dynamical evolution of the host dark halo with galaxy evolution, which is typically ignored. In the future, the relaxation response of dark matter haloes can also serve as a probe into the evolutionary history of the galaxies they host.









% \bibliography{references,references_new}
% \newpage
% \appendix
% Appendix if anything requested by referee
% \section{Correlation between relaxation characteristic parameters}
% \label{sec:appen-q0qy}
% Here, we present the correlation between the model-dependent relaxation offset parameter $Q_0 \equiv -q_0$ and the model-independent relaxation offset parameter $Q_y \equiv -q_y$ for haloes selected by their final masses at redshift $z\sim 0$. We find that there is a strong positive correlation between them at all halo masses (see 
% \figref{fig:dynam-correl-q0-qy-timeshift-func}). Note that for small shifts in either direction, there is a steep dip in the correlation; this is also seen in the auto-correlation function of both $Q_y$ and $Q_0$.
% Also we find that the relaxation offset in the inner and outer halo are poorly correlated especially for the high mass haloes. 


% \begin{figure}[htbp]
% \centering
% \includegraphics[width=.49\linewidth]{plots/dynam_relxn/Spea_correl_vs_shift_betw_q0-qy_timecorr.pdf}
% % \includegraphics[width=.49\linewidth]{plots/dynam_relxn/shift_betw_q0-qy_timecorr.pdf}
% \includegraphics[width=.49\linewidth]{plots/dynam_relxn/Spea_correl_vs_shift_betw_q0(r)_in-qy_fullcorr.pdf}
% % \includegraphics[width=.49\linewidth]{plots/dynam_relxn/shift_betw_q0(r)_in-qy_fullcorr.pdf}
% \includegraphics[width=.49\linewidth]{plots/dynam_relxn/Spea_correl_vs_shift_betw_q0(r)_out-qy_fullcorr.pdf}
% % \includegraphics[width=.49\linewidth]{plots/dynam_relxn/shift_betw_q0(r)_out-qy_fullcorr.pdf}
% \includegraphics[width=.49\linewidth]{plots/dynam_relxn/Spea_correl_vs_shift_betw_q0(r)_in-q0(r)_out_fullcorr.pdf}
% % \includegraphics[width=.49\linewidth]{plots/dynam_relxn/shift_betw_q0(r)_in-q0(r)_out_fullcorr.pdf}
% % \includegraphics[width=.49\linewidth]{plots/dynam_relxn/Spea_correl_vs_shift_betw_q0-qy_halhalcorr.pdf}
% \caption{Correlation between model dependent relaxation offset parameter $Q_0$ and the model-independent relaxation offset parameter $Q_y$ for haloes selected by their final masses at redshift $z\sim 0$.\PV{Will remove vertical lines}}
% \label{fig:dynam-correl-q0-qy-timeshift-func}
% \end{figure}
















% \end{document} % 

\input{Chapters/self_simil.tex} % 

%\input{Chapters/Chapter6} % Results and Discussion

\chapter{Conclusion}
\label{chap:conclusion}


% \chapter{Conclusion}

\section*{Summary of Key Findings}

This thesis presents a detailed investigation into the dynamical evolution of dark matter haloes in response to galaxy formation and evolution, utilizing both cosmological hydrodynamical simulations and a novel self-similar model. The core findings of this research provide significant insights into the complex interplay between baryonic processes and dark matter dynamics.

\subsection*{Dark Matter Relaxation in Simulations}
We have explored in detail the response of the dark matter content of a halo to the galaxy and gas it hosts. Understanding and accurately modelling this response is important for a number of applications including baryonification schemes for small-scale power spectrum emulation, rotation curve modelling, %
constraining the nature of dark matter using inner halo mass profiles, etc. 

Using haloes and galaxies identified in the IllustrisTNG and EAGLE simulations and matched to their gravity-only counterparts, our analysis demonstrates that the simplified analytical schemes used thus far to model the dark matter response \citep[e.g.,][]{1986ApJ...301...27B,2010MNRAS.407..435A,2015JCAP...12..049S} are inadequate in describing its detailed behaviour across a variety of halo and galaxy types. Specifically, we showed that the dark matter response, or relaxation relation (see equation~\ref{eq:qAR}), which connects the relaxation ratio $r_f/r_i$ to the mass ratio $M_i/M_f$ between unrelaxed (gravity-only) and relaxed (hydrodynamical) haloes, explicitly depends on halo-centric distance $r_f$ in the relaxed halo, in addition to being sensitive to a number of halo and galaxy properties including halo mass, halo concentration, stellar and gas mass fraction, and specific star formation rate. These effects, especially the dependence on halo-centric distance, have been typically neglected by existing quasi-adiabatic relaxation models. 

We presented a simple, physically motivated extension (equation~\ref{eq:chi-linear-q0}) of the existing models which accurately captures the dark matter response over 4 orders of magnitude in halo mass ($10^{10}\lesssim M/(\Mh)\lesssim 10^{14}$) and $\sim2$ orders of magnitude in relative halo-centric distance ($0.02\lesssim r_f/R_{\rm vir}\leq1$). Apart from an explicit radial dependence of the relaxation relation (e.g., equation~\ref{eq:q3-model} for low-mass haloes), a second novelty of our model is the inclusion of a parameter $q_0$ which characterises feedback-induced offsets seen in the relaxation relation measured in IllustrisTNG and EAGLE haloes in which, e.g., shells that do not show an overall change in radius ($r_f/r_i\simeq1$) nevertheless have $M_i/M_f>1$ (indicating loss of baryonic material). The existing quasi-adiabatic relaxation models do not allow for the existence of such shells, which are however captured well by our new null-offset parameter $q_0$ (see \secref{subsubsec:sim-relax} for a detailed discussion).
We argued that our results could have a significant impact on the applications listed above.

One of the key parameters, the relaxation offset parameter \( q_0 \), quantifies the excess relaxation of dark matter shells. We found that \( q_0 \) is typically stronger following the peak star formation epoch for a given population of haloes, indicating a strong correlation between star formation activity and halo relaxation.

\subsection*{Influence of Astrophysical Processes}
We investigated the influence of astrophysical modeling on the relaxation response of dark matter haloes at different epochs, specifically focusing on \( z=0 \) and \( z=1 \). The analysis is divided into three main parts, each shedding light on the role of various astrophysical processes in shaping the dark matter content of haloes.

\subsubsection*{Early Epoch in IllustrisTNG Simulations}
We began by examining the relaxation response at an earlier redshift (\( z=1 \)) using the IllustrisTNG simulations. Our study reveals that dark matter relaxation tends to be usually stronger at the earlier epoch compared to the present. We assess this using three distinct set of halo samples at \(z=1\), which highlight the variations in relaxation across different halo masses. Notably, we observe that cluster-scale haloes at \( z=1 \) show significant relaxation that becomes a function of the change in enclosed mass, in contrast to similar haloes at the present epoch.

We find that the relaxation relation at this earlier epoch can be described using the same locally linear quasi-adaiabtic model built on the analysis at \(z=0\), demonstrating the robustness of this approach in capturing the dark matter response across redshifts. Moreover, the parameters of the radially dependent relaxation are found to be more universal across a much wider range of masses at \( z=1 \). For example, the progenitors of even the most massive clusters are well characterized by the simple three-parameter model of relaxation that was developed with a focus on galactic-scale haloes at \( z=0 \).

\subsubsection*{Variation in Astrophysical Feedback Using CAMELS Simulations}
Next, we explore variations in astrophysical feedback strengths within the IllustrisTNG model using simulations from the CAMELS project, which varies four different feedback parameters: two for stellar feedback and two for AGN feedback. Our analysis shows that the parameters controlling the energy flux of the feedback have a significant impact on the relaxation of dark matter at different epochs. In contrast, the parameters governing the speed and burstiness of feedback have negligible effects on the halo relaxation response.

We find that variations in stellar feedback strengths have a larger impact among dwarf galaxy-scale haloes, while variations in AGN feedback parameters exert a stronger influence on Milky Way-scale haloes. Notably, the relaxation offset in the outer well-resolved regions is stronger at the present epoch than at \( z=1 \), contrasting with results from the inner regions explored in the IllustrisTNG simulations in the first part of this chapter.

The stronger implementation of AGN feedback tends to result in greater relaxation at both \( z=0 \) and \( z=1 \) in the outer regions of the haloes. However, in the slightly inner regions, stronger AGN feedback implementation leads to a weaker relaxation offset at \( z=0 \) and a stronger offset at \( z=1 \). We interpret this as a consequence of the overall reduction in total feedback at \( z=0 \) due to the suppression of star formation caused by higher AGN feedbacks in the past. These results highlight the significance of feedback mechanisms in building a physical understanding of dark matter halo relaxation.

\subsubsection*{Role of Astrophysical Models in the EAGLE Simulations}
Finally, we assessed the impact of different astrophysical models in the EAGLE simulations. Supernova feedback strengths show a similar trend to that observed in the CAMELS simulations. Additionally, we find that the gas equation of state has the strongest effect on the relaxation response of dark matter, particularly among haloes hosting dwarf galaxies.

Overall, this work underscores the intricate relationship between baryonic processes and dark matter halo relaxation, illustrating the variations that arise due to different astrophysical models and redshifts.

\subsection*{Causal Connections and Temporal Dynamics}
This study explored the dynamical evolution of dark matter's response to galaxies within populations of haloes simulated using the IllustrisTNG cosmological volumes. By constructing a detailed population of haloes and tracing their evolutionary tracks, we characterized the relaxation response of these haloes. This was performed by comparing haloes from hydrodynamical simulations, which include subgrid prescriptions for various astrophysical processes, with corresponding haloes from gravity-only simulations. Using a catalogue of evolving matched haloes, we examined the correlation between relaxation quantities and other halo/galaxy properties to elucidate their roles in mediating the relaxation response.

Firstly, we find that the radially-dependent linear relaxation relation model proposed in our previous work is applicable even at earlier redshifts, at least from redshift $z=5$. In this work, we have primarily studied the offset parameter $q_0$ in the relaxation relation that characterizes the amount of relaxation of the dark matter shells with no change in the enclosed mass. In a given population of haloes selected by their final mass, this offset is on average stronger during the peak star formation among those haloes. 
Our findings reveal that star formation activity significantly influences the offset in the halo relaxation response  over the entire evolutionary history of the haloes. While this connection with SFR is immediate on the relaxation in the inner haloes, it is seen 2 to 3 billion years later on average in the outer regions of both Milky Way scale haloes and halo groups. 
We also found that simple tracers of the stellar feedback processes through metal content only show a weaker connection with the relaxation than SFR itself. However, the wind accumulated from various feedback processes did have a stronger connection with the relaxation.

These insights enhance our understanding of the mechanisms driving halo relaxation and contribute to the development of more accurate models of halo profiles in baryonification procedures. 
For example, the knowledge of time lags can, in princple, allow modelling the observed dark matter distribution in large surveys such as Euclid with fewer parameters by exploiting correlations across different redshift bins.
And in semi-analytic galaxy formation models, this %will
may allow simple time-dependent transformation procedures to incorporate the dynamical evolution of the host dark halo with galaxy evolution, which is typically ignored. In the future, the relaxation response of dark matter haloes can also serve as a probe into the evolutionary history of the galaxies they host.

\subsection*{Development of a Self-Similar Model}
To complement the insights from hydrodynamical simulations, we developed a spherical self-similar model for galaxy formation. This model simultaneously and self-consistently solves for the evolution of gas and dark matter, producing pseudo galaxy disks within haloes. The iterative method introduced in this model allows for the study of quasi-adiabatic relaxation responses, aligning well with findings from full non-linear simulations.

By systematically varying parameters such as the accretion rate and gas equation of state, we explored their effects on the relaxation response. Our results showed that the accretion rate and the gas equation of state significantly influence the relaxation relation, while other parameters, like the cooling rate, have a minor effect. These findings offer a deeper understanding of the sensitivity of dark matter profiles to various astrophysical processes.

\section*{Applications and Broader Relevance}
The findings from this thesis have substantial implications for modeling dark matter halo profiles in astrophysical and cosmological contexts. The relaxation relations and timescales we derived can enhance the accuracy of baryonification schemes and semi-analytical galaxy formation models. For instance, understanding the time lags between star formation and halo relaxation can refine models of observed dark matter distributions in large-scale surveys like Euclid, potentially reducing the number of required parameters.

Moreover, the self-similar model developed here provides a computationally efficient framework for exploring the coupled impacts of multiple astrophysical processes on dark matter profiles. This approach is particularly valuable for studying the effects of baryons on rotation curves and gravitational lensing signals, and for improving cosmological parameter inference through emulators.



\section*{Concluding Remarks}
This thesis has elucidated the intricate relationship between baryonic astrophysical processes and dark matter dynamics, offering new insights into the mechanisms driving halo relaxation response. We find that the relaxation response is a dynamical process with a considerable amount of associated timescales. Hence the dark matter at different halo-centric distances take different amount of time to respond to changes in the baryonic distribution caused by the galactic astrophysical processes. A better model 

\section*{Future Directions}
While this thesis has made significant strides in understanding dark matter halo relaxation, several avenues for future research remain open. We plan to construct a unified model of relaxation that can predict the relaxation response of the dark matter halo as a function of the time-dependent evolution of the baryonic mass profile. In this regard we would primarily extend our time-correlation analysis to segregate the backreaction of astrophysical processes on the dark matter haloes into two parts. Firstly, the effect of those processes on the change in the baryonic mass profile and the second part is the effect of changing baryonic mass profile on the evolution of dark matter profile.

% but use more direct probes of the baryonic distribution rather than the 

Furthermore, exploring the potential of the self-similar model to yield straightforward analytical relations could benefit the development of quantitative models for relaxation. Incorporating the effects of star formation and central black hole development, along with their associated feedback mechanisms, into the self-similar model is a promising next step. Additionally, refining the model to more accurately replicate NFW density profiles in the virial region and incorporating more realistic cosmological mass accretion histories and major merger events will further enhance its applicability.
 % Conclusion

%% ----------------------------------------------------------------
% Now begin the Appendices, including them as separate files

\addtocontents{toc}{\vspace{2em}} % Add a gap in the Contents, for aesthetics

\appendix % Cue to tell LaTeX that the following 'chapters' are Appendices

% \input{Appendices/AppendixA}	% Appendix Title

%\input{Appendices/AppendixB} % Appendix Title

%\input{Appendices/AppendixC} % Appendix Title

\addtocontents{toc}{\vspace{2em}}  % Add a gap in the Contents, for aesthetics
\backmatter

%% ----------------------------------------------------------------
\label{Bibliography}
\lhead{\emph{Bibliography}}  % Change the left side page header to "Bibliography"
\bibliographystyle{unsrtnat}  % Use the "unsrtnat" BibTeX style for formatting the Bibliography
\bibliography{Bibliography,references-AdiabRelxn1,references-SelfSimRelxn}  % The references (bibliography) information are stored in the file named "Bibliography.bib"
% \bibliography{references-AdiabRelxn1}
% \bibliography{references-SelfSimRelxn}
\end{document}  % The End
%% ----------------------------------------------------------------